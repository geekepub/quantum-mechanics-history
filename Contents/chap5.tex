\chapter{原子核的内部}

§66开场白

原子、分子.晶体..…现在又该讲什么呢?
益子力学将领着我们去游历原子核的深处。那里还有许多奇迹将被揭示出来呢。
在二十年代,谁也想象不出,对原子核的探索能搞出些什么名堂。但物理学家的好奇心总是很重的,而原子核又蕴藏滔许多奥秘。在那些年代电,卧子力学正在庆贺自己在原子战斗中的最初的胜利,可是原子的核心却仍然深深地隐藏在五里雾中。
这里我们先谈谈,当时有关原子核的点滴知识究竟包括些什么。在十九世纪的最末年代里气法国人贝克勒耳十分偶然地发现某些物质能使照相底片感光。接着居里夫妇发现:周期表末端的三种化学元素一一-铺、年,,、铀都具有这种特性。
这种现象就叫作放射性。在那些日子里,理论家因为经典物理学不能对此作出解释而感到很尴尬。与此同时,有关神秘的放射现象的新事实却逐渐积累起来。这些事实表明,

*1B96年。一一译者注

•155•
  
放射线有三种类型:a、队和丫射线。
进一步的研究证实:"射线是由带正电的粒子构成的。
一个O粒子的电荷是电子的两倍,其质趾约为氢原子的四倍。
P射线与电子束没有区别。了射线,物理家学说,是一种极硬的电磁辐射,它的穿透能力比记录的保持者一一X射线还要大许多倍。
又过几年以后,英国物理学家卢瑟福与他的学生玻尔合作,详尽地描述了原子的行星模型:在原子中,电子就像行星一样围绕着它们的太阳——原子核旋转。这样越来越清楚:放射现象起源于核。
对于a粒子来说,这一点从开始起就是显而易见的:在原子中,除了包含几乎原子的全部质量的核以外,a粒子再找不到其他任何容身之地了。另一方面,电子却存在于核外的原子壳层之中。同时光子(电磁能鼠子)也经常地从这些壳层中飞出。可能P和了射线就产生于原子的电磁结构。
否。这分朗是不可能的。当原子放射P射线时,它并未离子化,它井未获得电荷。这就是说它的电子结构保持不变。电子在壳层间跳跃产生可见光和X射线光子;进一步计算这些光子对应着的能量表明:这样的能最只占r光子能量的一小部分。这进一步支持了如下的理论:fl和丫射线的放射起源于原子核内部。
几年以后,卢瑟福又给理论物理学家一些新的供入思考的食点在擂的G射线经过的路径里,卢瑟福放置一些氮原子核。记录叶立子与氮核撞击的熙相底片呈现了氧核的粮迹1

•lS'l•
  
炼金术士的梦想实现了:化学元素的转化已经发生,尽管采用的不是化学方法。
就在这一年,卢瑟福观察第一个核转变的同时,发现一种化学元素的原子核能够具有不同的质屋。计痹表明:这些不同质掀间的差数与一个氢原子的质蜇十分接近或为它的倍数。这些核就叫作同位素。

§67第一.步

放射性一一核转变——同位素。在建立原子核理论方面,现在该是采取笫一个步骤的时侯了。作为出发点的事实俱在,而且我们还有已经显露头角的忧子力学。
但理论家并不着恁。他们依旧件立在原始森林的边缘,倾听着它的喃喃私语,贪婪地嗅着它的芬芳,可是却害怕深入林间。把自己的新生婴儿一一屋子力学托付给这粗旷而陌生的环境。他们还没有这样的精神准备。
他们要求实验家先打开一条伸进林间的逍路。而这个,实验家不久就做到了:1932年英国人查特威克发现了中子。现在理论家就可以狩手干了。
一个基本事实还不清楚:原子的核倒底包括些什么粒子?事实很清楚,原子的核是一个复合的核。对此放射现象作出很好的说明:粒子从核飞出,但核依然存在。顺便提一下,一种粒子一一质子的存在,已确证无疑。
现在我们可以这样推测:核是由放射性衰变过程中出现

•158•
  
的那些粒子构成的。但这一设想还是太简单了。a粒子似乎具有与氮核一样的特性。还有比它更轻的核——氢核。因此,氢核应该是核大厦中的最小建筑石块。由于它是最基本的粒子,它获得了一个希腊名称proton*,即质子。
现在我们可以开始建造核模型了。我们必须考虑到这个基本原则:核的电荷必须与外部结构中电子的总负荷相等,但具有反相的符号一一正号。假如不是这样,原子就不会象现在那样是中性的。我们还知道核的质量:它粗咯地等于原子的质晟减去电子壳层的质量。
根据原子是中性的原则,曾假设:原子核由质子和电子构成;氢核有一个质子,但却没有电子;氮核有四个质子和两个电子,它的电荷是+4-2=+2,它的质批稍许大于氢核的匹倍。(我们知道,与质子相较,电子几乎是没有重福的,一一大约轻2,000倍!)按此假设继续下去:锥质凰为7,电荷+3,则认为锥核由7个质子和4个电子构成;硐核质垦为11,电荷+5,则认为硐核由11个质子和6个电子构成;氮核(分别为14,+7)是14个质子和7个电子;氧(16,+8)是16个质子和8个电子。余此类推。
一切都好象在正常地进行着。但这仅仅是好象如此。如果我们的建筑活动只局限于轻核,一切都还顺利。可是我们一且进入中型或大型结构领域中,这种一致性就会失效。你自己去看看吧。铁核的质量是56(更确切些它应被叫作核

*proton系由希腊文protos(意思是第一)与字尾-on(意思是最终的拉子)合成的。一一译者注

•159•
  
的质蜇数,因为它表明这个核的质量比质子的质屈大多少倍),电荷+26。这就要求有56个质子和30个电子;而铀核质量数是238,电荷+92,就应当有238个质子和146个电子。
这样看来,正如预料的那祥,大自然给每个新核增桥几个而不是一个质子。如果我们拒绝这个观点,说明核质量及电荷将遇到困难,而核建造的规则性也将因此而失效,同时还很难设想同位素是从哪里来的。可是从一开始,上述假设在有关核自旋的问题上就不能自圆其说。核的全部自旋必须等于构成核的各个粒子自旋的总和。例如,如果重氢(氝)的核确实是根据上述假设由两个质子和一个电子构成的,整个的自旋至少等于三个质子的自旋(质子和电子的自旋相等)。可是巾实上它只等于两个质子的自旋!这个分歧不是饥核独有的。总地来说,核自旋的计算值(根据上述假设得出的)与测蜇值是极少符合的。看来我们建造核的假设有毛病。
确实如此。核电子的职能是减少核的正电荷,使其符合于实验中观测到的数值。但它们还有一个更重要的职能。质子由于携带相同的电荷彼此排斥,正如原子的壳层中的屯子一样。这里须要电子将质子拴在一起。
简单的计算表明:核实际需要的电子粘合剂,要比用我们的建造方法所得到为多得多。另一方面,还有否认核电子存在的种种更有说服力的理由。这些我们将留在以后讲。
就在这时理论家对核是否由质子和屯子构成一事产生严重怀疑。跟着中子就出现了。就在中子被发现的同一年一一

•160•
  
1932年,海森堡和苏联物理学家伊凡宁科和塔姆提出了有力的、建立在数学基础上的假设:核完全是由质子和中子构成的。笫一步就这样完成了。

§68第二步

自然界在建造原子核的过程中,正如在建造电子壳层时那样,是很懂得节约的。唯一的区别在于在建造核时,它使用了两种石块-~质子和中子。
每次增添一个新质子时,自然界都力求原子核不致因质子间积累起来的相互排斥而土崩瓦解。在轻元素中——差不多一直到钙,第20号-—-核内的质子数和中子数大约各占一半。钙以后,中子数增加得比质子快。这种情况一直延续下去,而且越往后差别越大。铀核的质量数为238,质子92,中子竟达146个。
鉴于核结构已经巩固,自然界便开始变换其建筑手法:这里添几个中子,那里拿走几个。结果就产生了同位素,即同一元素的若干变种。例如锡的核就有十种稳定的同位素。
这样便一望而知,海森堡-伊凡宁科-塔姆假设是与核的质量及电荷数值绝好地吻合的。根据这个原理,氢核是由一个质子构成,质量数为4的氮核(氮4)由两个质子和两个中子构成,令里一7核有3个质子和4个中子没护11核有5个质子和6个中子,氮-14核由7个质子和7个中子构成,氧-16核

•161•
  
山8个质子和8个中子构成,如此等等。
这样一直进行下去,直到周期表的末端。
但我们对于中子又知道些什么呢?这个粒子的质昼几乎准确地与质子相等,而且名符扎实地是中性的,也就是说匕不带屯。
可是它义打什么权力占据着核里电子的位置呢?电子至少能将质子拴在一起。一个不带电的中子又如何能做到这点呢?
此刻我们发现:屯吸引力不足以说明核的稳定性。核真是个死硬分子。企图用化学方法,用高压或高温,或者用极强的电场来分裂核都从来没有获得成功。虽然用这样的武器来对付核外的电子结构却是卓有成效的。
因此,物理学家作出这样的结论:中子存在J一核内是有一定的逍理的。因此,也正是中子起若将核的质子结合在一起的胶接剂的作用。但我们不禁又间:中子凭借的是什么力量?不可能是电力,因为中子是中性的。
理论家不得不着手研究这个问题3在中子发现两年以后,塔姆和日本物理学家汤川秀树提出了一个杰出的理论:存在着一种非常强大的特殊的核力,也就是交换引力;这种力在极短的距离内作用于质子与中子之间。
交换力?这是个熟悉的名词。正是这种交换力将氢原子、氮原子、氧原子以及其他许多原子结合成相当粕定的分子。在这些分子中,原子之间不断地交换着它们的电子,从而使原子与原子紧密地结合在一起。

•162•
  
但似子核内又是一种什么性质的交换呢?质子和中子是附种不同的粒子3核不具有任何电子。质子和中子之间能交换什么呢?塔姆所作的计筛表明:对核粒子来说,电子交换所产生的内聚力是太小了。
但找们面前摆看两种选择:要么后退,并且把交换力作为错汉的概念而摒弃;要么勇往直前,并且声阴;尽管质子与中子表面上不相似,它们实际上相差无几,并且具有许多共同之处,因此它们之间能彼此转变:质子转变为中子,中子转变为质子·.)
这一思想是十分大胆的。1934年当这个假说被提出时,物质的从本要素之间的相互转变还没有被观察到。当然,在此两年以前,一个电子和一个正电子转变成一对r光子的事头已被彴定,但这一现象具有截然不同的性质。
物理学家进一步推理,认为如果两种粒子能彼此转变,它们应该在转变过程中交换某种东西。质子获得了这一某种东西转变成中子;中子失去这个东西转变成质子。与此相反,也还可能存在着另一种类型的交换:中子获得某种东西,而质子却失去了它。
核结构是极稳定的,质子与中子间的交换力,必须在粒子间极短的距离内起作用。从这一事实出发,汤川秀树为这个物质粒子—一这个某种东西——么]绘出一个肖象。它既可带正电,亦可带负电,带电批与一个质子的电荷(或电子的电荷)数值相等,它的质量比电子质量大200到300倍。
质子和中子的质量大约比电子大1800倍,而这个神秘的

•163.•
  
粒子所具的质岳介乎二者之间,因此获得了介子*这个名称。
这样我们便获得了下面这个核交换图画。质子发射一个正介子,因而失去了电荷而转变成一个中子;中子获得这个正介子而变成一个质子。但也还存在着另一种转变方式:一个中子借发射一个负介子而转变成一个质子;质子俘获这个负介子后转变成一个中子。

§69寻找这个神秘的介子

可是这些介子究竟在那儿呢?对放射性核又反覆进行实验。答案是个悲不含糊的否字!即使介子真的存在于核内,它们也绝不会离开故乡。看来介子宁愿坚持自己平凡而重要的工作也不自抛头露面。
于是物理学家又转而求助于那个有关核粒子的重要情报来原—-宇宙射线。在一年的时间里,介子被发现了。与汤川秀树的计算相符,介子的质低大约为电子的200倍。
理论家可以庆贺一番了。关于质子和中子是相互联系着的粒子,这一令人惊异的大胆设想以及介子的发现都产生于笔端。这是物理学最最卓越的成就之一!
但胜利的它悦转瞬即近。介予拒绝与原子核发生任何接触,对中子也极为冷淡,只是在正帘的电相互作用范围内屈从于质子i们已。物现学家被井得晕头轧向了:它会是那个周旋

*介子nJCSO'I丛由彷腊义111csos,意即中间,与语尾-on,意即对f合成的。一一译趴E

•16·1•
  
于质子和中子之间并与它们发生最强的相互作用的粒子吗?显然,他们想,它肯定不会是自然界的那个奇迹。必须继续寻找。
这次自然界和科学家玩了很久的捉迷戟。有关核构近的杰出发现已经取门,释枚核能的秘密已被渴露,最早的原子反应址和原f弹已经制成,可是这个狡滑的粒子仍然在躲避着人们的探测。直到1947年,著名的宇宙射线研究者鲍威尔才将它捕获。
这个校子也炖一个介子,但与前者不同:它不是207电子质届,伽足273这次不会惜了。新的介子—一-叫冗介子,以区别广上面那个冷淡的,u介子--与核粒子起着强相互作用。如果它在飞行中具们可观的能垦,它甚至能够分裂它所撞上的核"
概栝1世说,有关核力产生于质子与中子间的介予交换这一扇子力学的攸设已被出色地证实。顺便提一下,物理学家在这间题上一五是如此地自信,以致他们坚持向核森林的最深处推进,即使当讨连有关所觅介子的存在也无丝亳证扣尥

§70最强大的力

物埋子豕立即杠f矶亢这个新发现的核力。首先他们注衍到:核力作用距离是极短的。这点我们巳经说过。分子中的沁奂力是在原子之间的距离内起作用,它的数量级与原子的线及相同一一一亿分之一附米。而核交换力的距离比这个数

•165•
  
值还要小几万倍。只是当粒子间的距离接近粒子本身的线度时,核力才开始起作用。这就清楚地说朗:核力只可能存在于核的内部,而绝不危在核外起作用。
核力足迄今发现的最强的力。核力不仅完全压倒质于间的相互排斥---这个力在那样近的距离内是非常巨大的-~而且还能将质子紧紧地推在一个稳定的结构中。
物理学彖对核力或核稳定性,正如他们对所有物体一分子、原子、核一样,是用结合能来描述的。所谓结合能就是这样的能届,它必须被赋与一个粒子集团以便将这个渠团分解成它的粒子成员3
很自然,这个集团里的粒子愈多,这一能量也就愈大。因此我们常取钧一粒子的结合能来描述该某团的稳定性。这个能量是用电子伏特这个特殊单位来屈度的。一个屯子伏特就是在电场中电子通过一伏特电位差所获得的能凰。在我们的宏观世界里,这个单位是很小的,但对于原子世界来说,它已是十分可观的了。
对许多物质来说,分子之间的纽带甚至在室温下就破裂,这样在正常条件下,这些物质就以气休形式存在。这种分子之间的结合能的数量级为每一分子百分之几电子伏特。
将这些分子再分解成单个的原子,所需能橇则大得多,大约每一原子IO电子伏特。如果变换成温度的升高,这数值是惊人的:它相当于几千甚至几万度。
将原子的外围电子和核分隔开来则更加困难。我们知逍,原子中的电子根据它们与核的不同联系而具有不同的能蜇。

•166•
  








  

,
 
理学家不得不向地质学家、天文学家、甚至生物学家请教。显然,某元素的丰度对应着它的原子核在自然界中出现的频繁程度。这里自然界不光指地球,它指的是整个宇宙,即天义学家可用分光镜观测到的可见宇宙。
让我们比较一下这俩条曲线(图21)。它们之间有什么共同之处?首先,在左侧的角落我们看到上曲线的最高峰对应着氮4,碳12,氧16以及其他若干元索的核。所有这些数字都是4的倍数,似乎这些核包含的并不是单个的质子和中子,而直接就是“粒子。下曲线与之柜对应的部分,标示出这些元素在自然界中的相对上度接近100%。让我们沿着山脊继续走下去。我们发现上曲线中最显著的破折点与下曲线的高峰柜对应。一般说来,核愈稳定,它在自然界中的丰度愈大。
这便引导我们作出如下的结论:在原子核世界中,启然界建立了自己的自然选择规律。在生存竟争中,只有最强者能够生存。丰度最大的正是那些核,它们的中子和质子数为2、8、20,等等。其原因,下面讲到核壳层时,再来讨论。
这里,我们可以指出:把核说成是由“粒子构成的,也不完全正确。但有一件事是肯定的:在原子核世界中,由两个质子和两个中子构成的集团确实是非常稳定的。物理学家说,作用在这个数目的粒子之间的核力已达饱和。给这样的集团再增添一个额外的质子或中子是不可能的。例如氮核就拒绝接纳任何粒子进入其家庭。当然,在所有的核中,氮核是最不好客的:甚至连质量数为5的核(两个质子和三个中子,

•168•
  
或三个质子和两个中子)都不存在。
轼豖庭由于拒绝氏客而巩固了自身。如果把氢核(它只有一个队子,因此核内没有任何核力在起作用)除外,自然界中氮核就是最稳定的核。
饱和是核力所独行的一个新特性。核力的另一个新的不寻常的特性是它对电荷的独立性。核力对电荷根本不予理睬。无论是在一个质子和一个中子之间,还是在一对质子或一对中子之间,核力都同样好地起着作用。为什么会如此呢?直到今A物理学家也不很明白。

§71再谈核的稳定性

形成牢固的核结构的交换力,就是将质子与中子结合在一起的'j』(.但I贞f勹巾子能协作到什么地步?这里必定有个极限,不然这些粒子早就该融合成一体了。
当然,自然界是不会允许它们融合成一体的3核粒子在极近距离内相遇,除了强大的核引力而外,还有用以抵消前者的同样强大的不允许粒子彼此渗透的斥力。
这就是所谓核力作用距离的下限。我们巳经谈到了上限,那显然就是核粒子因彼此后退而可能拉开的最大距离,在此耟离以内核粒子仍然受到核力的约束。这个距离与核粒子自身的线度屈于同一个数量级。
这是个有趣的市实,因为它能说明结合能曲线的总趋势:结合能随社核质世数的增高而下降。真的,在具有很少的质

•169•
  
子与中千的轻核中,每个粒子都能凭借核力与其余粒子相联系。
饱和——它表明核力最喜欢结合四粒子组一究竟足怎么一回事呢?答案很简单。核粒子是彼此不可分割的,因此没有办法将那样的“四粒子”分离出来。例如在钠和氯离子的品体中,你试把对应原来的分子的原子对分离出来吧!氯化钠晶格中的这些钠离子和氯离子也能构成不同于晶体的NaCl分子,即原来意义上的分子,这点我们已经讲过了。
核具有的粒子愈多,它的体积自然就愈大己可是钰个粒子只能借核力与最邻近的粒子相联结户,因此在多粒子核巾,象一条锁链的结合便代替了所谓总休彴结合。这样的核将会失去稳定性,而结合愈象一条锁链,岱定性也就愈低,因为随着质子数目的增加,与核力起相反作用的质子间的斥力也会普通增加。
位于周期表末端的一些最大最重的核是颇不稳定的。自然界让它们自动地变得更为稳定。如果核能够甩掉一些多余的核粒子,正象船只抛掉一些玉舱物以保持浮性一样,这样的事便能够发生。核抛掷出去的多余的粒子就是放射线。
顺便说一下,你可能已经知道在周期表的始端和中部都有很多放射性核。当然,它们之中大多数不是自然界的产物而是人为的。物理学家用核粒子(主要是中子)来轰击本来是稳定的核,便能使核由于容纳粒子过多而失去平衡。
这些核要返回它们的稳定状态,但却不走回头路。此外,核的终极状态经常与初始状态不一样。当核增添了一个中子

•170•
  
后,它便安定不下来,其反应是抛射电子和了光子,直到它转化成一种截然不同的核。
这就叫人为放射现象。隐藏在它背后的实质就是:所有的核都不惜任何代价趋向稳定。不稳定的事物不能长久存在下去。回顷一下核在自然界中的丰度图。它清楚池表明:核愈是稳定,它存在得愈长久,因而这种元素的丰度也就愈大。

§72核内的隧道

一些非常复杂的定律支配翟核的稳定性。科学家研究这些定律巳三十多年,但宜到现在还没有充分认识它们。当然,其中有些定律的秘密正被揭露。
首先就足“放射性,或核的“衰变,它的发现甚至早于中子,尽管科学在当时对G粒子稳定性的内在原因还一无所知。
因此我们面前摆着两个问题:为什么“粒子会从核内飞出?为什么匝子和中子不单独地飞出?
让我们从第二个,也就是较难的一个问题开始吧。我们仔细考察结合能曲线后发现:拥有四粒子,即两个质子中子对的核(例如氮书碳12,氧16)要较它们邻近的核更为稳定。可是我们又发现,重放射性核正是借抛出这些四粒子而衰变的。我们应如何解释“粒子的这种爱憎兼有的品性呢?
当我们想起四粒子中的核力已达饱和因而不允许任何笫五个粒子进入其组织时,我们就感到更加困惑。我们不禁耍问:那些北氮核更重的核怎么还脂生存呢?

•171•
  
为了回答这些问题,让我们更仔细地看看叶立子是怎样生存彴,四粒子之间是怎样交换介子的。我们知道,可能的交换形八之中的一种就是;中子抛出一个负六介子而在这过程中转变为个朊子,与此同时,原米的质子吸收了这个介子,并在此一I时叫转变成一个巾子。
因此,平均起来,一个四粒子在任何时刻都有两个质子和两个中子。但设想一下:某个四粒子抛出的介子被邻近的一个四粒子中的质子俘获。这便一下子犯了两个罪:第一个四粒子籽要钉三个股子和一个巾子,而邻近的四粒子将要有三个中子和一个质子。为什么说是罪呢?泡利原理解释道:质子和中子具有与电子相同的自旋,因此加诸电子的种种戒律,质子和中子也必须遴循。{包利原理不允许一个以上的粒子在同一给定状态下具打同一意义的自旋。
"粒子是极其屁定的气这是因为它的两个质子和两个中子分别占有两个不同的能级--尽可能低的能级。两个质子处j二一个能级,而两个中子义处于另外一个能级。这所以可能是山于:在任何一瞬间,核内的质子和中子以不同的乔装出现,也就是说,它们实际上足不同的粒子。现在如果一个四粒子具付三个股子,则其中一个必须违反严格的泡利不相容原理,不然就必须去占领一个较高的拒址状态,也就是较低的结合能状态。
核粒子并不愿意犯罪;它们也不菩欢不稳定的状态。因此邻近的四粒子刚一l女主加五个介于,匕使立即将它释放出来。其结朵,这两个比邻的四粒千又恢复f它们的祁态。但两个

•172•
  
四粒子之同进行的这种瞬时的交换却在它们之间建立了相互联结的纽带。这些四粒子不再象先前那样彼此隔绝了。
跷Jf江核愈远,四书子产生的这种稳定作用愈弱。但重核又显示了四粒子的另一个明础的效应。由于核巳变得如此庞大,因此、芷如我们已纾指出的那样,这些核的外围粒子只能与它们的直接近邻发生相互作用。显然,郤近核的表面,粒子又重新结合成四粒子,因为这是最稳定的形式。
这很可能就是重核只抛射四粒子(a粒子)而不抛射质子或中子的原因。它们是怎样从核里逃出来的?核是粒子的有联系的党休,换旬话说,是一个用高位垒与自由粒子隔绝的位阱。我们知逍阱的深度(或垒的高度):它等于结合能。
我们欢刻就会明白,核垒与我们过去所说的位垒不同:跳过核垒不须作任何努力。核垒不是一个只有前墙而无后壁的台阶:核垒从象一堵围揣。这堵围墙虽然不厚,但很高。说得粔糙一点,围墙的厚度是由核力的作用距离来决定的,它的高度却表征核力的大小。
这里,监子力学又来管耳i了。量子力学说:放射性核抛射d粒子是一个隧道效应,这个效应与金属中电子经由隧道逸出,或半导体与绝缘体中的屯子经由隧道渗入传导带没有任何区别。任何时候波特性都在起作用:在一种场合下,是对电子而言的;在另一种情况下,是对叶立子而言的。
现在我们懂得四粒子的这种两面派行径了。事实上,也根本谈不到什么两面派:这一切都根源于以子几率。理论上,一个d粒子甚至能从一个氧核里飞出,但这个几率却小到可

•173•
  
以被忽略。在轻核中,对C粒子的飞逸而言,核垒的高度是非常大的(结合能非常大),在祖核中,核垒却很低(结合能小得多)。而隧道效应的几率在很大程度上是依赖于位垒的高度的:随竹高度的培加,几率将迅速下降9全部的秘密就在这里。
另一方面,重核对抛射叮立子的位垒,要较单个地释放质子和中子低得多。这忱是为什么只有四粒子而没有单个的粒千飞出的原因3

§73核是否也有壳层

与原子不同,核似乎没有一个中心体,在它的四周环绕着象屯子那/、的云层。在核的质子-中子结构发现后的若干年丸物理学家将核描绘成以质子和中子云的形式,大致均匀地弥漫于这微小空间里的核物质。
当然,核力饱和以及(t粒子蜕变现象的发现似乎表明:核物质不是倍无定形的,我们可以观察到叶立子的微小细胞的轮廓3当量子力学和实验继续深入核森林的时侯,越来越明显地石出,这里还有一丛丛的树木,而且棵棵树都有自己的形伏,不像从远处望去那样苍茫一片了。
我们知逍,由四粒子组成的粒子在核内居于最低能屎位罩,同时在所有的核粒子集团中它是最稳定的。对应这个位置的是一个单一的总能级,在这个总能级中,两个质子具有相反方向的自旋,两个中子也是如此。
在这个指定的核内,第二个四粒子占据另一个能级,第三

•Ii4•
  
个四粒子占据第三个能级,余此类推。随着四粒子数目的增加,核内翘来越高的能级被占满,很象原子中的电子那样。
m小人会说:并非所有的核都包括四粒子!确实如此。
这就总昧礼:在粒子数非四的倍数的核里,对应的佬级不会全被占扛沁
这杆的核非常象原子的某种外部结构,在这种结构中电子的最外壳层已经埂满,因而处于封闭而稳定的状态(回顾一下惰性气休讥这里也同样存在着被四粒子和大量其他核粒子拍满了的极为稳定的核壳层。
当然,外表的相似是不够的。我们盓要关于核壳层确实存在的页确凿的证据。让我们再来看看核稳定性和核丰度的曲线图。取几个最高峰为例,计筛一下与之相对应的核所包含的质子及中子数。
笫一个是氮4,它的核是一个由两个质子和两个中子构成的C粒子。下面是氧16,它包括8个质子和8个中子。接着是钙40,它包括20个质子和20个中子。如此进行下去。最后的高峰位于曲线的右端,它属于铅208,这个核具有82个质子和126个中子(除这些而外,我们还须添上具有50个质子的锡核。这个核是如此地稳定,因此自然界为它设计了10种稳定的同位素,与此相较,具有其他质子数的核只有2到5个私定的同位素)。
这样看来,最花定的核具有这样的质子-中子数:2,.8,20,50,82,126。请注意,这些核就像是惰性元素原子的翻版,这些原子的电子数为2,10,18,36,54,86。二者——各

•175•
  
在自己的世界内一都是稳定性的记录保持者。
这些质子和中子数被称为幻数。说来也真名不虚传,因为这些数字确实有某种神奇的力讨,不然为什么原子的核和它的电子壳层-一按照截然不同的规律生存着的两个世界一一都显示了结构上的稳定性?
当然,将幻数与最稳定的原子的电子数进行对比,便可看出二者之间存在看明显的分歧。这两组数字只是对氮——在两个世界中都是稳定性的记录保持者-—-才是吻合的。这两组数字出现分歧也并不是偶然的。相反地,如果两组数字完全吻合,那才过于奇怪了,因为核与外围电子云的生存条件是完全不同的。
但核里仍然存在着某些好象壳层的东西。实验也证明了这点。让我们行一看钾原子(第19)。它是单价的,也就是说它在惰性原子组叭满了的密封壳层外面还有一个电子。钾原子电子结构的全部自旋与这个价电子的自旋相等。这是很自然的,因为其他所有的电子的自旋都是成对的,同时由于每对电子的自旋方向相反,因而互相抵消,使整个自旋之和等于零。
现在将钾的电子结构与氧17同位素的核来进行比较:这个核除了四个四粒子而外,还有一个中子。于是我们应该预料到:氧核的自旋应与这一额外的中子的自旋相等。事实正是这样。
上面所谈到的吻合并非独一无二的。核自旋通过实验测出的数值,与建立在核壳层模型基础上的预见数值,是极好地符合的。
•l76•
  
§74y射线从何而来

当我们对第三种类郡的放射线一r射线的来源进行考杏时,电f亮层与核壳层的共同特征越来越多地被揭示出来。物理学家从对了射线的研究中建立了许多有关核家族寿命的重要巾实。
第一件引起我们注意的半便是核丫射线的光谱。这种光谱是由几条独立的线构成的。这意味若:核粒子只能具有非常严格地确定了的能扯,也就是说,核粒子必须存在于特定的状态中。粒子在这些状态间过渡便会产生'}'射线。
这些核的能级又是怎样的呢?这些能级又是怎样用核粒子装填起来的呢?这里核图几乎是一片空白。核具有确定的能级这一小实不应当引起任何惊讶。薛定谔方程所预见的这些能级,对任何有联系的粒子集团都适用,因而对核系统也自然适用。
在原子内部,描述粒子相互作用的公式是大家所熟知的库仑定律:电子间的相互排斥以及电子与核之间的相互吸引。因此,在原子范围内,我们可以将这个定律代入薛定谔方程。可是,核力的规律人们仍然弄不明白。
物理学家不得不掉过头来去解决一个相反的问题:观察r射线的光谱,并根据它来计符核内的能级以及粒子在其中的装填顺序。你还记得,物理学家在建立原子内的能级的时俟,也曾忙于解决上述问题。现在科学家迫切需要有关r射

•177•
  
线的各个光谱线的亮度及特征的情报,并希望能从中推导出一个支配核内粒子相互作用的定律。
可是事实证明核是一个不易摧垮的堡垒。甚至直到现在问题也没有完全解决。显然,在我们对核粒子的性质有所认,识以前,这个问题是不会得到彻底解决的。目前为探讨这个问题而采用的某些技术将在下章讨论。
尽管如此,有关核内存在能级,以及质子和中子几率云构成壳层的概念巳经结出丰硕果实Q这个概念使我们能够解释r射线的起源以及它所显示的许多有趣特征。
首先,为了发射一个r光子,核显然要从一个稳定的最低容许能态过渡到一个较高的能态。拿它与原子作个类比,核的后一状态也应叫作激发状态。丫光子一经发射,核就返回其初始状态或其他某个状态。
核力要较电力大几百万倍。由于这个缘故,核内能级之间的距离一般要比电子结构内的能级距离大得多。因此,可以很自然地预料:丫光子的能量也相应地比可见光光子的能量大得多。这意味着T光子的波长相应地短得多。这正是我们观察到的现象。在所有巳知辐射中,了射线具有最短的波长。
这就是为什么丫射线总是毫无例外地伴随着几乎所有的核放射性转变,因为这些转变无非是核向更稳定状态的过渡。靠抛射几个额外粒子来对核结构进行一次调整,有时还不足以导致完全的稳定。新核虽然比原来的核稳定,但仍处于激发状态。核结构重新组合的最后阶段,就是发射一个丫光子,
•178•
  
这样核便停止放射。
核也能通过一种电子壳层意想不到的途径,抛掉剩余能拯:核并不抛射Y光子,而偷偷地将其激发能批直接传递给电子云。但这个能点对原子大厦来说是太大了,这个礼品简直就像地震。当然,大厦仍然乾立着,但有些电子却被以极大的速良驱逐出境。这种现象非常成功地与丫射线的宜桵发射相颉顽,这就叫内轼换。

§75核象液滴吗

核记层,奇妙的核....真是标致得很啊!这幅图回虽然动人,但很多实验半实不能塞进壳层换型的框框。这并不足为怪。首先,如果核完层真的存在,它也必然与电子壳层迥然不同。这个核内的壳层概念本是挖空心思想出来的。核并没有一个被核粒子包围着的核心。此外,核这个满员了的集休,所容纳的粒子数也与原子外围的电子数十分不同。最后,核壳层应有两种类型:质子壳层和中子壳层。
当壳层这个术语从原子以外的世界挪用于核内世界时,它仅意昧若核粒子中的某些特定集团的某种孤立、稳定性与饱和状态。此外,这种状态也不是在任何时校、任何场合下郘会出现的。
认为芫层只能指包括很少的核粒子的轻核,是有些逍理
的。随召核的培大,各个能态失去了它的独立性,核在结构上越来越屁成一团。核粒千是那样多,核粒子云是那样地互相重

•1i9•
  
登,以致再也没行任何确定的粒子运动贞t好像它们郘已停止遵从匿子规律似的。
结果是,核失上了与原子相似的全部特征。壳层模型必须被放弃。但找们义能为核设计出一个什么样的新模型呢?
第二次世界大战爆发前不久,科学家曾暗示过核的液滴模型,至于根拈什么l甲巾,以陌再谈。核被描绘成一个外表均匀的实体,在实体里面并不存在任河行规则的结构(例如,"粒子或壳层)。单个的核粒子(核液的分子)被认为在液滴内处在经常的不规则运动状态之中。
结果是,核液获得了某种流动性。就像液滴一样,核具有边界,但这些边界是易动的,流畅的,并且能够由于各种外在和内在原因而改变形状。所有这些都不能使核的表面破裂:核液在液滴边界上的表面张力,能使核的表面保持原样。这种核表面张力与普通液体的表面张力完全是异曲同工的:核粒子由引力结合在一起,这个引力是不会被液滴外的任何其他力趾抵销的D核力将核液保持在这个液滴内。
它们的相似到此而止。让我们比较一下两种液体的密度。简单的计算表明:核粒子要比一般液体的分子紧凑几十亿倍。象从水龙头滴下的一滴水那样大小的核液滴足足重一于万吨!
真是大得凉人!我们知逍物体的性质在很大程度上取决于它的密度。将气体的密度改变一于倍,它就会变成一个遵循截然不同规律的品体。因此,一般溶液和核液之间显然没有任何内在的相似。二者的密度悬殊太远(相差几十亿倍),

•180•
  
而作用在核粒子之间的力根本不同于作用在分子之间的力。
但让我们观看一下外表吧,这里我们能够找到二者之间的类似。将一小珠水银放在一块玻璃板上并轻轻地敲击这个水银珠。它将会颤抖起来,细微的波浪将在它的表面散开。如果敲击得再重一些,这个水银珠将会分裂成若干更小的小珠。
这可能使人联想起当代物理学最重大的发现之一。1939年,科学界为一种耸人听闻的事件所震撼,在战前的那些日子里,这件事的凶险含意只有物理学家才能充分领会。那就是铀核裂变的发现。
不同国家的理论家,赶紧为原子核世界的这一惊人现象寻找解释。玻尔和苏联物理学家夫伦克耳迎头赶上。他们分别地提出了一个共同的理论。他们成功地用自己新发展起来的核液滴模型来解释铀核裂变。

§76液滴状的核分裂了

玻尔和夫伦克耳大致上是这样推理的:这里有一个处于正常状态下的核,核粒子的运动甚至有某些规律;如果核结构是稳定的,它的居民就应一直过着一个平静的、与世隔绝的生活。
于是,突然间,闯进了一个不速之客——一个不知从哪里来的粒子。它一头栽了进来,闹得大家坐立不安。在一片混乱的间好声中,这个核住宅竟成了个群魔殿。

•181•
  
顿时这个新粒子和其他粒子便宾主难分了。新粒子带进来的能晟立刻分给所有的核粒子,因此现在不管是新粒子,还是核的其他粒子,都不能浇开这个核。这样,一个新核便形成了c玻尔称之为复合核。
这一状态拧续不久。当其中一个粒子获得一个足够行力的砬撞时,它便翻越核边界的位垒而离开核。如果逸出的粒子不同十闯进的粒子,则审件的整个过程就叫作核反应。这个名称是恰罚的,因为开始的核已不同于最终的核。芷象在化于里一杆,开始的物质不同于化学反应中产生的物质3
复合核内粒子的一片骚乱,酷似分子在一滴液休内的无规则的热运;如单个的分子不时地从液滴里蒸发出去。这与下面的估况非常相似:当核受到外部粒子撞击而加热时,核用的粒子也能蒸发出来。
在这种悄况下核内真地发生些什么,谁也不很清楚。但我们能够说,从它的表面来罚,它真就像是一个热的液滴c让我们石-石液滴的表面吧。它始终处于激动状态之中:它颇抖访,当一个分子跑掉了,另一个分子立刻会占据它的位置。
人们早就观察到:液体表面振动幅度十分密切地依赖于液滴内液休的表面张力。我们巳经说过,核液滴的表面张力产生于核引力。核愈大愈重,核引力就愈弱,因而也就愈难将核粒子结合在一起。对于重核而言,甚至相对轻微的颠簸也能在它的表面上产生危险的振动。
这样的颠簸可以借一个中子撞击一个重而颇不稳定的铀核而产生(回忆一下:由于不稳定,这些核是放射性的)。有

•182•
  
时候,仅仅一点最轻傲的顺簸,例如热中子与之碰撞而产生的勋簸,也可以击碎一个袖235核,这里所说的热中子的能赞,娑狡原子核所特有的能员小几亿倍。
一滴水是怎样分裂的呢?高速电影拍摄术回答了这个网题J如果撞击得法的话,小水珠就会开始振动,巨浪就会出现在它的表面。接芢小水珠便伸延成细长形,最后终于在腰部陌开。
对水珠分裂的更复杂悄状也作了观察:这些水珠分裂成许多较小的水珠,它们的体积经常是参差不齐的。
玻尔和夫伦克耳的假设是:当中子撞击重而不稳定的核时,核入面的类似变形引起核裂变。

§77核裂变的秘密

但为什么中子的撞击能引起核裂变?重核为什么宁肯瓦解成儿大块,也不肯蒸发出单个的粒子,正象在轻核和中质昼核的人为放射中所见到的那样?
对第一个问题的回答是:将核与外部世界隔开的那堵围墙,正象我们曾经描述的那样,具有两个壁,这两个壁又是不对称的。
从墙内来看,核围墙的坡度对质子比对中子更为缓和。
围墙的高度是核力造成的。但对质子来说,由于它们之间的相互排斥,这高度降低了。由于位垒的存在,在正常情况下粒于不离开核。这个核是相对地稳定的。

•18~'f;,
  
从墙外来看,围墙就有些两样了。对质子来说,位垒依然存在,这是因为核的质子共同排斥所有的与它们同类的不速之客。对中子来说,外位垒不复存在了,因为中子在电上是中忙的。另一方面,核内却存在着一个中子可以坠入其中的位阱-—-当中子坠入核内时,它往往就杲在那里。
因此,如果质子要想进入核内,尤其是进入重的多质子核匠它就不得不拥有向达儿百兆电子伏特的巨大能量。中子却不须任何能星。这便说明:为什么能量很低的中子(甚至能量只达百分之几电子伏特的热中子)也能进入核内。
现在我们能够回答笫二个问题了。有人可能以为中子进人铀235核内会使它的负载增大到这样的程度,以致它会崩裂成碎片。巾实上这个中子并不是能够定局的一着棋。这个核在不危及其稳定性的情况下,能够容纳三个额外的中子,以形成铀238的核。
结果怎样呢?新的中子既不能使核的负载过重,又不能阴显地增加它的能量,也不能给核液滴带来某种刺激。那么铀235的裂变是怎样产生的呢?
情况是真够复杂的。这里我们又一次遇见了那些星子。
关键是:只有当中子具有一个非常确定的能量时,铀235的核才能裂变e这个确定的能量对应着铀235核的稳定状态能级与紧挨着的激发状态能级之间的距离。也就是说,如果中子的能量对应着上述两种状态的能噩差,它便能最有效地激发铀核。
铀215核的激发状态和稳定状态之间的能量差是非常小

•184•
  
的。看起来这个核一旦进入激发状态,就会象轻核似池动作起来:发射一个丫光子和某一粒子,然后返回原来的或某种其他的稳定状态。巾实却不然。
原因是这样的。我们说过,重核宁愿抛射G粒子(四粒子组),不愿抛射单个粒子。这是由于:对“粒子的发射,位垒远较发射单个核粒子为低。事实上,在铀235核裂变过程中,抛射象核碎片那样的大块所须克服的位垒是非常低的。
这个核一旦进入激发状态,就能瞒踝地爬过这道矮矮的裂变位垒而瓦解或若干碎片。
这种情况在分子中也能见到。从分子里即使抛射出一个孤零零的电子也须要相当可观的能盘。但将这个分子分裂成单个的原子,所须能戳则小得多。这就是为什么在化学反应中,分子并不分解成电子,而分解成原子或原子集团(根)。
铀238核由于中子而产生的裂变与铀235裂变非常相似。对铀238来说,激发状态与稳定的起始状态之间隔着一个相当宽的,足足有一兆电子伏特的能蠹距离。为使这样的核上升到激发状态,需要速度极高,能量极大的中子。

§78原子核究竟能有多少个
你可能巳经猜出原子核的数目确实是有限的。核愈重,它就愈不稳定。即使是一个铀核,在它自发地抛掉a粒子而过渡到一个更为稳定的状态以前,就已平均存在了几十亿年之久。根据并不困难的计算,比铀更重的核在发射一个“粒

•185•
  
子以前也能生存颇长的时间o
当然,还有另一件事情也限制了重量范围。我们刚刚讲过,对于重核分裂成几大块而言,核周围形成的位垒是很低的。可是一一对,你已经猜到了—一一个核碎片从位垒下面
•	渗过的几率也是不容忽视的。
不需要中子或任何形式的激发,核也会自动地分裂并以隧道方式穿过位垒。事实果真如此吗?1940年,自然界点头说:是。重核的天然分裂是苏联物理学家福辽洛夫和彼得夏克发现的。它不是噩子力学的猜想,而是一个被验证了的事实。
核愈重,包含的粒子愈多,则这种裂变的几率也就愈大。
但铀核的天然裂变是绝少的,这个几率几乎等于零。至于锅(第98号),它的核对自发裂变而言的平均寿命只有几年,而不是几十亿年。
最后我们还有这样的核,它的裂变位垒全然消失。这种类型的核对裂变并无抗拒。实际上,这种核永远不会形成,因为它一且形成便立即瓦解为碎片。建造原子的标准化工程表*中的最后一个序数是120。这意味着在任何情况下,核(当然也就是原子)不能具有i20个以上的质子。
芷是质子的数目断然决定核对裂变而言的稳定性。在重核内质子之间的斥力急剧增加,而相距甚远的表面粒子之间的核引力则迅速降低。

*指周期表。一一-译者注

•186•
  
其结果,核的表而上出现翻腾芯的质于,中了却在一旁歇凉。质子之间的斥力将核表面撕成碎片:核分裂成几大块。

§79核是壳层和液滴的结合

我们刚刚讨沦了原子核的两个模型。一个模型提供了与原子有些相似的壳层结构。另一个模型则使人联想起液滴。两个模型之中,究竟那一个更接近于真理呢?
最合理的回答是:两个模型都好,但各有其特殊的应用范围。对于描绘一个未受外界因素影响的恬静的核来说,壳层结构更为适用。若核处于紧张状态之中,如果整个的核沸腾起来,因而粒子在相互有力地碰撞着并从核中蒸发出来,要是情况变得这样糟糕以致核分裂成碎片……如果事情是这样的,则液滴模型能够应付裕如。
那又何不将两种模型结合成一个,以便同时很好地描述两种不同类型的现象?然而从普朗克量子理论中我们就已看出:理论的结合不能象裁缝活那样拼拼凑凑。
核的联合模型,即所谓普遍化模型,在十年前由玻尔的儿子,著名的丹麦物理学家欧基·玻尔提出的。当然,这个理论继承了老模型的某些特征,但仍然与它们很不相同。
这个普追化理论的要点是:当核内的质子与中子数等于或接近于幻数时,壳的行为是壳层类型的。不然,核就要象液滴一样池活动。不仅如此,当已埴满的封密壳层之外的粒子数目达到下一壳层粒子总数的2/3左右时,核的液滴特性表

•l87•
  
现得牡别突出。
这就是说,核的已埴满的壳层以外的粒子引起核的全部诡谕行径:从单个粒子的发射到整个核的破裂。与此同时,已坦满壳层的粒子表现得谦虚得多:它们不直接参与核的这些活动。
这里,我们有必要重温一下原子外部结构中的电子壳层。
你还记得,惰性原子密封壳层中的电子象隐士般地与世隔绝。与此同时,未填满的壳层中的电子则积极地结识邻近的原子,以便形成分子和晶体,或参与化学反应。
普遍化模型认为:核粒子之间并没有很多的直接相互作用,壳层特征也不是最重要的方面。除了成对的粒子的相互作用而外,这里也很可能有一种集体形式的粒子相互作用,而这种作用能更好地反映在液滴模型中。这些车实体现在核表面的变形之中,结果是:核不能保有质子电荷的球状分布。当然,这些书实还体现在共他许多核特征之中。
普遍化模型所预期的原子核的电力、磁力以及其他特征与实验很好地吻合。
关于模型以及物理学家怎样利用模型来描述原子核特征,已经谈得够多了。这些并不是全部,还有很多模型也曾被想象出来。
模型这么多是利呢还是弊?看来还是个弊。因为尽管核具有多方面的性格,它小实上只有一个面孔。这许许多多模型之中,每一个都不钻,但不是在这,就是在那方面不能令人满意。这说明:虽然核是个统一体,但这个统一体又是非常

•188•
  
难以把握并理解的。
就像在不同的照明条件下,从不同的角度拍摄的十来张相片,每张相片都只是整个景物的微小的、局部的映象。很自

然,将这些片断凑成一个完整的形象是很困难的。
就原子核来说,主要困难在于:我们对核力知道得还不够。
这些力鼠对粒子的电荷根本置之不理,它们只在短距离内起作用,并且是很强的。我们还可以补充说:这些力正如所有的交换力一样,依赖于相互作用着的粒子的自旋在方向上的相互关系。
当物理学家能够窥测并了解核粒子本身的内部结构时,核力便会被更好地认识。研究核粒子就是探索一个崭新的世界-—比原子核可能更为广阔的新世界,物理学现在只是刚刚开始接近它。

§80核内飞出它所没有的粒子

我们已经知道a粒子和r光子是怎样从核内逸出的。但P粒子,即平常的电子,是怎样被发射出来的呢?
大约在三十年前物理学家就着手研究这个问题,那时候他们是满怀信心的。当时看来fl放射不会长久得不到说明。自然界并不急于供出自己的谜底。不久以前量子力学对核的G和丫射线作出解释,至于P射线,直到今天物理学也未能将它完全征服。

•189•
  
P放射可能是原子核哀变最常见的形式,可是址子力学竞然不能对它作出解泽,这种困埮实在恼人。自从1934年小居里夫妇(弗雷钮里克·约里夏和其妻艾琳·居里)发现用中子轰中核产':i人为放廿以夹,尤共足核反应堆诞生,从而使大规桢轰击成为可倍以后,地蛛上木曾听说过的新放射性核便跺原不断地吮入物理卢家的辛中。
在过去的四分之一世纪里,一千余种新的放射性同位素巳玻人为地制选出来,其中大多数放射{J'而非”粒子。
解释8衰变的第一个,也是最主要的一个困难就是电子不危存在于核内。早先,当我们讨论核的质子-中子换型时,我们就提到过为什么会如此。现在我们就来谈谈屯子不应存在于原子核内的主要理由。
关键问题是不能将电子塞进核内!如果我们能够想办法将整个几率云驱入核内,则电子可能被认为就存在于核内。但即使屯子速度特别高,其能匮已经达到核能的数阰级,德布罗意电子波的波长也仍较核的线度大几百倍。而电子云的线度,正如我们在氢原子中所看到的那样,是与电子的波长屈于同一个数量级的。
核内无电子容身之地的另一理由是:电子的自旋与核粒子的自旋合在一起,将会引出不正确的核自旋数值。
物理学家一旦接受了上述推理后,便毫不含糊地剥夺了电子在核内的安息之地。既然电子从来不曾存在于核内,它们从核内逸出又是怎么一回事呢?核是一个非常重的粒子,可是它却生出这一极轻的电子,宾好像一颗小小的子弹从贝

•190•
  
莎巨炮*口甩飞出一样。
这个核真是个奇迹。不仅如此,电子从核内飞出违反了物理学的两条基本定机能景守恒走律和庄动篮定恒定律。

§81电子有个同伙
物理学中有这样一些基本定律,整个的科学大厦都建立在它们之上。这些定律对不同等级的世界和不同的现象都生效3
一个就是运动既不能被产生也不能被消灭**。一种形式的运动可以转化成另一种形式的运动。运动能改变形式,甚至能变得不可察觉,但运动永远不会消灭。
在经业物理学诞生初期,科学觉得运动的某种盘度是必需的。这不仅是为了描述运动,而且要衡懋它、计算它。两个新数蛁便被引进物理学:能懿和动昼。
运动既不能创造也不能消灭这一命题反映在:诸物体参加某共同活动时,它们的总能址和总动最保持不变。大炮的后坐,运转行的发动机的屈升,打桩,还有其他无数例子,都丝亳不爽地说守看两大定律:能屈守衡和动匪守衡。对于旋转运动来说,一个具有同样根本性质的定律就是角动掀守恒定律。甚个花样溜冰运动员都要利用这个定律。当他将手臂收

*扣gHertha是第一次世界大战中饱军攻打巴黎用的巨炮。一-一译者注
**恩格斯:"运动和物质本身一样,是既不能创造也不能稍灭向飞«反杜林论»第一编第六节笫57页。一一译者注

•191•
  
找来时,旋转就会加快。
我们可以想象,当P粒子被发现能具有从零到某个最大值之间的任何能量数值时,物理学家是多么地惊慌失措啊!因为在当时人们郡认为:核是一个埽子系统,并且具有一些确定的脂级。
换旬话说,核内的任何过程(如3粒子的发射)只饶按下述方式进行:核从一个确定的能级过渡到另一个确定的能级。这当然意味着:这个能量差,也就是被P粒子带走的能量,也必须是同样地确定的。
在6蜕变中,电子能量光谱并没有显示出对应着叨确能量的光谱线的任何细微痕迹。这些表明:要么核归根到底并不遵守盘子定律,尽管在其他过程中遵守得很好;要么核的P衰变违反了能量守恒定律。
还不只是能最守恒定律。电子从核里带走它的能量,同时也带走了它的自旋——这个与电子的本质密切地结合在一起的自旋。但是我们发现,在8粒子发射以后,核自旋仍然保持原样。可能电子将其自旋留在核里了吧?不,那是绝对不可能的,正象一个电子没有电荷,一台钢琴没有琴键,一个科学家没有头脑一样不可能。
对此现象的想法是五花八门的。有些科学家深深地信仰核生活的量子定律,以致他们建议牺牲能量守恒定律。他们企图把这个定律作为仅仅是经典的而扔掉。
其他物理学家认为那是不行的。这个想法很快就被放弃,但要走出这个困境仍然无路可寻。于是泡利声明:电子

•192•
  
这个罪犯有个同谋。这个同谋又长得象什么样子呢?
{I蜕变中,核获得了一个额外的正电荷,其电量在数值上恰好等f被释放的电子的电荷。这就好象核被离子化了。这暗示壮这个同谋是不带电的,在电上是中性的。
这个同谋应当有这祥一个自旋,它在数值上与电子的自旋相等,在方向上与后者相反。两个数值相抵消,给出的数值仍然是零。这样,当电子和它的同伙被抛出以后,核自旋正如所要求的那样,仍保持原来的数值。最后,这个电子和它的伙伴带走的能量,等于核p衰变中电子可能带走的最大能最。
这个最大能量是量子化了的,也就是说,它恰好等于核在8衰变前后的两个能级之差。但这个能量可以任意方式在电子及其同伙之间进行分配。这个分摊过程是不受量子力学制约的,因而也就不受任何强加于它的限制。
这样,核内能量的蠹子化被保留着,能量和自旋守恒定律也未被违反。泡利发现的这个摆脱困境的办法是真够聪明的。
这里还有一个睹礁。这个电子小偷立即被捉住了,可是它的同伙却狡滑得令人气愤。哼,我说的吧!怀疑派发言了。于是物理学又摊出了他们的最后一张牌:他们间接地计算了这个伙伴的最后一个特征它的质量。确切的计算是不可能的,但可以肯定地说,这个质匮小得可以忽略,至少比电子的质烘小一千倍。
这个幽灵般的伙伴终于被找到了。它没有电荷,其质量也几乎等于零;它所有的全部家当就是能量和自旋。没有电

•193•
  
荷这一点使它象不久前发现的中子,只是它比后者要轻一百万倍。这样人们就给它取了个爱称:中微子,或小中子。
可是这两种粒子并无共同之处。中子积极地与质子相互作用,与它们相碰撞并形成坚实的原子核。中微子呢?却是一个不具肉身的幽灵。计览表明,中微子能穿越整个的可见宇宙—一几百万光年——而不显示其存在。看来它是从来不与任何东西相互作用的。
如果允许扯得稍远一点的话,我们可以补充说:只是在削几年中微f才最终地、明确地被间接的事实听证实。8衰变理论的处埮与核力理论的处境基本一样。后者悬而未决达十年之久,直到它的存在终于以冗介子的发现而被证实。而泡利与意大利物理学家费密共同提出的P衰变理论作为悬案竞达四分之一世纪!
让我们言归正传吧。我们还是要找出:从P哀变的核里飞出来的电子到底是从哪里来的。

§82电子诞生于核内

在微观世界里,我们姆迈出一步就会遇见一些奇迹,这些,我们已经习以为常了。在下章里,我们将涉及一个绝对普遍的现象一一粒子间的相互转化。我们知道,在微观世界里,这种相互转化是这样自然而且寻常,正象在我们的宏观世界里,市物是相对地有常并且稳定一样。
我们已经和一种粒子间的相互转化打过交逍,那就是居

•194•
  



  
说,每件车都得从内部来做,也只有这样才能做得好。我们曾经将质子比作核大厦的建筑石块,将中子比作水泥:水泥把石块粘结成坚固的建筑物c)但如术大厦的建造并不牢固,或者大厦遭到来I上j外部的猛裴冲击一一例如一个中子的冲击,则又将出现什么呢?
自然界恢复平衡的办法是:将水泥转化为石块,如果后者太多的话;或者将石块转化为水泥,如果石块过剩并危及核大厦的话。
当这些转化发生时,电荷的剩余或亏损*就会被从核内抛出。质子砖石变成中子水泥须要甩掉其电荷,并以正电子-电子的带有正电的镜中映象一的形式将它抛射出去。当中子转化为质子时,它射出一个电子,从而增加了核的总电荷。
这些转化历时多久呢?不是12分钟。我们已经说过:中子在核里的地位根本异千它的自由兄第的生涯。有时候核内的条件是那样的,以致它对这种不稳定状态难以忍耐那怕千分之几秒。
有时候核内的这些条件抑制了中子或质子的衰变。这祥核就能生存一个很长的时期——-往往平均几百年或几千年__然后才开始P衰变。总的来说,这也没有什么很特殊的地方。核粒子的生存条件,正象不同的标准核结构的数目一样,是十分繁多的。

*电荷的剩余指正电荷,电荷的亏损指负电荷。一译者注

•196•
  
目前战子力学还不能准确地估计P放射性核的平均寿命。这不仅由f人们对核建筑只具有近似的知识(归根到底,只够得上是有关核力的知识)。关键间题是:量子力学还不能够合理地说明自山中f蜕变这个市实。
隐拔在这蜕变之Ji-i的是某些奇异的力址。这些力橇的发现,近年来已使i'l多物理概念革命化。在下一章里找们就要谈到这些力虽。

§83饕饕的核

让我们再来看看8衰变的某些非常有趣的特征。其中之一就是核并不总是仅仅抛射电子。
有时出现这样的韦:从核内飞出的是电子的镜中映象。
它们与电子的唯一区别就在于电荷:它们带有正电。解释这种类型的P衰变(附带地说,它没有一般的电子放射衰变那样常见)就更加困难。中子不能抛射正电子。质子却能抛射正电子然后变成中子。但与中子不同,质子就P衰变而言是绝对地稳定的。
r是我们义碰到f这个老问题:从来不曾存在于核内的粒子怎么能从核屯跑出来呢?情况看来更加复杂了。我们能够懂彴屯子是从何而来的——那就是从核中子。但正电子又是从哪电来的呢?
答案已经找到,这点我们必须留待下一章来讨论。可别着急啊。如果你其能沉住气的话,我们就会告诉你,最后这幕

•197•
  
是真够动人心弦的。现在要求你的只是一小点耐心。
为了让读者不至于太着急,我们大咯地谈谈核是如何鲸吞原子云巾的电子的。物理学家把这种凶残行为叫作电子俘获。
这又如何可能呢?本书开宗明义:玻尔发现的原子生活的晟子规律,间接地反映了原子自杀的不可能性。为此引入了轨道概念。这样电子从不至于因为辐射而损失能散,也不至于坠落在核上。
在核屯迳很小,而且电子也寥寥无几的轻原子中,械子力学的戍律是玻严格地遵守着的:电子几率云不会进人核占据的区域J在项核中,最深层的电子(它们的壳层最靠近核)所处的环垃沈十分不同。
在它们的外侧,众多的电子在排斥它们;而在它们的内侧,这个正的带有正电的核又同样有力地吸引着它们。电子再也不能长久忍耐这种腹背受力的局面了。电子云开始向禁区推进。原子电子进入核内似乎有点可能性(几率)了。如果有个几率,则自然界或迟或早总要在某些原子中实现这个几率。这就是电子被核俘获。
现在核里多了一个电子又怎么样?核的电堡减少一个单位,就像在正电子P衰变中那样。自旋呢?核自旋仍然不变,虽然它从电子那里得到了这个额外部分。
如果情况确实如此,唯一可能的结论是:电子带进来的自旋,又被另一粒子一一电子的老伙伴中微子——带了出去。不同的地方是:这个中微子不是与电子同一批释放出来的,

•198•
  
而是罚屯子在核内消失时出现的。
因此在原子内部,这一谋杀案的唯一见证者就是中微子,那个飘忽的、不只肉身的幽灵。我们知道,要审讯这个家伙是特别困难的,几乎是不可能的。
可是,在最近几半里,这个审讯终于完成了。审讯悄况如下。我们的见证者--—巾微子——陷和另一个核碰撞并将质子转化为一个正电子和一个中子,也就是下章所要讨论的一个过程的反过���。这个反过程就叫作逆P衰变飞
要想对单个的中微子进行实验是绝无希望的。唯一的途径是结娸许多中微子兵团,这样就可能有少数几个落网。
市实上也正是这样做的。核反应堆开始产生出强大的中子流。当中子被反应堆的壁吸收时,它们便诱导出人为放射现象。当然,被激发的核——蜕变着的核的碎片也为这种放射现象作出不小的贡献。
这就是P放射。一个核反应堆每秒钟内能产生巨大数量的中微子;这些巾微子非常容易地穿越核反应堆防护罢,后者却将中子和T射线禁捆在其中。
反应堆旁安置着一个非常大的闪烁计数器。
计数器盛有一种液体(甲苯)。液体内含有大量氢核,当丫光子出现时,即发生闪烁。俘获中微子时发生的正电子立即与液体原子构的电子相结合而湮没,就在它们湮没的地方,高能r光子出现(关于这一动人事件的细节,可见下章)。计

*也就是k俘获,在此过程中,核俘获R光层的一个电子,同时从核内飞出一个中微子。一一斗予者注
、•199•
  
数器发生最初闪烁时的悄况就是这样的。
为了便于俘获中子,液体内添加了化学纯福,这是因为镐的核贪婪地吸收着自由中子。这个中子的俘获同样伴随着T光子的发射,因而也同样产生闪烁。
这样,相隔时间约为百万分之一秒的两次闪烁就应为我们预期的现象提供某种线索。
这些双闪烁(那是非常稀罕的,反应堆工作许多小时才会出现几次)确实被记录下来了。原因只有一个:一个中彼子与一个质子碰撞产生出一个正电子和一个中子。这里,泡利提出的关于微观世界的一个新粒子的假设,一个诞生于笔端的假设,终于在四分之一世纪以后被证实了。
往后的研究表明:在所有微观粒子中,中微子确实是最不平凡的一个。但此是后话。
我们跟着量子力学在核世界里的游历就到此结束。当然,这电还有许多荆棘、陪礁和陷阱;说真的,它们是无穷无尽的。

现在我们就要走进一个更加晦暗的世界一物质的基本粒予的世界。这个世界更加朋确地显示了这样的规律性:实物粒子具有波特性,波也具有实物特性。
