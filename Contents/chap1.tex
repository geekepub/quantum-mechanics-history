\chapter{从经典力学到量子力学}

\section{代导言}

原子能、放射性同位素、半导体、基本粒子、脉泽、激光——这些都是熟悉的术语了,然而它们之中年龄最大的也还不到 25 岁。它们都是二十世纪物理学的产儿。

在我们这个时代\footnote{这里指 20 世纪。——编者注},知识以一种神奇的速度向前发展着,它每迈出新的一步都开辟了一个新的前景。古老的科学正焕发着第二个青春。物理学迈步在其他学科的前列,领先进入未知世界。当前沿阵地逐渐展开时,进攻的速度会缓慢下来,为新的挺进积蓄力量。

为了探索自然界的秘密,物理学必须拥有高效率的仪器,以便进行精确而有说服力的实验。物理学总部里聚集着成千上万名理论家,他们不断在作向科学进军的布署,研究从实验中缴获的战利品。战斗并不是在黑暗中进行的。强大的物理学理论光芒照亮了整个战场。现代物理学的强大探照灯就是“相对论”和“量子力学”。

量子力学是与 20 世纪一起来到人间的。它的生辰是:1900 年 12 月 17 日。就在这一天,德国物理学家普朗克在柏林科学院物理学会的一次会议上,作了有关尝试克服热辐射理论中的困难的报告。

困难是科学中惯见的事。科学家每天都会遇到一些困难。可是普朗克与困难的遭遇战却具有一种非常特殊的意义。因为它预示着物理学在今后许多年里的发展。

一株硕大的新知识之树在普朗克表述的一些基本概念中成长起来了。这些基本概念超出了最狂妄的科学小说家的幻想,但却成了许多令人惊异的发现的起点。从普朗克的概念中生长出量子力学;后者更开拓了一个崭新的世界——这就是原子、原子核和基木粒子的微观世界。

\section{新世界的轮廓}

在 20 世纪以前,人们对于原子难道就一无所知吗?从某种意义上来说,他们是知道一些的,也就是说,他们猜想并推测到了一些。

人类喜爱思索的头脑早已推测到这些东西,早已想象出那些只是在几世纪以后才变为现实的东西。

在远古的年代里,当入类的足迹还远远没有踏上发朗之路以前,人们就已经猜出,在他们居住的狭小天地以外,还会有别的土地、动物和人。

同样,人们还感觉到有个微观世界存在着,虽然它的实际发现还是很久以后的事。人们用不着长途跋涉去寻找这个新世界,因为它就在眼前,就在环绕着他们的一切物体中。

在往昔,思想家冥思苦想自然界是怎样从混沌之中创造出我们周围的这个世界的。这个世界为什么会存在着形形色色的东西?他们问道:“自然界是否就像一个建筑师那样,用小石块来建造大房屋?那么这些小石块又是些什么东西呢?”

高山峻岭被水、风和火山的神秘力量剥蚀净尽。岩石一块块地松脱下来,随着时间的流逝逐步崩裂成碎片。千万年过去了,这些碎石又风化成尘埃。

物质的这种不断分解有没有止境呢?有没有那样小的微粒,甚至自然界都不能把它们再分割了呢?回答是有的。古代许多哲学家如伊壁鸠鲁,德谟克利特就是这样说的。这些微粒被取名为“原子”。它们的主要特点是不能再进一步分割。在希腊文里,“原子”的意思就是不可分的。

那么原子又是什么样的呢?在古代,这个问题一直是得不到回答的。原子可能是一个坚实的不可穿透的球体,但也可能不是这样的。另一个问题是:原子又有多少不同的品种?或许有一千种,或许只有一种。某些哲学家(例如希腊的亚里士多德便是其中之一)认为很可能有四种。他们认为整个世界是由四种元素——水、空气、土和火——构成的。而这些元素本身又被认为是由原子构成的。

人们现在或许会这样想:仅仅有这点微不足道的知识,哪里还谈得上什么进步。确实是这祥。可是科学迈出的最初步伐往往有广度而没有深度。多少事物环绕着人类!第一件事是找出它们之间的相互关系,往后,也只是往后,才能去认识它们的构造。

当科学还在襁褓之中的时候,原子的概念是一个天才的猜想。但它仅仅是一个猜想。它既非来自某种观察,又没有任何实验的依据。就这样,原子很快就被人们遗忘得一干二净了。

直到 19 世纪初,人们才又重新想起了原子,最好是说又重新创造了原子。创造它们的不是物理学家,而是化学家。

19 世纪初期,不论对社会史学家来说,还是对科学史学家来说,都是一个有趣的时代:拿破仑正在重新划分欧洲国家的疆域;另一方面,在当时屈指可数的几个宁静的实验室中,人们在对事物的本性进行着重大的重新估计。看来似乎是相当牢靠的概念又被人们重新考虑。

英国的杨氏和法国的夫累涅尔奠定了光的波动学说的基础。挪威的阿贝耳和法国的伽洛瓦为现代数学大厦奠立了基石。法国人拉瓦锡和英国人道尔顿以事实证明化学能够创造奇迹。那个时期的化学家、物理学家和数学家作出了一系列的杰出发现,为 19 世纪后半期精密科学的繁荣开辟了道路。

1815 年,一个不知名的英国学者普劳特发表了这样一种观点:有一种微小的粒子,它参与多种多样的化学反应,而其自身不被破坏或重建。这种粒子显然就是原子。

就在那些年代里,著名的法国科学家拉格兰日为经典力学创立了完整而优美的形式,可是后来人们发现,在这种完美的形式中原子却没有栖身之地。

\section{经典力学的宗庙}

科学中没有无源之流。

量子力学可以公正地被称作牛顿开创的经典力学的智慧之子。

当然把经典力学的开创归功于牛顿一人是不十分正确的。文艺复兴时期许多伟大的思想家如列奥纳多·达·芬奇、伽利略、荷兰数学家西门·斯蒂文以及法国人布莱斯·巴斯噶都曾钻研过尔后形成经典力学基础的问题。从物体运动的所有这些分散的研究中,牛顿创立了一个统一而和谐的理论。

我们知道经典力学的确切生年:1687 年。就在那年,牛顿的著作《自然哲学的数学原理》在伦敦出现。那时,自然科学还得借用哲学这个名称与世见面。

牛顿的著作首次制定了经典力学的三个基本原则。这些原则后来被称作牛顿三定律,也就是每一个小学生都要学习的物理定律。

牛顿着手建造的力学大厦远远超过了这三条定律,而且这大厦也早就竣工。站在现代科学所处的优越地位来考察,经典力学的状况确是这样。

在辽阔的虚空中存在着数不清的各种各样的物体,大至巨星,小至尘埃。在遥远的过去曾经有过一个时刻,整个宇宙都不在运动,都处于一种绝对静止的状态之中。

陶醉于自己的创造成就的上帝终于给了它第一个刺激,从此便使它获得了生气。上帝的任务就这样完成了。从此以后,宇宙中所有的物体都按照确定的规律运动着并相互作用着。这种规律为数甚多,但是说到底可以归结为几个基本规律,其中包括牛顿三定律。

从这一刹那起,绝对不再有任何偶然的事物了。每一事件都是预先决定了的。任何随心所欲的事都不再是可能的了。从那时起,宇宙的交响乐一直在奏着最美好的和声。

继牛顿之后的一个多世纪里,建立在牛顿力学基础上的这种绝对严谨的宇宙秩序,使所有的物理学家感到十分满意。每当发现宇宙中的某件新事物丝毫不爽地符合这一理论时,他们都感到无比的慰借。在很长的一段时间里,自然界是允许人们这样看待它的。

但是好景不长。科学家很快就认识到再没有比僵死的教条更不稳定的东西了。一些简直不能纳入旧框框的事实开始显露出来。

19 世纪末,牛顿力学处于危机之中。人们逐渐清楚地看到:这个危机意味着普遍决定论\footnote{决定论是唯物主义观点,它是与唯心主义的“意志自由”相对立的。但这里的“普遍决定论”指的是“机械决定论”,达到顶点时即为宿命论。参看《简明哲学辞典》142页、476页。——译者注}——科学上被称为机械决定论——的崩溃。宇宙毕竟不那么简单,不能上满一次发条便永远走动。

量子力学不仅带来了新的知识。它对世界上的现象作出了根本不同的解释。科学首次对偶然性给予了充分的肯定。

也许不应责备物理学家们的大惊小怪。虽然垮台了的仅仅是他们杜撰的那种永恒决定论,可是物理学家却认为合理的决定论也站不住脚了,宇宙是由绝对的无政府状态统治着的,而事物也就不再服从确定的规律了。

过了相当长的一段时间以后,物理学家才从这一深刻的危机中找到了出路。

\section{宗庙倒坍了}

好奇心要了猫的命。这一谚语对理论来说也可能适用,尽管某个理论今天看来已经没有问题,而且能够解释所有的事实。

当科学对广大范围的现象作了研究并发展到某个阶段时,一种理论便会出现。这个理论的目的是想从某个观点解释现象。

但当新的事实被发现不能纳入这理论的狭小框框之内时,这同一个理论又被证明是不完善的,甚至是错误的。

只要物理学被局限于力学范围内,经典力学是令人十分满意的。但在 19 世纪,物理学扩大了研究范围:对热过程的研究产生了热力学;对光的研究产生了光学;对电和磁现象的研究则成为电动力学的起点。当此之际,物理学是颇感怡然自得的。所有的新发现都能很好地套进现成的模子中。

然而正当经典力学大厦逐渐升高的时候,它的庞大的正面出现了疲劳痕迹和危险的裂缝,最后在新事实的轰击下,整个建筑物便倒坍了。

这些最重要的事实之一就是光的不寻常的恒定速度。最仔细、最客观的实验表明:光的行为根本不同于其他所有已知领域内观察到的任何现象。

为了使光的行为符合经典物理学的框框,科学家不得不设计出一种叫以太的媒质。这种媒质——以太,按照经典物理学的规律,应当具有十分奇妙的特性。下面我们还要回到以太这个题目上来,并对它进行更仔细的考查。但现在需要说明的是:新的以太也挽救不了古老的物理学。

经典物理学的另一绊脚石是加热物体的热辐射。

最后还有放射现象的发现。在经典物理学占有绝对统治地位的最后几年里,放射现象给它以毁灭性的打击,这是因为放射现象的神秘过程不仅破坏了原子核,而且摧毁了物理学的基础——那些从常识的观点看去显得十分明白的原则。就从经典力学建筑物的这些裂缝中,生长出相对论和量子理论。

\section{新理论的名称是怎样取得的}

量子力学诞生于 20 世纪初。但它为什么取了这个名称?事实上,这个名称只是无力地反映着新物理学所涉及的那些事物的内容。

大概物理学的任何一个部门都不能逃避术语上的某种含糊不清。理由有很多,但这些理由主要应追溯到它们的历史根源。

首先,新理论为什么叫力学?新理论中不存在什么机械的\footnote{"机械的“与“力学”二词在英语中具有同一词源。——译者注}东西,而且以后我们还会看到也根本不可能有什么机械的东西。“力学”这个名词只有在普遍意义下使用时才是合理的,正像我们谈到“钟表力学”时,指的是它的运转原理。将量子力学的概念范围与物理学本身的广阔定义等批齐观,则更恰当些。

其次,为什么叫量子?在拉丁文里,量子的意思是“分立的部分”或“数量”。往后我们将看到:新物理学事实上所讨论的就是外围世界的诸种特性中的“分立性”。这是它的基本原则之一。另一方面,我们将要看到:这种分立性绝对不是普遍的,也不是任何时候、任何地方都可以被找到的。

不仅如此,上面所说的只是奖章的一面而已。另一个同样特殊的方面是事物的二重性。事物的双重性质寓于下面这一事实:同一实体(物体)将粒子特性和波特性结合在一起。

新科学就其最根本的内容取名波动力学。但这里我们也只谈了一半,因为量子这个概念还没有提到。

我们的结论是:新物理学理论的所有名称没有一个是令人满意的。难道真不能想出一个与主题的实际内容更加一致的名称?

将新术语引进科学是一件吃力不讨好的工作。新术语的引进是缓慢的,而它们的变动则更为缓慢。物理学家懂得这些新术语所包含的意思,因此我们只好将它们学会。

\section{物理学家建立了模型}

设想系在绳子端部的一个球在你的头的上方旋转着。这显然是十分简单的,因为这一切你都可以亲眼看到。经典物理学恰好就是这样发展起来的:它是从对我们周围的物体和现象的观察中发展起来的。

试在一个光滑的水平的桌面上滚动一个球,在手的动作停止后,即力已停止作用于球后,球仍继续运动。这一观察以及其他类似的观察,导至牛顿阐明的惯性定律:力学的第一条基本定律。

一个球在没有被手推动,或被另一个球撞击之前,是不会自己运动的。一个在光滑的桌子上运动着的球,与一个静止的球,有一共同点:它们都不受任何外力的作用。

当然,绳端上的球时刻受到一个力的作用,这个力使球偏离于自由运动所固有的直线轨道。静止在桌子上的那个球,在手施给它的力的作用下,将开始运动并将获得速度(力越大,速度越快)。这一观察产生了牛顿笫二定律。

观察家——还是牛顿——撇开了日常现象,转向星空去寻找曾使古代哲学家膛目的“天体合谐”的线索。什么力量使行星以这种方式而不以其他方式围绕太阳旋转?

“合谐“这个词表示一个有祑序的系统,表示支配天休运动的某种规律在起作用。当然问题不只是“天体”中的某一个。一定会有这样一条定律,它支配着环绕太阳的行星和我们的地球的运动,以及环绕行星的卫星的运动。

我们可以回顾一下系在绳端的球的运动。环绕太阳的行星运动的确很像球的均匀的圆周运动,虽然前者转得较慢而且没有绳子系着它。简言之,如果在一种情况下力在起作用,则可以合理地设想,在另一种情况下力也在起作用。

当然没有办法直接察觉到支配行星的力的作用。但是力是存在着的。牛顿发现了这个力。我们知道这就是物体相互间的吸引力。牛顿的天才之处在于:他察觉到了球的运动与行星轨道运动之间的共同的东西。

当然对我们来说重要的只是:系在绳端的球可能是最早的物理模型之一。从对规模小得多的事物的研究中,人们认识了像行星运动那样壮观的自然现象,当然是根据这样的假设:二者都是由相似的定律支配着的。但随着产生了下面这一问题:这种类比的作法是否在任何地方、任何时候都合理?换句话说,把一种现象的定律,扩大应用于另一种大得多或小得多的现象,是否正确?

在牛顿的时代,答案是简单的:既然观察确证了某种以小型现象为基础计算出来的大型现象的发展,或与此相反,任何一个定律都应当是不受限制地正确的。

大致相同的答案今天还可以听到。当然,提出这个问题的角度是有些两样了。首先,牛顿相信,宇宙是统一的;其次,宇宙的生命,不论在人类的规模上,还是在行星和恒星的规模上,都是由相同的规律支配着的。

从现代科学的有利地位出发,我们完全同意笫一个论点。至于第二个论点,我们当然不能得出这祥的结论:一种现象的内部活动是以它的外部运动为绳墨的。

鹦鹉学人说话,但如果认为它学舌的时候也在思考,则是一种天真的想法\footnote{应当说鹦鹉懂得自己所说的话的意思。请参看《自然辩证法》人民出版社1971年版第152页。——译者注}。

认识的复杂性在于:在微观、宏观和宇观的\footnote{即巨大的天体范围的。——译者注}物质世界等级中,绝对不同的定律在起作用;把一般事物的规律扩大应用到其他世界等级上去是会遇到严格限制的。

由于物理学家不能正确理解上述重要结论,因此当他们遇到微观世界中那些不恪守规矩的实体时,就必然感到十分懊恼。这些物理学家一旦认识到微观粒子不能被纳人普通概念的框框之中时,他们就开始谈论什么混沌状态,谈论什么没有规律的自然界。但下面我们就要谈到,事情绝非如此。

模型的表象在自然科学的发展中曾起到,并继续起着重要的作用。一些伟大的发明曾借助于双手建造的模型,或者借助只存在于头脑里的筷型;后一种情况更常见一些,因为有些模型是制造不出来的。

用绳子系着的球是一个简单的摸型。随着时间的推移,更复杂的模型被发展出来了。它们越来越复杂,越来越光怪陆离。虽然这些模型可能变得很奇恃,但它们有一个共同之处。它们都是用我们日常生活在其中的这个世界,即我们能够看见并感觉到的这个世界的原材料制造出来的。

这是人类头脑的一个特点。玄而又玄的抽象概念和一般概念总是来自现实。

\section{不是每件事物都能用模型说明}

从 19 世纪末以来,用人们熟悉的模型方法去研究自然界中的新事物并不总是成功的。以太模型就是一例。它的创造者把它奉为经典物理学的救星,可是它却不能说明光速的显著的恒定性。

让我们想象一下这个以太。它是某种绝对坚硬和绝对透明的东西。是不可击碎的玻璃吗?尽管以太是坚硬的,它必须允许所有不同的物体在其中自由运动。不仅如此,这些物体还应该能够牵引以太,产生一种像风的东西,一种真正的虚无缭渺的以太风\footnote{英文ethereal wind一语双关,因为ethereal有两重意义:(1)虚无缥缈的,(2)以太的。——译者注}。

许多年来,物理学家曾尝试掌握以太的这些奇妙特性,但他们失败了。以太被证明是一种没有事实根据的臆造。

然而以太概念还不是唯一的没有根据的实体。没有一个经典物理学家的原子模型能够说明铀、镭和其他化学元素神秘地释放能量这一现象:千百万年以来,在没有外部来源的情况下,这种能量的放射从来没有间断过。

爱因斯坦的光子假设又给予古老的模型以打击。光作为电磁波从光源向各方向传播这一概念虽然有些复杂,但还是能够被纳入经典模型之中的。

我们已习惯于认为波总是物质媒质的运动:海波是水这个媒质的运动,声波是空气这个媒质的运动。可是电磁波却能够在绝对的真空中传播。

在这意义上,像牛顿那样把光想象成一股又小又轻的粒子流,倒比较容易些。这些粒子被炽热的物体发射出来,向各个方向飞去;而当它们进入眼睛时,便会刺激视神经,使之产生光的感觉。现在想象这些粒子如何在虚空中运动,便没有什么困难了。

但像爱因斯坦那样,想象光同时具有波和微粒特性,对我们来说却简直是无法办到的。

从玻尔和卢瑟福建立的原子模型中,我们获得了一个可以想象的图画:微小的粒子——电子——在确定的轨道中绕着一个细小的核转动。这些轨道的尺寸要比电子和核大几万倍。

如果我们的想象力稍许再多一点,我们就可以为原子勾划出一种“空的“结构,因为我们自己就生活在一个行星系统中,而在这个系统里,“电子”(或行星)的尺寸,要比环绕“核”(或太阳)的轨道小几千倍。

然而仅仅几年以后,德布罗意把上述图象完全搅乱了。他说:电子、核以及一般说来我们世界上所有的物质的“建筑砖石”都具有爱因斯坦为光子引入的那种二重性:它们同时具有波特性和微粒(粒子)特性。其结果,物质微粒包括原子,就像早些时候的光微粒一样,不再是可以想象的了。

\section{看不见、摸不着的世界}

物理学家的日子不好过了。在往日,每当开辟了伸进新世界的道路的时候,他们总是确信:有出入的只可能是细节,不是本质。

可是现在,他们的处境就和古代的探险家一样:任何东西都可能被碰上,比如怪物或半人半兽。对于一个昏热的头脑来说,幻想是没有边际的。

物理学家的处境比那些探险家还要尴尬。探险家遇到正常的生物和位置虽两样,但本质却是相同的土地、山脉和海洋,虽然感到失望,但心情还是愉快的。可是在新世界里,科学家看到了如此离奇的东西,甚至任何名称都不足使人会意。连幻想都难以把这个新的、不寻常的原子世界描绘出来。

但建立某种概念是发展科学的要求,不管这些概念是多么地异乎寻常。建立量子力学是困难的,但又是必须的。

根据周围世界提供的可以想象的模型来建立学说当然要容易一些。可是如果微观世界是异样地建造起来的,则又如何呢?如果没有那种能用的模型又怎么办呢?

当然,如果不可能设计出一些可以想见的模型的话,我们就只好用那些完全不可想象的模型来进行研究了。

年复一年地过去了——虽然时间也不太长——这些模型变得那样地不可想象,但它们对物理学家却依旧十分珍贵,以致没有人想放弃它们。上述情况是很糟糕的,这是因为迟早有一天——请允许我们扯得稍远一点——这些模型终究会袚抛弃,并让一些更不寻常、更难捉摸的模型取而代之。这就是科学发展的过程。

20 世纪物理学家的伟大就在于:他们能够穿越用许多远离日常生活的抽象概念和模型筑成的迷宫,达到他们的目的,并成功地建立起关于微观新世界的意义深远的理论。不仅如此,在这个基础上,物理学家取得了一些整个文明史中最伟大的成就。他们发现了核能的秘密——那个亘古以来被囚禁在瓶子里的神怪\footnote{引自《天方夜谭》中“渔夫与神怪”的故享。——译者注}。如果没有量子力学,就不会有今天的原子能工业和电子学。

\section{困难而有趣味}

量子力学概念的不寻常的性质,以及这些概念不能被恰当地想象出来这一事实,使这个学科难以掌握。当然,有的缺点是量子力学本身固有的。可是这门学科之所以难于掌握,不仅因为它的范围在不断扩大,方法在不断改进:我们知道,描述某种处于不断变化和发展——尤其是迅速的发展——状态中的事物,要比描述那些已被确认的理论困难得多。不仅因为如此,而且因为物理学家至今还在对量子力学的意义,以及量子力学所描绘的微观世界的特殊方面,争论不休。

我们现在已经进入了空间时代,这个时代又召唤物理学家走在前面开辟道路。宇宙空间物理学根本不同于地球物理学,这在于:对前者而言,微观世界是头等重要的。

小大相成的古老观念在星际空间中被印证。巨星与微小的原子不仅融合,而且作为一个整体存在着\footnote{这段的第一句话引自西方谚语“两极相逄”(extremes meet)。这段话与庄子的“六合为巨,不离其内:秋毫为小,待之成体”;班固的“相反相成“是一个意思。——译者注}。

要想不借助某种可想象的表象来通俗地讲述科学几乎是不可能的。因此在讲述量子力学时,我们将尽量引用自然界中的模型或类比,如果前者找不到的话。当然这些类比不会是精确的或深刻的。它们只是帮助我们获得对某些事物梗概的了解。

例如,正像我们将要谈到的那样,“电子围绕原子核旋转”这句话讲给我们听,并不比“雪是白的,很像盐,并且是从天上落下来的”这样一句话讲给热带居民听,意义更多一些。原子中电子的运动,以及这样运动着的电子的本质,要较我们今天对它们的认识和描述复杂得无可比拟。不仅今天如此,明天也是如此,一千年以后也是如此。

真的,量子力学的发展进一步证实了电子特性的无限多样性和不可穷尽性\footnote{列宁:“电子和原子一样,也是不可穷尽的”。《唯物主义和经验批判主义》人民出版社1960年版笫262页。——译者注}飞不仅对电子来说是这样,其他任何事物也是这样。

今天,我们对周围世界仍然只具有一些相当支离破碎的知识。我们也才刚刚开始进入地球的外壳,进入海洋和大气层的深处。我们也才刚刚开始认识田野、森林、河流和沙漠的生活规律。

如果是这样,我们如何能够指望对那些难以观察的原子世界、原子核和基本粒子知道得更多些呢?对这门科学的探索还要进行几百年、几千年。直到现在我们也还没有离开这条巨大的知识河流的发源地。

尽管如此,呈现在新世界探索者面前的是一些多么令人惊异的事物啊!这门新科学为技术、工业、农业和医学开辟了一个多么令人鼓舞、多么灿烂辉煌的前景啊!核电力站、放射性同位素、太阳能电池——这些只是无数实际应用中的几个例子而巳。我们已经开始掌握受控热核反应,并正在深入探索星际空间。所有这些出现在光辉的现在和灿烂的未来的伟大成就,都是在 20 世纪里从一粒微小的种子中滋生出来的。60 年前,普朗克在科学知识的肥沃土壤上播下了这粒种子,此后又有许多杰出的科学家对它进行精心的培育。
