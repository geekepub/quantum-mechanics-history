\chapter{从量子力学到何处去}

\section{难以定义的定义}

质匮、电荷、自旋、宇称...…请你给粒子的这些特征分别下个确切的定义吧!同时要记住每个定义都必须能独立存在,也就是说,在描述某个量的时候不要引入另一个量。例如在描述质量的时侯,不要使用重力的概念;在描述电荷的时侯,不要使用引力或斥力的概念。
如果你使用其他的最来描述某一个量,你的收获将很微小。这些概念我们经常在使用着,但今天世界上没有一个物理学家能够说出这些概念的底蕴。
这就是量子力学今天的处境。量子力学广泛地使用从经典物理学借来的质匿、电荷以及诸如此类的概念。当然量子力学也发现了某些新的、自己特有的、对粒子进行描述的概念,例如自旋和宇称。但它对这些特征的底蕴,正象对质量和电荷的底蕴一样,说不出什么东西来。
这样看来,质量究竟是什么呢?答案有两个。第一,质量是物体内物质的数烘的抵度。它也可以被理解为单位体积的某种物质的原子核的数篮(因为原子核构成原子质显的主体)。而原子核的质甚也可以被解释为核粒子质子和中子的数匮。

一个质子的质量又是什么呢?质子的质骸是否象过去的定义那样,是质子内物质数最的量度?但什么又是属度?什么又是物质?匮度这一概念的本身就指出某种东西能被分成更小的部分。看来质子似乎不能再继续分割了。我们只能猜测质子内的物质是什么样的。
当我们说质子的质最近似等于10-24克,我们的意思只是每一克的物质大约包括1024个质子。由此看来,将质量定义为质子或其他微观粒子的物质的暨度,是没有什么意思的。
质最的第二个定义是:它是一个物体的惯性的量度,换句话说,它是物体沟其状态的任何改变所产生的阻力的匿度。在最菇本的实例中,质量决定一个物体对改变其在空间位置所产生的阻力。
因此,我们或许应当把一个质子的质量理解为质子在其他粒子的力的作用下被推动时所表现出的某种抗拒。这个定义也是不能令入满意的。力代表相互作用,因而归根到底,代表一个场的作用。当质子的速度增加时,它从场那里得到额外的质量;当其速度减少时,质子又将这部分质量还给场。尽管所获得的和所失去的这部分质鼠很小,但这部分质量却存在着。由此可见,质量是一个可变的量,因而也就失去了作为一个确定量度的特性。
于是我们发现:在微观世界里,质量本身也必须用某种东西来量度。在我们讨论的例子中,质子的质量是根据相对论公式,由质子的静质量以及它的运动速度与光速之比来决定的。
似乎有了一线希望。对于一种给定类型的粒子来说,静顶量的确是一个不变的量。如果静质量改变了,粒子本身也就要改变。难道不能由此推论出,静质量也是惯性的一个量度吗?当然这里所说的惯性,不是就一般的机械运动——在空间里的移动——而言的,而是就最广义的运动—一粒子的转变-—一而言的。
这看来颇有点接近于真理了。我们记得,当粒子的动能接近其原能-~后者确实是由静质量来决定的——时,粒子便获得真正转化为它的场量子的可能性。
如果事情确实如此,则静质量就成了粒千性质稳定性的盐度。对于某些粒子来说,这一质量是不很大的,因而在颇低的能量状态下就能转化为场量子。对其他粒子而言,静质量则大得多,因而粒子也就显然稳定得多。
从目前的观点来看,粒子既能产生真实的转变,也能产生所谓虚转变,后者就是构成它们相互作用的基础。因此质址便获得了另一个特征:它决定虚场量子的能量。
所有这些使质量成为一个十分复杂的概念。一方面,质量是粒子的某种如上所说的特征;另一方面,质量是粒子的所有相互作用的决定因素。
毫无疑问,粒子的其他特征也是同样地复杂的。今天,决定微观世界统一体的最深刻本质的全部间题,就是物理学所面临的尚未征服的山峰中的最高峰:物质的两种基本形式—久物和场一之间的相互关系。
仗物粒子具有场的特性。场量子也具有实物的特性......二者之中,那一个是最根本的,或主要的呢?是实物呢?还是场呢?
一个抇纪以前,当物理学刚刚获得场的概念的时候,答案是阴确的:实物是主要的。实物粒子在其周围产生一个场。场只是处押粒了的相互作用的一个附属工具。没有实物就不会有场。
随彷时闾的推移,人们发现:场能产生粒子,粒子能湮没并转化为场。因此场也不象人们假想的那样,只是一个附屈的飞具了3
可是物理学家又走向另一个极端。紧跟着爱因斯坦的足迹,他们宣称:场是主要的——宇宙间所有事物都是统一的、普迎的场的形形色色的表现。实物粒子仅仅是场凝聚成的斑点。没有场也就根本没有实物。
爱因斯坦花了许多年的时间去建立一个能广包所有已知类型的场和粒子的统一场理论,但他所有的尝试都失败了。物理学家逐渐认识到:主要的既非场也非实物,场和实物正如所描述的那样,都同等地是物质的两个根本的、主要的方面。
这种观点看来是正确的,因而统一场和统一实物的信奉者之间的争论可以休矣。可是物理学家仍继续在争辩着:人们关于微观世界的知识究竟正确到什么程度?人们的观念能符合这些存在的真实木质吗?人们将头脑里想出的理论强加于自然界,难道不会出错吗?代表大物体世界的人类难逍真能认识发生在原子、核以及基本粒子的微观世界中的事物和市件吗?
人类能够认识自然界的规律,并且越来越接近于真理。
但认识过程是永远不会终结的,关于世界的任何知识也永远不可能绝对地精确。物理学家以这些命题作为基础去接近这个问题:他应该怎样去理解物质的两种基本形式的相互关系。
首先,是否可能存在着一个统一的场或统一的实物?不可能。场和实物是物质的存在及其发展的两种对立着的形式。这两种形式中的任何一个形式也不能离开另一个形式而存在。它们是同一只奖章的两面。虽然是对立着的,但它们是统一的,不可分离地联结着的:场具有实物的属性,实物也具有场的属性飞
我们关于物质存在的两种形式和它们之间的相互联系的观念,是否具有某种程度的真理性呢?对这个间题的回答是肯定的:它们肯定地具有真理性,因为这些观念,虽然是不精确的,但整个说来却是正确的。总地说来,各种观察都符合这些观念的要求,而建立在这些观念之上的种种预见也都是有效的。
既然如此,为什么物理学家继续争论应该怎样解释巳经

*参石«毛泽东选乐»第五卷第347页:"各种事物都有对立的两个侧面".
”按照对立先一这个辩证法的根本规律,对立面是斗争的,又是统一的,是互相扞斥的,又尺互相联系的,在一定条件下互相转化的。,仁一详者注


获得的成果呢?首先是由于并非所有的物理学家都谙熟辩证唯物主义。敌对的哲学,尤其是那个最有害的流振-它被称为主观唯心、主义硬说世界只存在于人的想象之中,因而自然规律充其量不过是人类头脑的产物。这样一种哲学甚至使某些著名科学家也倾向于认为物理学的发现没有很大的真实意义。这些科学家认为世界是不可知的。
由于微观世界不能直娄地观察到,由于人们不能看见它,从而相信它的存在,因而上述错误观点也就更容易找到市场。更重要的是:微观世界的特性根本异于我们周围的习惯世界。这种差异是如此之大,以致我们的日常概念不能反映出微观世界的真正本质。
在科学发展过程中,新概念的产生是非常缓慢的。人类毕竟是生活在日常事物和一般观念的世界之中的,因此人们的心理也就固执着这些观念。从这种观念转变到另一种能够正确描述微观世界的,不可思议的观念是十分困难的。人们又不得不去实现这种转变。把一个微观粒子说成并想象成不仅仅是个粒子,把一个场说成是比场还多的什么东西,该是多么困难啊而这里,困难还不仅存在于所用的词上,困难尤其存在于人们的形象、概念和观念之中。
量子力学能够将旧概念结合于新的粒子-波、正电子-空穴以及介子一批子等形象之中。但在物理学家的头脑里,这些双重的存在还没有完全融合成统一的、真实的存在。
这种融合正是不久的末来须要做的事。

\section{量子力学小传}

量子力学在它存在的六十多年里已经历了三个发展阶段。
第一个阶段是从普朗克到德布罗意,即从光波的微粒属性的发现起到实物粒子的波属性的发现止,前后共25年。在这些年代里,爱因斯坦发展了光的微粒(光子)学说,玻尔发展了关于原子结构和原子现象早期的、很不完善的学说。
量子力学发展的第二个阶段是从德布罗意1924年的发现开始的。在短短的五年里,新理论的基本工具建立起来了。狄拉克将量子力学和爱因斯坦的相对论综合在一起。在这时期(一直到第二次世界大战),原子核的理论也被建立起来。
最后,也就是笫三个时期,是从笫二次世界大战后开始的。量子力学已经扩展到基本粒子和物质的第二基本形式—-场的研究上去了。
在第三阶段里,量子力学遇到了越来越大的困难。继其早年的辉煌胜利之后,一系列的挫折和失败接踵而来。
总地说来,尽管在原子和分子领域中量子力学是好的,但对付那些特硬坚果——基本粒子的构造和它们的相互作用,量子力学却表现得不够有力。
今天,实验已经远远地走在理论的前面。可是这个理论却仍然局限于解释原子核深处进行着的过程,而探讨基本粒子的本质已经是提到日程上的间题了。


械子力学还没有成功地解决这些间题。二十年前,噩子力学的缺点还只隐约地出现在朦胧的远方,可是现在,它们却越来越显而易见了。星子力学返老还童的时机巳经成熟。
这种情况难逍不与本世纪初经典力学的处境相似吗?
一方面,似乎没有任何事实与量子力学的基本命题相抵触。只是象这样的谥子力学对许多现象不能作出解释。也就是说,在解释这些现象上,理论本身一一而不是支持这个理论的科学家——巳经无能为力。或许须要扩大这个理论的体系,也或许须要增揉新的、重要的、不自相矛盾的命题来加强这个理论。
也可能出现这种情况:这些新命题将与早先的命题格格不入。这样,在某段时间里,我们将会感到失望。全能的理论过去不曾存在过,今后也绝对不会存在。每个理论大体就像人的一生:它有自己柔弱的童年;强壮的青年-~这时它能解决几十个特别艰难的问题;平静的壮年——这时它的前进步伐巳经缓慢下来了,可是理论的领域却在扩大:它将越来越广泛的现象包括进来,它伸进技术和工业领域中,并与其他学科建立起联系。最后老年终于来到,这时理论在新书实一自已发现的那些事实一一的冲击下显得软弱无力。
于是一个停滞的时期便会到来。至少看起来是这样,虽然实际情况不是这样。新思想一直在滋生着,它们发现旧理论的框框过分狭窄。迟早会有这么一天,新思想将从束缚它们的甲壳里脱颖而出,跟着科学也将出现一个巨大的跃进。
从上述的年龄尺度上看,展子力学今天巳达到了壮年的顶点,而老年的足迹已经悄悄地跟了上来。懿子力学与许许多多重要技术成就有着联系,它所研究的间题包括从星系结构到原子核和基本粒子这样一个广阔的炮围。今天,量子力学是最强大的微观世界物理学理论。
没有任何其他理沦能与它相较量,但一个能与它较鼠的理论肖定又是需要的。在这一物理学领域内进行工作的科学家,有的正在尝试增加不与它的基本原理相矛盾的新内容,来使它恢复肖春。有的在尝试改变量子力学的实质,并以更加先进的理论代替它。把它牺牲掉吧,他们说。可是还没有一个人能够夸耀巳经取得任何成就。
越来越多的物理学家倾向于这样一种观点:需要的是某种更加不寻常的东西,某种更加疯狂的理论。没有一个人害怕这些字眼了,因为任何根本新的东西都要遇到来自旧东西的极大阻力。量子力学出生时的境遇也正是这样。许多人都说它是疯狂的。但是现在,可以说再也找不到一个拒不接受这一理论的科学家了。
不管怎样,有一件事是肯定的:物理学正处在一个新的巨大跃进的前夕。这个跃进不是跃入黑暗之中,因为科学家十分清楚地看到新物理学必由的道路,以及道路上的一些驿站。
这些驿站包括以下这些。对所有已知的和未知的粒子的严密的、统一的、系统的整理。实物粒子的结构和内在特征。作用于原子核内的力的性质。关于物质的两种基本形式一一实物和场—一之间的相互关系的确切定律。运动着的物质的预见并非神或天才的慰赐,而是客观实际,在它的深处隐藏着社会发展的规律飞
科学不能等一个头等重要的问题自己去成熟。远在这些新问题获得头等重要性之前,科学家就在刻苦钻研这些间题,不管他们是否意识到这点。
科学是人类社会的最前哨,是未来的侦察兵,是现时最可信赖的保卫者。
量子力学的发现和发展很好地说明了上述情况。现在让我们看看量子力学的第二生命吧。
原子核的设想是1912年前后形成的。二十年以后,这种设想具有了鲜明的轮廓。构成核的粒子巳经明确,作用于核粒子之间的力巳被发现,并得到了解释。原子核的不可接近性——既是物质上的也是概念上的——没有使物理学家停步不前。十三年后便诞生了原子时代。当然这个时代是以美国人在广岛和长畸投下可怕的原子弹开始的:这两颗原子弹带来的不是富足,而是死亡和毁灭。短短几年的时间过去了,1954年苏联建成的世界上第一座原子能发电站开始工作。
量子力学最先在原子反应堆舱中获得了技术上的应用。
在反应堆舱里,中子流将重核击碎,从而产生热及电。
科学家于是又转向轻核-—-氢的同位素,尝试从中取得更多的能量。

*应当说:在人类社会发展的总过程中实践走在科学的前面,但也不否认在某段时间内和某些具体问题上科学预见的力益。恩格斯:“以前人们夸说的只是生产应归功于科学的那些事:但科学应归功于生产的事却多得无限。"«自然辩证法>163页。一一译者注


这里,量子力学也将大有作为。它计算聚变反应过程,并估计产生的能益。
还有呢?那将是些新间题。一些比我们目前巳经知道的还要固难得多的问题。但未来的科学家将要拥有比今天更好的装备。
直到最近研究人员才开始重视自己的发现所带来的成果。本世纪初杨A.约费开始对所谓废料感到兴趣。那时候他很难想象出半导体的发展前景的。
但是如果没有篮子力学,半导体也就不可能获得生命。
批子力学不仅仅解释了它们的重要特性,而且还提示了改进半导体的先进方法。今天,被称为固体能带理论的这部分量子力学,巳成为千千万万电子学研究人员和工程师的引航星。
那些体积小而效率高的电子装置已经引起工业和技术的根本变化。没有一个工厂、车辆或通讯设备不采用电子装置。人类活动的各个领域几乎无不受惠于电子学。
科学家已经着手从事最大胆的工程之一:利用半导体将普照大地的太阳能转化为电能,以代替几乎枯竭了的石油资源。第一批半导体太阳能蓄电池正在工作,把太阳光线转化为电能。设计人员正在设计一种适合在月球上和太阳系的其他行星上工作的太阳能蓄电池,为这些星球上的第一批居民提供能掠。
这方面的一个有趣的特征是:在地球上,半导体装备占据很大的面积(以便从日光中获得足够的能斌),这样就影响了庄稼的生长和家畜的放牧。但在月球上却不会出这个问题。
可是我们将怎样把大蜇的电能输送到地球上来呢?地球上使用的输电线显然是用不上的。不仅如此,常规类型输电的耗损是非常巨大的3
大约十年以前,苏联物理学家法布里堪持提出关于电磁波量子放大器的主张。这样,量子力学首先转化为量子放大器这样有形的器皿,然后又转化为量子振荡器。量子力学使一系列的装置得以诞生,它们包括:脉泽(无线电波放大器和产生器)以及莱塞(光束放大器和产生器)。这就是人们常说的科学幻想变成了现实。
本书开始时,我们谈到了那些决定原子的电磁辐射的量子力学定律。这些定律在很早以前(到现在巳三十多年,在量子力学史上这段时间就不算短了)就已稳固地建立起来了。这些定律的建立是那样稳固,因而在五十年代没有人再为这些定律花费心思了。
当那些喜欢创根问底的研究人员从一个新的角度重新考虑间题时,这些定律又放射出意想不到的光彩,从而使一系列新的、效率非凡的仪器得以诞生。
与量子力学俱来的关于微观世界的观念和概念促成了许多不寻常的技术上的成就,而上面我们只稍稍提了一下其中少数几个。量子力学继续深入技术和工业领域之内。量子力学的第二生命是极为丰富多彩的。我们亲眼看到了它的开端。它的未来将是最大胆的科学幻想小说所不能估量的。