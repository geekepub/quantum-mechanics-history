\chapter{译者的话}

为了在本世纪内实现毛主席、周总理为我国四个现代化制定的宏伟目标,以华国锋同志为首的党中央号召:要在广大于部、群众和青少年中形成爱科学、学科学、用科学的社会顷气。在这一号召的鼓舞下,译者把《量子力学史话》译成中文,希望它能对我国科学的普及起到一些作用。

这本书,着重讲述了量子力学的起源和发展,以及量子力学的基本概念。为使读者易于掌握这一理论的发展过程,现扼要地叙述一下各章的要点。

十九世纪末,建立在牛顿三大定律之上的经典物理学在热辐射、以太、光电效应、放射现象等问题上遇到了严重困难。笫一个打破僵局的是普朗克。1900年他在自己提出的热辐射公式中,假定能量是一份份地辐射出来的,每一份叫做一个能量量子。1905年爱因斯坦提出了光电效应理论:作为电磁波的光同时也是微粒,这散粒就叫光量子或光子,光子也有动量和质量,这样便确立了光的波粒二象性。1924年德布罗意从光波具有微粒特性出发,推广到实物粒子,如电子,也应具有波特性。德布罗意根据物质世界的统一性提出的这个大胆假说,四年以后便被戴维孙等人的电子衍射实验所证实,但德布罗意用“波包”米说明粒子的存在却归于失败。1927年德国物理学家薛定谔和海森堡提出了物质波就是”几率波”的概念,自此以后量子力学便在他们的引导下迅速地发展起来。

本书笫一、二、三章详细地叙述了这个发展过程。

从第四到第六章转入量子力学在微观世界——原子、分子、晶体、核、基本粒子——中的运用。

笫六章除了量子力学在基本粒子中的运用外,又回到理论上的发展:1929年英国科学家狄拉克将相对论结合薛定谔方程得出“相对论性不变”原理,并据此预见反粒子的存在。

笫七章以较大篇幅从哲学上来讨论量子力学的现状:在解决基本粒子的本质及相互转化等问题上,它已显得软弱无力;在阵疼中诞生的匮子力学又陷入了新的阵疼中,后者预示着一个与传统时空观念彻底决裂的物理理论大革命。

这本书以史话的形式,通俗的语言,生动、形象的比喻阐述了许多重要的概念,如波粒二象性、测不准原理、爱因斯坦的时空观念、宇称守恒等等。看来作者似乎是以大自然为主人公来写一本小说。由于自然界蕴藏着无穷无尽的奥秘与力量,因此它的传记也就必然十分丰富、真实、生动。

这本书的主要对象是青年学生、教师及科技人员。这本书会帮助他们培养对自然科学的热爱,鼓舞他们探索自然的信心和勇气。对于进一步深造的青年来说,这本书也会充当一名入门向导。不仅如此,译者认为对哲学工作者来说,本书也为学习和研究自然辩证法提供了丰富多采的现代科学参考资料。

译者写于一九七八年三月十六日,北京
