\chapter{原子、分子、晶体}

§50云代替了轨道

物理学中任何其他一个部门都没有象量子力学发展得那样快。在德布罗意波概念诞生后五年左右的时间里,量子力学的方法和数学运算在每个重要环节上都已搞了出来;具有重大科学价值的成果已经取得;为了充分利用这些成果,研究在向纵深发展。
到了1928年底,批子力学已经成为一座巍然耸立的大厦,一门基础广泛、上层建筑协调、十分健全而成熟的科学。与经典力学相比,它在内容上也同样地丰富多采。经典力学用了两百年的时间才达到的完善境地,最子力学只花了五年的时间便已达到。这就是二十世纪的步伐!
好像决了堤的山洪逐渐平静地在原野上铺开一样,量子力学经历了五年的迅猛发展之后也安定下来了。发展变得更加平稳了:它将一组组的新现象纳入了自己的工作范围,掌握着这些现象,并用新的观点解释它们。
矗子力学的第一个收获是原子。在普朗克和玻尔的引导下,这门新物理学便从原子着手。原子是匮子力学首先感到兴趣的对象。
量子力学的第一件事就是用它的新观点重新考虑原子结

•108•
  
构。玻尔引进了电子轨追的概念。这点,我们知道,是个不坚定的步骤:它带着经典力孕的味道o扯于力学不在轨道概念臣兜朋子,它断然地拒绝了这种概念。所谓轨道是电子在原子内运动的途径,但蜇子力学很正确地坚持:在微观世界中电于运动途径这一概念是无意义的。
代替轨道的又是什么呢?那就是原子内电子在不同位置上的几率的分布。我们知道,原子内电子的总能量是由它与核的距离来决定的。因而一组容许能域就对应着一组与核的.容许距离。
但我们总有些不乐意完全放弃轨道概念:轨道概念使我们能够容易地想象出原子的样子。晕子力学家说:“好吧,你既然喜欢轨道,那就保留着它吧。你不妨通过那些点,即具有指定容许能拯的电子出现的儿率最大的那些地方,划一条线。就把这条线当作你的轨道吧!但决不要忘记,你的电子并不是一个质点,它的波将它自身抹得模糊不悄,因此你的轨道实际上也只是个膝想。”
“好吧!“我们对量子力学家说,井向他道了谢。我们照他的意见划出了轨道。这样我们就满意地得到了一系列精致的曲线。于是鼠子力学家又补充说:“你知道这些轨道为什么这样钉趣?那就是因为:每一个轨道都容纳一个整数的德布罗意电子波。最靠近核的一个轨道容一个波,笫二个轨道容纳两个波,第三个轨道,三个波,余此类推*%

*这里所说的轨道就是壳层。诮参看§54;壳层序数与该壳层所容纳的电千波数一致。一一译者注

•109•
  
这的确是很有趣的,它是德布罗意波的普遍性的一个新的证明。
于是愤子力学家说:“你的有形轨道固然不错,但不要过于高兴了,因为它们根本就不存在。你尽星想象出一幅几率云图画吧,用它来代替轨道中的电子。几率云正是原子中的电子c云较稠密的地方,就是电子存在的几率较大的地方;云较稀薄或较透明的地方,就是几率较小的地方。看一看这些云层的像片吧。”
像片?难逍他们终于成功地捕获到这些行踪飘忽的电子了吗?并非真地如此。想要迥避那个测不准关系是绝无可能的。这些并不就是原子的像片,只不过是一些貌似原子中电子的几率云密度分布的特殊云雾模型而已。

 

图9

•110•
  
在这些图片中,我们看到了电子云具有不同的形状。其中有些是球形的,另一些是长椭圆形的,就像一支雪茄。这种多样性是由庄原子中电子能量不仅仅依赖于它与原子核之间的距离。
附带说一下,对于最简单的原子,如氢原子,确是如此。
氢原子的核场中只有一个电子。它们之间的作用是两个粒子——它们带有符号相反、数值相等的电量一一的相互作用。
这个相互作用是用库伦定律来描述的。而这一相互作用的能蜇仅依赖于电子与核之间的距离。这就很消楚,为什么氢原子的电子云是球形的:球表面各点与中心,即氢核,是等距离的。因此,电子云的各点都对应着相同的电子能星。
当原子中有更多的电子时,存在于它们之间的以及它们与核之间的电相互作用图画便不再象氢原子中的那样单纯。电子现在不仅受到核的吸引,它们彼此之间也在互相排斥着。
自然,在复杂原子的多电子家庭中,所有的电子都被家庭的中心——核-一-特别地吸引着,虽然他们之间也在闹着纠纷。自然界非常聪明地利用着这些紧张关系,并且将秩序带进原子家庭中。
从像片中可以看出这种秩序是什么样子的。这些电子云具有十分错综复杂的形状,它们以十分繁复的形式彼此贯穿着。如果我们能将这幅图画制成立体模型,并给各个部分染上不同颜色,我们会面对着这奇妙的彩色惊叹不巳,是的,这是与呆板的电子轨逍图画迵然不同的。

•III•
  
§51单调寓于多样性之中

这幅图回可能很悦目,但若要在这茫然一片的电子云雾中找出各云层的来龙去脉,却是一件不容易的事。
让我们访问一下大自然这个原子建筑师的创作室吧,看看他是怎样建造那些微型珍宝,那些被称为原子的绚丽多彩的坚实建筑物的。
大自然手中的建筑材料包括电子和原子核。使它们结合在一起的混艇土也已知道:这就是具有正电荷的核对电子的引力。
在建筑师的创作室里,第一件引起我们注意的东西是一张巨大的图表一门捷列夫元素周期表。到今天,104个方格巳被埴满,也就是说已有104种化学元素被发现*。
这些就是大自然用来生产宇宙间无穷无尽的原子建筑物的蓝图。一百多个蓝图!
乍看起来,一个图一个样子。但大自然比最懂得合理化的建筑师还更节约。
首先,让我们探讨一下原子建筑师是根据什么基本原理为原子大厦添砖加瓦的。这个原理是奥地利科学家沃尔夫

*目前已有107个元素被发现。但104、105、106、lOi号尚无中文命名.而笫107号至今还有争议。目前有的物理学家已从理论上展望到第九周期和第218号元素,但也有入认为周期表终点是175号。请参看«光明日报)d978.7.19第二版谢强、金日光作«从化学元素周期系来认识理论和实践的关系况——怪者注

•112•
  
讨·泡利在篮子力学创始年代里发现的,因而便以他命名。
这个原理不论对原子还是对微观世界的许多其他粒子集阳都同样地适用。泡利原理指出:在微观粒子的任何集团中,容计能悄的每一状态所能容纳的粒子不超过一个。
当fl,往后发现,这个原理并非绝对地普遍。对于某几种类吧的似观粒子来说,这个原理不适用兄我们将不去讨论例外情况,因而可以声明,对于电子来说,不管它们形成何种类型的仄肌这一法则绝对有效。
这酐、电子集团即原子。另一个原子形成另一个集团。
但对丁们一种元素的所有原子来说,它们的电子家庭都是一模一样的。

§52另一个奇迹——但尚未得到解释

这里我们不得不离题片刻,来谈谈电子的自旋。自旋的意义将在以后讨论,但有一件事必须说明:用经典物理学的观点来韶释自旋是行不通的。自旋的发现者天真地认为:所谓自旋就是电子自身的旋转。
地球绕着太阳转动,但它还绕着自己的轴旋转。电子围绕着核转动,但它也绕着自己的轴旋转。
这难道不是再消楚不过的吗?但现在姑且把它置之脑

后。

*对于自旋为0或1的粒子失效,参看§97。—一译者注

•113.,
  
电子是在绕着核转动吗?肯定不是。原子内电子的运动要远远复杂得多。将这运动想象成经典意义下的转动是对真实书物的彻底歪曲。
电子其地在绕自轴旋转吗?这种说法是再荒谬不过的了。不妨想象一下,电子的轴能是个什么东西呢?量子力学不将电子视作一个球体,而将它视作一个点\一个点的轴是没有意义的,更不用说点围绕自身或在自身之上转动了。这种描述我们是听然无法接受的。
我们似乎陷入了这样的因境:这种自旋实在无法想象出来。当然我们也从未得到一个粒子波(电子)和波粒子(光子)的令人满意的图象。
原子中屯子自旋的存在是凭借下述巾实显示出来的:对于原子中电子围绕原子核运动而具有的角动揽(动徵矩),我们还必须增加某一个阰而这个最是属于电子的固有运动的。换句话说,这个隘与电子所处的状态无关:不管电子是靠近核的,或者象在一块金属中那样处于远离核的半自由状态,或者象它在星际虚空中那样处于百分之百的自由状态中,这个拭也不受任何影响。电子的自旋永远保持不变,并永远与其启身联系在一起。
看来电子的自旋既可以加在其角动量(对应电子围绕核的运动)之上,也可以从后者减去。这一概念可以表述如下:电子总角动星的两个数值似乎与电子的两个相反的自身运动

牛这也是不确切的。i女者在下一罩将获得一个更好的解释。一一职注

•lH•
  
相对应。而这两个相反的自身运动实际上是没有任何区别的。这里实际二字具有十分确切的含义:这两种运动在原子内部具有相同的、不受任何外力影响的能晟。
结果是,胪子内每一容许能级可以容纳两个(而不是一个)在相反意义上自旋的电子飞

§53原子建筑师在工作

现在让我们更仔细地看石陈列在原子建筑师创作室里的蓝图吧(图10)。
笫一号蓝图:氢原子。我们不去细看它了,因为这个图太简单。可是说来也有点奇怪,科学竟用了几个世纪的时间来认识这个简单的氢原子。
笫二号蓝图:氮原子。这张图看来也没有什么趣味。我们刚刚说过,每个电子云可以由两个电子构成。这暗示着,氮原子与氢原子并无很大区别。模型就揭示了这点。但氮的电子云的密度要大一倍,并且更接近于核,如此而巳。云里有两个电子,而不是一个。
在捚原子(第三号蓝图)中,我们看到了第二个球形屯子云包住了第一个云层——氮云层。这是很自然的,因为泡利原理不允许原子的一个能层容纳两个以上的电子。

*这足泡利原理的另一种衣主。但由于两个相反自应的电子并不处于绝对笘同的能级中,因而不与泡利原理的第一种表述—-每一状态只能容纳一个校于——相矛肵~参石§51、57。一一译者注

•115•
  







  
安匿其中。显然,建筑师对这种设计是感到满意的,因为在下一张图(碳原子)中,他又在这垂直式楼层中安插了另一名房客。
往下的四张设计图并没有什么新鲜玩意儿了。除了第一个纵贯楼层外,又添了两个互成120°角的纵贯楼层。
大自然就是这样地将爱吵架的房客安置在他的小小的原子楼之中。关键是:这些房客并不打架,这点,如果想让房子稳得住的话,是完全必要的。而房子必须稳住。
上面的叙述意义在于说明自然界在原子建筑中所遵循的一个总原理。这就是最佳能量分布原理。
电子的相互排斥将大大增加原子的位能。但在自然界中,位能愈低,建筑愈稳。从椅子上摔下来不是一件有趣的半,可是拌在地上后,你却非常稳定了,因为你不再有位能了。
原子世界中也有这种力争稳定的倾向。最稳定的原子正是具有最小位能的原子。自然界在制作原子蓝图时,煞费若心去克服电子相互间的敌意,办法就是巧妙地让它们被核吸引着。
直到现在所说的原子的能罩原理也只关系到原子建筑物的复杂的内部布局。但事情不仅如此。在原子大厦中,住房分配还有更加奇妙的地方。

§54发了疯的原子

上面我们谈到了有关原子大厦的结构以及房屋分配的两

•117•
  
个基本原理:泡利原理以及最佳能量分布原理。
这里,电子的波性质是怎祥呈现出来的?首先,我们用带电的几率云代替了电子轨道。不仅如此。原子中德布罗意波还有另一特性:这种波决定着原子建筑物的容量。
回顾一下,电子云的特征是:它容纳具有整数值的德布罗意波。事实上,这个波数不仅决定着早先的轨道数,而且决定着全体电子所形成的电子云团的密度。而这个云团所包含的各个云层,容纳同一数目的您布罗意波。
物理学家给这个云层或联合电子云层(我们知道它是由一定数卧的云偶构成的)取了一个颇不恰当的名称一一壳层。盘子力学也为壳层的容屁(即壳层所能容纳的电子的最大可能数目N)与壳的序数n(这数与该壳层所容纳的电子波数一致)建立了一个关系式,这个关系式具有很简单的形式
N=2矿
这样,第一壳层(叫K)能容纳2X12=2个电子;第二壳层(L)2X22=8个电子;第三壳层(M),2X32=18个电子;第四壳层(N),2X42=32个电子;第五兑层(0),2X52=50个电子;依此类推。
请记忆一下这些数目。现在让我们回到原子建筑物上来。我们知道第一个须要装填的壳层是具有最低能量的最小壳层——K壳层。这个壳层在氮原子中已被填满。实际上这个亮层是一座只能容纳两名房客的平房。
笫二壳层则复杂得多。它不仅包括第二层楼,而且还包栝三个纵贯楼层,其中每个纵贯楼层又由两部分构成。这个

•118..
  
壳层到元素周期表第十格中的氖原子C笫10号蓝图),便已安置满员。
第三壳层最多可能容纳18个房客。这个壳层安置成员的方式也如第二壳层一样,到氪(第18号)住满8个便暂告截止。首先须将笫三层楼住满,然后再装填三个纵贯楼层。但在组后面的元素钾中,这一严格的秩序开始被破坏。
这里有五个纵贯楼层须要装填,但设计方案两祥。与前三个壳层不同,这些纵贯楼层变得更加狭长。新房客拒绝住进这种不方便的楼层,要求更好的居住条件。
最后建筑师将它安置在第四层楼的新居内。为了不使它感到寂寞,在下面一个原子钙中又增加了另一个电子。
这无疑是最佳能量分配原理。关键是,如果一个电子居住在较高的一层楼内,而较低的一层并未完全住满,则这个原子会更加稳定。在这样的原子中,电子相互排斥造成的位能比较小。
但E然界又恢复其老制度了。从抗(笫21号)到铜(第29号),新房客又被安插在那些狭长而不舒适的纵贯楼层中。
那些楼上住着电子而楼下空着的原子具有许多不寻常的特性,因此被叫作“异常“原子。在周期表中它们不时地出现。
严格地说来,第三壳层应在银原子(第28号)中填满。但大自然在填淌这一壳层之前便已着手填下一个壳层,因此,第二壳层在锌原子(第30号)中才填满。
往后的情况也没有什么改善。一个壳层还没有填满,下

•119•
  
一个壳层便开始装填了。从锁到铜的装填方式又重复出现在从忆(第39号)到把(第46号)和从翎(第57号)到镜(第70号)的原子系列中。从此以后直到最后一个原子(104),所有原子的装填规律都是有缺陷的,例如,甚至有两个或三个壳层是空看的。下章就要讲到,为什么这些原子的壳层从未填满电子。
这里看起来好像有些不对称,但从能量角度来看,这种办法是再好不过的了。
这样看来,决定原子结构中的电子分布的波定律也不是全能的。波定律经常须要由另一个具有同等重要性和权威性的定律来补充,这就是原子结构稳定性定律。

§55原子及化学

在离开原子建筑师的创作室之前,让我们仔细地瞧瞧元素周期表。
左侧是七个周期:第一周期2个,第二周期8个,第三周期还是8个,第四、五周期18个,第六周期32个(包括列于表的下方的稀土族或锄族),第七周期17个(前面曾说过,装填不满的理由将见下章)。
让我们再回来看看那些描述电子壳层容量的数值:2,8,18,32等等。这些数字恰好与上面所说的各周期中的元素数目相同。但为什么周期表中有些数字得到重复:2,8,8,18,18,32(目前暂不讨论第七周期)。这些重复着的数字恰好说

•120•
  
明:原于中电子装填次序在这些地方被打乱。因而第三周期以氮(第18号)而不是以镌(第28号)结束。自此往后,这种移动(以及由于违反装埴次序而造成的其他移动)一直统续到周期表的末尾。
其结果,各壳层容最与各周期容量并不能准确而简单地吻合。但各周期容凰不超过其相对应的壳层容量。这样,元素周期表的一个重要特征便由于量子描述而得到圆满的解释。
现在看看表的上方:这里从1属*到VIII属然后到0属。
从学校的化学课电我们枕学得:属代表原子价。
严格说来,这也不确切。首先,这些并不单是价,而是对氪(或者,根捉某些人的意见,对氢)而言的价。其次,什么是价?
在学校里我们学得:某元素几价就是说它能和儿个氢原子化合。今天,必种早期的理解属十描述式化学。这种化学只会将溶液倾注于趴窅中,并在本生灯上加热。理论化学早已奠定在物理学基础之上。
价——更正确点说,即对揽而言的价广—表征的是原子的最外(也就是离核最远)壳层的电子数。根据这个定义,价与属的序数相符合,但周期表中的最末两行例外。这里更正确的做法是:将最末一行改写成VIII屈,而将0,1,2分别标记在倒数第二行的上方。这样做是钉充分的根据的。
另一间题是:为什么原子的最外壳层屯子不超过8个?

*或译为族,但这样锅族就要译为锅系。—---译者注

•121•
  
只须回顾一下住房分配规则,就可以立即明白这点。第一壳层只容纳2个;笫二壳层8个;第三壳层应有18个,但到氧己装填8个后,第三壳层的建造便暂告一段落3这以后最外壳层是第四壳层,而第三壳层(现在成了内层)也同时开始装坟。笫四以及其余壳层的建造也是这样进行的3
最外壳层一旦填满8个电子,再继续装桢便有弊无利。
此时一个新的壳层便会出现,而未填满的壳层则退居于原子的较深处。这个壳层是否最终能填满电子,是无关紧要的,因为只有原子的最外壳层才能决定它的化学特性。
因此,原子的化学品性只可能划分为与最外壳层电子数朼气坟的八种类型。在进一步探讨这问题之前,有必要指出:这个龙层坦满8个电子后,要较它有全空着或半空着的楼层时位能小得多。这就意味着,具有这种最外壳层的原子是极屁定的,因而它的化学特性也是稳定的。
最外壳层填满8个电子的原子竟然高贵到如此程度,以致它们迥避与普通原子群众的接触。因此它们被取名为贵族,或惰族。它们占据了周期表的最末一行。
原子世界中的贵族在平民中间踱来踱去,平民尽显在模仿负族的举止。也就是说,原子平民尽最大努力将最外壳层配备成8电子组。
由于它们不能独立做到这点,因而经常如饥似渴地寻找#侣结果便造成了化学家所说的反应。真的,那是一种自找牺牲:一个原子把自己的衣裳全都献给了另一个原子,而巨己却赤裸裸地一丝不挂。当然这话也说得有点过头,并且

•122•
  
拒的也只是氖以后的原子。
为了说明问题,让我们以钠与氯之间的反应为例。这个反应形成氯化钠分子c钠原子在它的第三壳层里行一个额外的电子。而氯原子在自己的最外壳层有一个由7个电子构成的漂亮的小组。钠原子棣慨池献出了那独一的电子,这样氯便获得了一个高贵的8电子外壳层。
但钠也同样地得到了好处。它现在也能炫示一下高贵的气态氖所具有的8电子外观。两个平民转眼间变成了两个贵族,唯一的要求是:他们两个必须作为一个分子走在一起。
这样看来,原子可以被划分为施者和受者两类。那些外壳屯子在4个以下的为旌者,4个以上的为受者。许然,获得两个电于婓比割舍6个电子容易得多(顺便说一下,俄原子的悄况就是这样)。
在第IV属中我们找到了好几个懒汉。它们的最外壳屯子正好是4个,因此它们在取舍上犹豫不决。它们获得了两性元索这样的名称,也就是说它们半阴半阳。这些元素儿乎能起任何性质的化学变化。
现在看看第VIII属。那里我们遇到了一些发了疯的原子。它们本不该杲在那里,因为它们的最外壳层最多也只有一个或两个电子。关捷是:与最外壳层紧接着的那个内壳层对最外壳层电子的行为起着显著而极为复杂的影响。其结果,这些原子能做出砓难预料的事情。例如,它们的价可以变动:在这一反应中是这个价,在另一反应中则是另一个价。它们之所以被划归第VIII属,只是因为它们对氧而言的最高

•123•
  
价可拒达到8,也讥是说,每一个达祥的原于能把4个氧原子拴在自己的身边。
但不婓以为躲在周期表其他方格里的其他的发了疯的原子会表现得稍好一些。决不是这样。这些原子耍的花招与笫Vlll屈伙伴的惯技如出一辙。
门捷列夫元素周期表并没有反映出这点,而我们也不能这祥要求它,因为在设计该表的时候,还没有人了解原子的构造。今天科学家也不急于改变这个表,因为对异常元素的行为,我们仍有许多东西搞不清楚。如果每一件事都弄明白了,这个表也许还会有所改进o

§56光谱的诞生

既然我们已经能够用量子力学的新眼光来看一个原子,因此我们也就可以懂得它是怎样辐射的。我们记得,玻尔理论对原子光谱的起源作了解释,但不拒为光谱规律作出正确的描述。只有鼠子力学才能对光谱规律描绘得淋漓尽致。
在解释光谱起源上,量子力学与玻尔理论基本相符。原子中的电子从一个能级跃至另一个能级时,两个能级的能量差便体现在一个电磁能量子,即光子之中。当然,事情还没有

仁兀o

电子从何方跃至何方?只要保留电子轨道概念,这种眺跃是能够容易地想象出来的:仿佛电子从一个轨道或一组跑道跃至另一个轨道或一组跑道。如果能量减小了,一个光子

•124•
  
谊会诞生3如采能哏坰加[,则临跳跃前,一个光子或任何其他场的一个能属子已被吸收。
但星子力学却用电子云代替了轨道。这样便不容易想象出电子的过渡。我们不得不将它想象成原子中电子云在形状和姿态上的瞬时变化。一个光子的发射或吸收将搅动这个原子浆糊,使其整个形状发生变化。
至于电子的跳跃,监子力学拒绝使用形象的描述;但扯子力学的描述却获得了一个新的性质——几率的性质。在玻尔理论中,电子从一个轨道跃至另一个轨道总是可能的,而且这种跳跃的几率丝毫不依赖于轨道的类型。玻尔理论之所以失败就在于此。
量子力学证明上述结论是错误的。电子跳跃几率十分明显地依赖于电子跳跃前后的电子云形状。在这种形势下,粗略地说,云层重叠愈广,相互渗透愈深,跳跃的几率也就愈大。
形象地说,电子能从一种状态跳到另一种状态中去,正象旅客一样,能从一列火车跳到另一列火车上去,只要当时两列火车在并排地行驶着。把这个比拟引深一步:这位旅客必须有足够的、用来跳跃的能量,两列火车必须并排地行驶着,列车愈长(这样,彼此并排的范围愈大),两列火车愈贴近,这位旅客实现这一过渡也就愈容易。
原子中出现的事情也十分相似。这里,列车就是电子云的形状,而我们知道,这种形状可以是各式各祥的:有的是球形的,布的是雪茄形的等等。
对电子云形状的研究导致颇为简单的情况。具有一个共

•12.5•
  
同中心(原子核)的两个球伏云只有极少的相互渗透。我们甚至能有把握地说:两个云层彼此根本不接触。这就意味着:在对应的两种状态之间,电子不可能跳跃。现在将雪茄放在球体内。雪茄愈粗愈短,它们相互渗透得愈多。两个雪茄也能彼此横截,但计剪将更为复杂。有一点是清楚的:一个短而粗的雪茄与一个长而细的雪茄,要较球与短雪茄能更深池相互渗透。
这样我们就能求出电子从一个球状云到一个细长云的跳跃儿率,以及电子在两个细长云之间的跳跃几率。在旦子力学中,规定原子内电子过渡几率大小的定律就叫作淕垀枣则o匮子力学家非常严格地制定了这些定则:某些跳跃是容许的,另一些跳跃是被禁止的,因为它们的几率较小6但大自然并不遵守这个禁令。
轻原子中,选择定则遵守得较好,因为这些原子中的电子数茧很少,云层彼此交截并不频繁,但在重的、多电子原子中,云层交织成一团,敷子力学的限制或禁令大体上是不生效的。
正是在电子云的这种奇妙的、瞬息万变的频抖中,电子的跳跃导至光子的产生。光子进入分光镜,再出来时则被整理成各种类型,井产生出具有虹的全部色彩的光谱线。原子每秒钟发射的光子愈多,光谱线就愈亮。
如果原子的数目保持恒定,则光谱线的亮度只依赖于一件事:原子中电子跳跃的频率。而我们知逍,这个频率是被跳跃几率决定的。不同的云层间存在着不同的几率,有些较大,有些简直等于零。

•126•
  
每一光子能最与一定的几率相对应,前者决定光谱线,后者决定亮度。这就使原子光谱包含一系列不同亮度的谱线。
文字描述是容易的,但计算电子云的相互渗透,以及在这基础上计犹屯子跳跃的几率,则是十分困难的。可是量子力学出色地解决了这个间题:它获得的结果与被观察到的光谱极好地吻合。现在,光谱学大厦已奠定在花岗石般牢固的基础之上丘

§57宽线和双线

看来光谱学家现在总应该满意了。但巾实并不是这样。
光谱分析技木发展迅速,仪器越来越灵敏,效率越来越高。可是这时光谱学家又遇到了两个新问题,等待理论家来解答。
一个光子是否和某一个频率的一条光谱线,或者和某一个波长相对应?是这样。但为什么分光镜投射在照相底片上的光谱线不是一条细线,而是一条相当宽的线?
在量子力学未出现以前,物理学家为这个天真的间题可佳白白地绞了好儿年的脑汁,现在只要稍稍地思索一下就行了。造成上述现象的原因是电子的波特性,而后者又是以测不准关系为其永恒的特征的。
我们说过,原子内的电子具有十分确定的能鼠。那么测不准又从何说起呢?开始能量是确定的,最终能量也是确定的,二者之差一一这个数量与光子的能匮相对应一一也必定足一个绝对准确的数阰。

•12i•
  
可是,亳厘之失就出在这里。我们记得,准确的能级只是对电子的定态而言的,而定态是绝对不变化的(否则就叫作稳定状态)。现在电子的跳跃是对某种稳定状态的破坏。这一且发生,海森堡关系立即起作用。
电子在两次跳跃之间寿命究竟们多长?它是可以变化的因此命其为心。根据§49中的公式,我们立即得出光子能皇的不准量:

h
AE,.,.,—
At

这里使用关f能量子的普朗克关系,我们能够很容易地进一步得出光子频率的不准量。这一不准量与原子中电子的定居生活时间存在着这样一个简单的关系飞
1
Aw"'-
At

换旬话说,原子中电子的生活愈稳定,愈安祥,光谱线就愈狭窄(因为光谱线对应的是电子向另一伏态的过渡),反之亦然。这就是为什么在高温高压下,当原子中许多电子都活跃起来的时候,光谱线变宽并变得模糊不消。
笫二个问题是与下面这个事实相联系的:有许多这样的光谱线,其中每一条都好象是与某单一的波长相对应着的,但实际上却是一束非常紧密地贴在一起的光谱线。由于光谱技术获得了新的进展,光谱线的这种精细结构才被揭示出来。

*圆频率w=2兀v。—一译者注

•128•
  
这样看来,电子在相同状态之间的跳跃能产生不同的一-尽管差异十分微小的——光子。因此,如果说物理学家能准确地决定原子中电子的能鳖,那只不过是一句空话。
物理学家曾经愤慨地拒绝过有关测不准的猜想,但为此他们不得不提出有关自旋的假说。自旋的发现恰好是光谱的精细性质引起的。
事实证明,在产生光谱过程中,具有相反自旋的两个电子所处的共同状态并不真是共同的。如果我们在这里去描述电子的角动量与自旋之间的复杂关系,我们将扯得太远;有关这个问题以后还要谈一些。但我们能够说:这种复杂关系决定了自旋不同的电子具有稍微不同的能量。这就是为什么会出现双光谱线:代替一条光谱线的是一对具有相同亮度的李生光谱线。
实际上,只是当最外壳层仅有一个电子时,这种李生光谱线才会照例地出现。如果这一壳层有更多的电子,则原来的一条光谱线将包含三重线、四重线以及更大的线族。原子世界与人类家庭不同,在原子世界中这种现象是很平常的。
以上就是量子力学对光谱学家的两个难题所作的回答。关于原子的故事就讲到这里。下面我们将要谈谈原子的家庭生活一一分子和以晶体形式出现的整个原子军团。

§58原子结婚了

回忆一下平民原子是如何在模仿惰性元素的贵族原子

•129•
  
的。一对对的平民原子共享着华丽的外衣。有时候,三个、四个、甚至更多的伙伴加入进来。
从远处看去,这个花招还是耍得不错的。整个分子有时也象一个惰性元素的原子那样,在原子群中泰然自若地穿过。但近处一瞧,骗局便被戳穿。
在分子电,这些成员巳经不再是原子了。它们之中有的衣旮过奢,有的衣不蔽体-—-前者叫负离子,后者叫正离子。电子衣裳重新分配以后,失去衣裳的原子不让得到衣裳的女友涌掉。这个衣裳槛楼的伙计是不喜欢孤独的。这样结合起来的分子,用科学术语来说,就叫作离子分子。这种分子中的结合力,主婓就是具有不同电荷的离子间的一般电吸引力。至此,坻子力学还没有什么事可做。
离子分子是多种多样的。这里,周期表左边的原子和右边的原子相结合。原子愈是靠近表的两端,结合成的家庭愈是牢固。如果原子出身于表中互相靠近的属,则这种婚姻是不牢靠的。
但还有同样多的分子,它们中的原子出于一种全然不同的理由结合在一起。这样形成的家庭最简单的就是氢分子。这类分子包括所有的单元素分子(例如氧、氮、氯分子)以及那些分子,它们的原子要么都属于门捷列夫表的左面,要么都属于它的右面。这些分子就叫作共价分子。
这里便须请屎子力学来解释它们的存在了。设想一个氢原子遇到了另一个氢原子。他们就像两个单身汉,嫉妒那些有家的人。一个说:“把你的衣裳给我吧,这样我们就能形成

•130•
  
^尸
#.卢孕0i=
``
 
 

 
	H	+	H	=	Hs
图11

一个分千啦。”
另一个骄傲地反驳说:“我也同样有权向你提出这样的
要求哩广
“把衣裳交换一下,怎么祥沪
“这那电能行?我们俩的衣裳是一式一样的呀。”
就在这时候,一直在旁听着的原子建筑师——他现在不建造原子而建造分子了一一走上前来,建议道:“你们还是合伙吧!你们的家底都不玑谁也别想制一套贵族式的8电子衣裳。让一个电子一会儿在这个原子中住住,一会儿在那个原子中住仕,另一个电子也可以照着干。”
“这怎么行呢沪他们齐声戟道:“交换电子的主意我们早就考虑过了。”
“你们怎么糊涂了呢?你忘记了这样办一个原子将不时地拥有两个电子,与此同时另一个原子却一个电子也没有。这样你们看起来就像两个带有不同电荷的离子。在离子分子中,一个原子献出了电子,另一个原子获得了电子,因此分子中的原子几乎任何时候都是离子化了的。可是对你们来说,只须交换一下电子就行了。一会儿让电子绕若你转,让他光着身子;另一会儿再反过来。”

•131•
  
“我们将要多快地交换着电子呢?”他们问道,这时有些沃跃欱试了。
”相当快广建筑师说。“如果使用玻尔的半经典语言,我会说,电子大约在一个原子中旋转一周后就会跑到另一个原子中去。它好象溜了个8字花样。”
“好吧!让我们试试吧。“原子说。
结果是,一个很好的、很牢固的家庭建成了。只有量子力学才能揣摩出大自然的这种巧妙安排。至于这种导致分子形成的相同原子间的相互作用,星子力学家很正确地把它叫作“交换相互作用。“经典物理学绝不可能想出这样的点子。
量子力学是这样来描述电子的交换的:只要原子之间保持一定的距离,它们的电子云就不会重叠;但当这些原子彼此足够靠近时,电子云就有相当大的一部分互相交织,这就使一个原子的电子接近另一个原子的核的几率,即交换几率,明显地增大。
这个几率究竟有多大呢?对氢分子来说,约为15%。换旬话说,在每一小时内,有10分钟的时间两个电子全都在氢分子的某一原子中,而另一原子此时则一个电子也没有。
但这个几率足够使两个原子牢固地结合在一个分子之中吗?英国科学家海特勒和伦敦运用最子力学进行计算,结果是肯定的。真的,这里理论与实验极好地吻合。
在分子世界里,通过电子交换来组织富有原子和贫穷原子的协作是司空见惯的。
例如,氮原子(第7号)只有7个电子。在内层的两个不

•132•
  

二十世纪初,物理学在固体特性方面已经积累了可观的资料。我们知逍,固体分晶体和非晶体,它们以各种不同方式导热导电,光与声在不同的固休中的传播也不同。但这门名叫固态物理学的学科,妍释其中任何一个特性都有巨大的困难。
但这又是非常重要的,因为迅速发展着的技术,芷在利用各种新的天然材料。需要最是如此巨大,以致不得不求助于人造材料。
我们需要具有高硬峎、高导电性、高耐热性以及其他许多牡性的各种材料。但从哪里取得这些材料呢?一种办法是对所有已知材料使用通常方法进行某种结合一一这实际上是一种炼金术。这里还有另外一个办法,但它须要利用虽千力学。
这方面,仅仅在几年的时间里,就有一个突破。这方面的尝试是从理俯品休一一主要是金屈品休一一的结构开始的。
从晶休下手是最好不过的了。晶体是原子在空间中的打/秩序、有节女的分布,它的形状就像格子*。但一般的格子是
平面的,而这种格子是立体的C在格子中,晶体的原子彼此保
持沾固定的距离:这忧叫格距**。在一般情况下,晶体打三
个格距,对应右格子的三度:长、宽、i飞c
自然界兀纯粹的元索是不寻常的,更祁见的是它们的化合物。这种晶体格子是由几种类型的原子构成的。一个简单的例子就是冰结品:它具有氢原子和氧原子。这巠,与水的

*也可译成点防。---计者汴**或点阵间隔C-一一详者注

•13·1•
  
分子式相一致,氢原子的数目是氧原子的两倍。
另一个例子是氯化钠(NaCl)晶格。在格子交接处,即节点上,交替地安置着钠离子和氯离子。注意,它们是离子而非原子。这是很重要的,因为当食盐分子凝聚成固休时,原子键的离子性质仍然保留着。
这样,食盐便不再作为分子存在了,因为那样的分子不能被分离出来。真的,每个钠离子都被氯离子围绕着,而每个氯离子又被钠离子围绕符。因此,你哪里还能找到那个食盐分子呢!
在这样的品体里,作用在离子间的力就是一般的电力。
一个钠离子吸引行最靠近的氯离子。这些氯离子反过来又吸引符其他的钠离子,但却排斥着附近的氯离子。引力和斥力的相互作用导致了离子构型的某种平衡。这个构型就是晶格。
这种排列的确平衡而且稳定。如果某个离子被轰击而偏离原来的位暨,这个离子对不同离子的引力便会减弱,可是同类离子对它的斥力却大大增加。这两种力量的联合行动迫使离子返回原来的位置。
严格说来,由于热运动的不规则撞击,每个离子都时刻在相对其平衡位置振动,就像一个被几根弹簧拉着的小球那样。离子在品格内的热振动决定固体的许多重要特性。
就像在离子分子中一样,量子力学在离子品体中也没有很大的用武之地。但这时物理学又转攻金属晶体,因为后者在现代技术中是最重要的。

•I35.
  
这里情况却十分不同了。假设整个的格子是由单一金属,即同一种原子构成的。很自然,离子的电荷没有差别。那么当一个原子轻率地抛弃一个电子时,其他原子为什么也统统效尤呢?
估况果真是这样的吗?屈子力学回顾了它在氢分子上所获得的新胜利。如果金属品休真的是由多少亿亿个原子组成的一个巨大的共价分子,情况难道不正是这样的吗?
这个聪明的想法证明是正确的。大自然也并没有什么更多的新花招,也不能从容地应付各种局面。大自然在两个原子间所施展的电子交换技俩效果不错,因而便把这一诀窍扩大到更庞大的电子集团中去。
更何况这样的安排也不是很容易、很简单的。以后还会有很多机会证明:甚至对匮子力学而言,固体也是难以攻破的磷堡o

§60晶体的骨架和多层结构

当金属原子结合成晶体时,它们确实将最外层(价)电子化为公有。这就导致晶体的某种骨架结构的形成。位于格子节点上的是缓慢地移动着的离子,它们的周围环绕着共同的一层薄薄的、易动的电子云。这云层起着将带有相同电荷的敌对离子结合在一起的胶接剂的作用。反过来,离子也是胶接剂,它们使电子不致飞向四方。
我们曾经说过:金属中的电子几乎是自由的。由于每个

•136•
  
原子都对公共福利作出了贡献,因而一个电子不再是某一个原子所私有的了,而却是众原子的亿万公仆中的一员。这样的电子可以在晶体内自由地游荡着—一一个微观的费加罗%
当然,不是所有的电子都这般地自由。每个原子只献出一个或两个电子,而将其余的电子紧紧地束缚在原子中原来的地方。尽管如此,自由电子大军也是极为庞大的:每立方厘米的金属内有I022到Ioz3个。
如果允许打个比方的话,可以说金属晶体的社会组织比离子晶体的要好;因为后者有些象奴隶制,在那里所有的电子都被用锁链系在原子上。金属则更接近于封建制度:农奴主给他的农奴少许自由以棺取地租。这一进步立刻给金属以一些新的特征,并使它能够传导电流。
如果将一个苦通的电场施加于一个离子品体上,后者的电子云布局只会稍稍改变:它们将变得细长一些。这样就造成了所诮晶体的极化。没有一个电子能摆脱它的离子,而离子本身也如从前一样,被死死地拴在节点上。由于没有电荷的自由携带者,因而也就不能产生电流。离子晶体是绝缘体。
在金属中,大鼠的电子正准备用自己携带的电荷造成一个很好的电流。

但半导体又应安排在哪里呢?稍后一些它们就要出现。现在我们先研究一下噩子力学为金属确立的一个重要事实。这个问题是:金属中好似集体化了的电子具有什么样的

*费加罗是法国戏剧家博勹舍的作品«寨耳维的裁缝»和«费加罗的婚礼,中的主角,他的特点足欢乐、活泼而好动。——译者注

•137•
  
能量?答案是简单的:不再束缚于原子之上的电子似乎能够具有任何能湛。我们记得,对于自由电子而言,它们的能散水平失去量子性质。
但不要过早作出这样的结论。当然,电子已经离开了自己的原子,可是它们并没有离开这块金属。它们不再遵守原子的规律;但对于作为一个整体的这块金属而言,还有一些总的规律在制约符整个电予仗体(虽然不是某一电子个体)的活动。
现在看看这些规律。你记得原子的规律是从解薛定谔方程得出的。因此,在探索金属(金属品体)中的电子活动规律时,物理学家也依样画葫芦。他们为周期性电场内的电子活动解出薛定谔方程,而这个周期性电场,就是以固定间距分布在金属晶格节点上的正离子造成的。
这里先扯一点题外的话。截至目前,当我们谈到一个原子对邻近原子的作用时,我们似乎总想到某些外部的表现。例如,原子彼此吸引而形成分子。
可是,与此同时,在原子自身的内部又发生着什么呢?事实证明:原子内部电子云在改变着自己的形状。在擞子力学得到充分发展之前,德国物理学家斯塔克就发现了这个现象。斯塔克发现:当强电场施加于一块材料上时,它发射的光谱线就会分裂。
这种分裂与前面讨论的李生光谱线毫不相干。但它们之间却有某些共同的地方,这点已被匮子力学证实。这种光谱线的分裂是与原子中电子能级的分裂相对应的。

•138•
  
概括地说,施加于原子的电场将使电子的能级破裂。而当一个原子足够接近另一个原子时,这个原子的电场(在此悄况下已相当强)的作用,与上面所描述的电场的作用,并无任何本质上的差别。
当然,在分子形成的过程中,与构成分子的原子相对应的能级便消失。各能级自身破裂,互相掺合,能量水平起伏变化,这样便产生了对应着整个分子的所谓分子能级。
关于分子的上述情况,在品体中表现得更为明显。在晶体内,巨大数目的原子紧密地结合在一起,而这种结合方式在整个品体内都是一样的。真的,一个晶体无非是一个庞大的冻结了的分子。
这个分子的全部原子所形成的共同电场将每个原子的能级分裂成巨大数址的、层层紧挨的支能级。这里,外电子容许能级划分的分立性和明确性儿乎全然消失。因此看起来晶体中的电子能够具有任意的能量。

 

图14

•139••
  





  







  
后,品格节点上的离子振动能虽将变得非常大。这个能品能被传递给电子,因而它们有时就能获得足够的能批腾跃到传导带。于是绝缘体开始导屯。这就叫热致击穿。
当然,解释这种击穿现象并不须要戳子力学,因为这种现象只,意味右电子已经冲破原子的狭隘天地,进入f传导带,成了几乎自由的电子。解放这个电子所须的能擂,正好与隔开地下室与一楼的禁带的宽度相等。
这种情况可以描述如下:热撞击将电子从它的原子中抛出,使后者电离;而被释放的这个电子现在虽然可以自由地运动,但仍然不能脱离这块绝缘休。
但当一个卜分强大的电场施加于绝缘体时,它也同样变成个导屯休。且慢。这是否与上章所讲的金战电子冷发射相同呢?但绝缘休又不是一块金属,而是一个离子晶体!在金屈的屯子冷发射过程中,屯子完全脱浇了金屈,而这里,电子只从价带跃至传导带。
虽然存在着上述差异,它们毕竟是同一现象。两个例子

 

图16

•142•
  
 

图17

都揭示一个奇迹:隧逍效应。当然,如果禁区不是一个位垒的话王它还能是个什么呢'?是的,它就是一个具有实际上是无限宽度(当然对电子来说)的位垒。它就是一个只有前壁的台阶。电炀,如前所述,将禁带(位垒)偏转,使其出现了一个后壁。结果是,位垒获得了一个有限宽度(图17)。
其余的则与金屈完全相同。电子开始从价带渗过位垒而进入传导带。起初电流很小:穿越的几率是很低的,只有少数电子进入传导带。但当这个电流出现在晶体内时,就会使它发热,就像电流使电炉丝发热一样。这个加热过程反转过来又为传导带增添了新的电子大军,因而绝缘体内的电流自动上升。在极短的时间内绝缘体将被电击穿。而这一现象又伴随着热击穿-绝缘休础化。它已不再有用,因此必须扔掉。
但也可以用比较缓和的方式使绝缘体内产生电流3这种电流极为微翡并且绝对无害。这种电流是用照射离子品体的方法产生出来的。光子轰击品体,把电子从价带中抛至传导

•143•
  
带。这是一种真正的光电效应,但却没有电子的发射。可以说这种发射完全是在绝缘体的内部进行的。晶体不会被损坏,因此它能够得到各种实际应用。

§62电是怎样在金属内流动的

在二十性纪里,谁都不好意思提出上面这个问题。
电子离开屯掠,在电场的驱动下沿着导线运动,然后再进入屯原--这一切不正象水被泵入水管中那样吗?
但是我们不必为提出这个问题而感到害操。电的阻力是那里来的?导体不是水管,它的壁也并不粗糙。既然金属充满了那么多的电流运载名,它为什么又对电流产生阻力呢?正如其他许多天真的问题一样,它的答案很不简单。电流巳被发现一百五十年之久,大约在三十年前,问题才被弄清楚。
下面是经典物理学对电阻的解释。电子的定向运动——¬即我们所说的电流-—{在任何时候,都被金属骨架内离子的热振动所困挠。这些振动妨碍着电子的运动。在发生地震的时候,建筑物的墙壁、地板上下颠簸,左右摇晃,同时在剧烈振动,建筑物里的人群开始瞎跑:金属中产生电流的时候,电子的运动也是这样的。
显然,墙壁和地板的振动愈小,在建筑物里走动就愈容易。在绝对温度零度下,离子的热运动完全停止,电阻也应当下降到零。对于几乎不含杂质的非常纯的金属来说,上面讲

•144,
  
的很接近市实。但问题完全出在这些杂质上面。温度下降肘,这些不洁净的金属的电阻并不趋近于零,却趋近于某个依赖庄金属中杂质的含械和类型的非零值。杂质愈多,这个剩余电阻也就愈高3
对这一间题经典物理学有什么可说的?无话可说。经典物理学不把金属原子和杂肤原子区别开来。在同一温度下二者以相同的方式振动,因而也就以完全一样的方式阻碍着电子的运动。
这里晕子力学的见解确实略胜一筹。晶格中的不同原子彼此显然有区别,就好象它们真的具有不同的颜色。那么我们又如何去解释电阻呢?
首先,必须回顾一下那个精采的电子对品体的衍射实验,我们有关量子力学的讲话就是从它开始的。在那个实验里,电子撞击晶体原子的外层,部分地被反射并在照相底片上形成衍射环。
难道我们不可以把金属里的电子流看成一束电子射线吗?当然可以。这里,电子朝着一个总方向流动,若作射线来看,它无非比较粗壮,占据了金属的整个截面而已。这就必然导致下述情况:电子在金属内通过,应该产生对品格离子的所谓内衍射。如果我们能将照相底片放置在这块金属之内,我们也应该能够得到一个衍射阴样。
衍射具有一个有趣的特征:对波产生散射的物体只要形状稍稍有点不规则,轮廓分明的控样便消失,照相底片上将均匀地形成娄翁。照物理学家的说法,在这种情况下,这些波的

~145•
  
散射巳变成均匀的。
金属品体的整齐结构,也会由于离子的振动和杂质原子的存在而产生不规则的变形。其结果,形成电流的电子的波向各个不同方向散射。.
一般说来,杂质原子的体积和电子壳层与金属原子的大不相同。杂质原子使品格变形。如果将比喻引深一步的话,可以说,杂质原子将大厦的走廊弄曲,墙壁弄弯,并使地板走形。很明显,即使地板和墙壁都巳停止振动,这些缺陷也将依然存在。同时也很明显,杂质原子造成金属品格的畸变是与温度无关的;即使在绝对零度下,这种畸变也依然存在。电子波在这些品格缺陷处的散射,就是金属剩余电阻的原因,而对经典物理学来说,这简直是不可思议的。
由此可见,作为电流的导体来说,金属是远非完善的。当然不是说所有的金属都是这样,也不是说它们在任何情况下都是这样。自然界感到有必要提供一些传导性能更好的东匹,因而创造了超导体。
一些金属和合金——目前为数仍很少——在极低温度下表现很异常。在绝对温度约10度的条件下,这些材料几乎丧失全部的电阻。这个早在半个世纪以前就被发现了的现象,现在叫作超导性。
经典物理学不能解释这个现象。有趣的是,甚至卓越的量子力学也惨淡经营了三十来年,才对此提出了一个合理的解释。
超导性之谜直到儿年以前才被揭开。为此作出重大贡献

•146•
  
的是苏联物理学家波格留波夫和他的学生。进一步谈论赵导性将扯行太远。这里我们只尝试作些粗枝大叶但又生动形象的比拟()
奇妙的追导性是山下面这一事实引起的:在接近绝对零度的极低汕良下,好几种金属的电子云和离子骨架之间的相互作用,'.i!:f金屈结构的某种不寻常的特性而产生急剧变化。在这以前,电子大军的衍一土兵都在单独作战,而在超导性的低温状态下,电子结成了对子。
在电子与离子的战斗中,上述变化立即产生效果。在此以前,单独与离子作战的电子很容易被迫退出现役,而现在电子小组可以满不在乎池抵挡住这些单个离子的进击。看来电子好象对来势汹汹的离子的包围不再介意了。电子大军的困难大大减少了。最后,金属对电子的阻力便土崩瓦解。
用物理学的语言来说,这种新型战斗的特点是:对应金属内电子运动的波长的数量级现在要比离子的间距大几千或几万倍。如果你细心地阅读了本章的话,这些新战术的秘密便会了若指掌:电子对的波长要远远超过道路上的离子障碍物的线度,因而在正常条件下伴随金属中电子流动的那些单个电子的散射便消失,这样电阻也就跟着一起消失。
电子大军的这种理想组织只有在温度足够低的条件下才能保持。当温度上升到某一极限以上时,离子与电子的冲突便会打散这些电子对,使电子又复成为散兵游勇。力量的对比发生了变化,金属的电阻又恢复。
这样看来电是怎样在金属内流动的这个问题还是值得一

•147•
  
间的。
 

.:.:...;.
 

§63奇妙的半导体

 
在自然界卫,大懦的东西既不屈于电流的导休,也不属于绝缘体,而却屈于半导体。
它们的“半”或中间特性已经证明是非常有用的。半导体佪世才儿十年,可是它已促成了一个真正的技术革命。它们所具备的特性是人们所熟悉的:与绝缘体不同,半导体能在室温下导电;与导体又不同,随着温度的上升,它们的电阻不仅不增加,反而降低。
自然界曾在绝缘休、半导体和导体之间划了一条鲜明的界线。事实上我们已经知道它们之间的鸿沟究竟是什么。那就是位于充满电子的价带和包含许多未被电子占据的支能级的导带之间的第一禁区。
对于绝缘体来说,这一禁区很高,因此电子从地下室爬出,越过禁区而跃入一楼所须的能晕也就很大。这个能量只有在高温下才能获得(回忆一下热击穿)。
在半导体内这一禁区低得多,电子跃入一楼所须的能量可以在室温条件下获得。这就是半导体能在常温下导电的原因。

换句话说,即使对半导体施加一个很弱的电场,它的传导带里也会产生一个定同的心子流。现任让我们看看地下室屯发生些什么。

•148•
  
在地下室里,巾物也在发展着。关键是:当电子从地下室出来并进入一楼时,它留下一个空房。拥挤不堪的地下室开始调整住房。可是现在只允许一个电子迁入这个空房,因而近水楼台先得月。最邻近的一个电子立即迁入。但它照样也留下一个空房,后者又被另一个电子占据。
这些地下室的电子向一楼里的自由运动电子学步,可是它们只会从一个房间跳至另一个房间。真有点象只袋阻在学人跑步哩!跑步的人作短而快的跳跃,从远处石去,就好象他在以匀速行进。袋鼠却不然,它的步子大而少。
我们不妨作这样一个比喻:城中心有一个电子刚刚攒了家,其他电子争占空房,结果空房一步步移向城边。
电子的这种行宫被取了一个相当不休面的名称一空穴。空穴的作用与离开空穴的电子的作用恰恰相对立:在电场中,空穴就像一个带正电的粒子,向相反的方向运动。另一个不同点是:它的跳跃间距大而节奏慢。
在低温状态下,所有的电子都被严密地囚禁在地下室里。
当然,随看温度的升高,越来越多的电子被释放。其结果,电流增大,半导体的电阻降低。而对金属导体来说,情况恰好相反。
以上只讲了纯粹的半导体。这种电流的作用过程叫作内导性*。当然,对技术专冢来说,纯粹的半导体是没有什么趣味的。半导休所屈示的全部奇迹都是与所谓杂质同来的。

*戊内啖传导性。--汗者生

•149•
  
§64有用的“灰尘”

灰尘、杂质--在不受控制的情况下是坏东西,但如果以汒确的比例出现,则真是很好的东西。半导体与一般的物体并没有什么两杆,它们也会弄脏的。各种杂质都可以进入它们的品休内,这些都是偶然的,不受欢迎的。但有些杂质,如果使用的剂墙受到严格控制的话,则会是非常有用的。它们也就是那些创芷奇迹的灰尘:
对烹制雌的来说是美味的调料,对烹制雄鹅来说则未必如此*o如果你想得到一个具有高导电性的金属,任何杂质都是有古的。理由是:杂质混进晶格以后将会使它变形。这些变形或缺陷,将要使携带电流的电子的波产生散射。其结果,金属的导电率下降,电阻上升。
但这些晶格上的缺陷正是半导体成功的关键。事情是这样的:晶休的能带结构对晶格类型反应极为灵敏。每一种晶休邵有自己的能带系统。
当然,杂质原子也并不改变晶格的形状,而只改变最邻近处的形状。在这些区域内,整个晶体所共有的能带型式将有显著的改变。情况是这样的:在隔开价带与传导带的禁带中又出现了一些额外的容许电子能级。这些能级只是当品休内含有杂质原子时才会产生。为了与半导体的整个晶体内的能
	.	"
*四方谚itf:"烹制雄l房的i周料,也是烹制雌稻的调料”,此处用其反义。
——译者注
•1SO•
  
级区别开来,它们被叫作本地能级飞
金漓巾杂质的含量也影响着导电性能,但这种影响也是单调的一-杂质愈多,导电率愈低,虽然后者的变化范围是相当小的.•'可是对半导体来说,导电率不仅随着杂质原子的数目而变化,而且随祈杂质原子的类型而变化,而这种变化可能是于百万倍!

§65叉慷慨又贪婪的原子

目前砓普通的杂质半导体是以绪和硅化学元素为基础的。石一行元素周期表:硅是第14号;绪,第32号。它们都扂了第IV属。你还记得我们是怎样称呼这一属的吗?我们把它叫作中间旅。确实如此。绪和硅既非导体,又非绝缘体,它们是典型的半导体。
这两种原子的最外壳层,都同样容纳着4个电子。当原子构成晶体时,所有这些电子就形成了这些原子和其他一些原子之间的键。它们是地下室里的奴隶。因此在低温下,硅和错不导电。
但让我们在褚里加进一点邻近的某种原子,例如第V属里的碑(笫33号)。这些碑原子往往喧宾夺主,探走绪原子,并取代其在品格中的位置。每个神原子这样做之后,必须行使那些被逐的错原子的职能。

*或局部能级。__一译者注

•151•
  
碑原子的最外壳层有5个电子。其中4个将用于形成化学键一一这个键本是晶格中被攒走了的铭原子的键。而第五个电子呢,却失业了。
计算表明:这个电子的能匿恰恰与禁带里的本地能级相
	对应,但靠近上限。只须很小一点能量	它要比禁带本身
的高度小10至15倍——就足以将这个电子推进传导带中。
神原子在处理这个额外电子上是十分慷慨的。它将这个电子捐献给晶体的电子群,因此它就叫施主。相应的这些电子能级仇叫作诸施主能级(图19)。

 

 

图18
 
 

图19

 
现在让我们用绪左侧某属中的元素,例如硐(第5号),来代替碑。砌在笫III属中,这就是说它的最外层电子只有3个。当硐取代晶格的绪原子后,它只能接管4个化学键中的3个。
情况发生如下:硕原子从邻近的错原子中偷了一个电

•152•
  






  
原子跃至另4个原子。与此柜较,传导带里的电子则更象一个长跑运动员,举着短促而轻快的脚步平稳前进。我们说过,传导带里的电子能级也是彼此隔开的,但能级间的距离是如此地狭小,以致可以认为这些佬级实际上巳经融合成一体。
现在回到本题上来9让我们认将溯与硅同时掺进绪中。
这样绪l眢获f!J什么类型的传导?这显然要看两种杂质原子数显的比例。如杲硅较多的话,传导将是屯子性的;如果淜较多的话,传导将是空穴性的。
由f混合比例不同,半导休便获得了不同的童要用途。
具有上还两类杂质的半导体,能够使一个方向的电流停止,而让反方向的电流通过。这意味右,这样的半导体就是整流器Q
半寻休的另一功用是籽小屯压转变成大电压(这还是由于它们贝彴谓打电阻的能力)。这就是说它们可以用作放大器Q
这些体积小、紧凑、坚固而经济的半导体元件对大而笨重的电子管来说,早已显示了无可比拟的优越性。
光子射在半导体上时,会将电子从价带轰至传导带。因此当半导体受到照射时,它所在的电路内就将出现电流。这说明半导休能将光能直接转化为电能。半导体制成的光电元件已经得到应用,其效率比金属的还要高。
这一领域内的工作是由苏联物理学家约非和他的同市开创的。
硅蓄电池在沙漠地区能将灼热的太阳光流转化为电,后

•154•
  
者将驱动湘溉系统里的电动机,把水引向大地的干早角落。半寻体岱电池在宇宙航行中也已得到应用。
半导体也能将热能直接转化为电能。庞大而笨拙的热电站系统不再湃耍了,在那里,热首先要把水转化为飨汽,妹汽驱动沿与发电机转子相联的涡轮。这种设备已经过时,总有一天它会彻底绝迹。与此相较,半导体已经作为趾才发电机而得到应用。这种发电机能将煤汕灯的热转化为电。用半导休制成的电冰箱没有运动部件。
但这些只是个开始。一个灿烂的未来正在招负价这些奇妙的小晶体。
