\chapter{附录}

I本书论及的若干重要公式

l.牛顿第二定律03、6、40)
f=ma
2.万有引力定律06)

f=k
m1m2
r
l

3.库伦定律(§50、74)
f=k妞
rl

4.斯蒂芬-波尔兹曼定律(热辐射第一定律):
R=rJT1
黑体的辐射能力,即它每秒以光和热的形式发射的能匮正比于绝对温度的四次方012)o<f=5.71•10一5尔格/秒·厘米2•度%
5.维恩位移定律(热辐射第二定律):

匕=—c'
T

随着黑体浪度的上升,对应着它所发射的光线的最大亮

•292•
  
度的波长丛将要变短并向光谱的紫色区移动012)。
C'=2886微米·度。
6.瑞利-金斯定律:
从=压亡欠T心c3

热物体辐射强度正比于它的绝对温度,而反比于波长的平方(§13)。式中V为频率,其与波长的关系为心=c,C为光速。
7.普朗克公式:
	氏,T=卫•	v3
	cl	eh吹T-1

r:,T代表单位频率间隔的面发光度。h为普朗克常数,e自然对数的底(§15)。
8.普朗克关系:
E=hv
E为量子的能量,h=6.6X10一”尔格·秒,"频率(§17)。
9.德布罗意关系(§69):
h
儿=--
mv

10.经典力学的波方程(§40):

x=acosc.o(t一9

11.薛定谔方程(§40):

•293•
  
8甲沪
访—=--守IJf(自由粒子)
	at	2m
8甲炉
in-=--勺2少十UIJf(在以位能U表征的场中)
	ot	2m

妇)十竺E凶=0(自由粒子,定态)胪

2m
勺协十—-(E-U)中=0(在以位能U表征的场中,定态)方?

12.表征光子的几个数星(§41):
	h	h
E=加(能批)p=-(动量)m=-(质量)儿},c
13.海森堡测不准关系(§42):
~x•~Vx~¬h
m

14.自由粒子波函数飞§48):
伊=中oXe11
-L(Et一r·P)
15.壳层容量N与壳层序数”的关系(§54):
N=2n2

16.测不准关系的另一种形式的表述(§49、57、99):k
AE"'-
At

*波函数的平方具有几率的意义,是本书最五要的内容之一。光的波动说认为光在某处的强度正比于该处振幅的平方;根据傥粒说,则正比于该处某体元内找到一个光子的几率C将两种学说的吻合推广到实物粒子,得出波函数平方具有几率恣义。一一一译者注

•294•
  
,;

1
/;);(I)"'-
At

式中(J)=2吵
17.爱因斯坦相对沦公式(§85、86):

f11v=
m。
....—一
~1-v2炉
Iv=t。~1-句,2
18.爱因斯坦方程(§86、89、95、99):
E。=moc2

II本书中出现的主要量子物理学家简介(按英文字母顺序)
玻尔(Bohr,Niels)1885-1962,丹麦人。提出电子轨道概念,对应原理等,是旧量子力学主要创始人之一。笫二次世界大战前与夫伦克尔提出核液滴模型。曾获1922年诺贝尔物理奖(§21、23一26、49、75)
德布罗意(DeBroglie,LouisVictor)1892一,法国人,1924年提出对匮子力学发展有决定意义的物质波。曾获1929年诺贝尔物理奖(§7、27一33、96)。
狄拉克·(Dirac,PaulAdrienMaurice)1902一,英国人,1929年提出”相对论性不变“原理。与薛定谔合得1933年诺贝尔物理奖(§84、87、88)。
爱因斯坦(Einstein,Albert)1879-1955,德国犹太人,1905

•295•
  
年创立狭义相对论,提出光子学说,1916年创立广义相对论。晚年建立“统一场“理论,但未达预期目的。伟八导师列宁称他为大革新家,敬爱的周总理对他作过高度的评价。曾获1921年诺贝尔物理奖
海森堡(Heisenberg,、Werner)1901一,德国人,创立懿子力学的矩阵形式,他的测不准关系在量子力学中有重要地位。曾获1932年诺贝尔物理奖(§34、40、42、49、57、99)。
李政道1926一,美籍中国物理学家,与杨振宁提出在弱相互作用中宇称守恒能够失效。与杨振宁合得1957年诺贝尔物理奖(§IlO)。
普朗克(Planck,MaxKarlErnstLudwig)1855一1947,德国人,1900年提出能量子假说,是篮子力学的第一个创始人。获1918年诺贝尔物理奖(§15-17)。
泡利(Pauli,Wolfgang)1900一1958。奥地利物理学家。泡利不相容原理在量子力学中占有重要地位。1934年与费密提出中微子理论,25年后被证实。曾获1945年诺贝尔物理奖
薛定谔(亦译作施勒定格)(Schrodinger,Erwin)1887一1961德国人,他的波方程是量子力学中最基本的公式。与狄拉克合得1933年诺贝尔物理奖(§34、40..45)。
杨振宁1922一,美籍中国物理学家,与李政道合得1957年诺贝尔物理奖(§110)。
汤川秀树(Yukawa,Hideki)1907一,日本物理学家,提出核交换力和冗介子理论。获1949年诺贝尔物理奖(§68)。

•296•
 

,,气
  
III量子力学发展大事记

1687牛顿《自然哲学的数学原理沉在伦敦问世(§13)。1872斯托列托夫进行光电效应实验(§18、94)。
	1881	迈克耳孙进行光在以太中相对不同参照系的不同传播速度的试验,得出否定结果(§4、7、91)。
1896法国贝克勒耳发现放射现象(§66)。
1898居里夫妇发现锦、专卜、铀的放射性(§66)。
汤姆生建立笫一个“正电云“原子模型(§22)。
	1900	普朗克提出热辐射公式,并假定能晕以量子E=加形式发射,量子力学诞生(§15-17、93)。
列别杰夫发现光压(§94)。
1905爱因斯坦提出光子理论、狭义相对论(§14,18一20、
85一87)。
1911卢瑟福提出原子的行星模型07、22)。
	1912	玻尔用量子理论解释光的发射和光谱,提出电子轨道概念(§23一26)。
劳厄进行X光衍射实验(§31)。
1914考塞耳奠定量子化学基础(§26)。
1916爱因斯坦提出广义相对论(§93、100)。
索末菲提出有关光谱起源的更确切的理论(§26)。
1919卢瑟福以“粒子分裂氮核而获得氧核(§66)。
探测队在阿拉伯沙漠观测日蚀,所得结果符合广义相

•297•
  
,
 

对论的预期(§,93)。
1924德布罗意提出物质波的假设(§27)。
1925乌兰贝克、古兹米特提出自旋(§88)。
1927在比利时首都布鲁塞尔的科学大会上,海森堡和薛定
谔提出了新的量子力学理论,奠定了新的量子力学基
础(§33)。
1928德拜·谢勒重复劳厄的X光衍射实验,获得一组组衍
射环(§31)。
美国科学家戴维孙、革末,苏联科学家塔尔塔科夫斯基
成功地进行了电子衍射实验(§31、34、38一40)。
1929狄拉克提出”相对论性不变”原理(§87)。
1932布拉克持、欧卡尼里在宇宙射线中发现正电子(§91)。
查特威克发现中子(§67)。
海森堡、伊凡宁科、塔姆提出核由质子中子构成(§67)。
1934塔姆和汤川秀树提出核交换力理论。汤川秀树预言冗
介子(§68)。
小居里夫妇用中子击核而产生人为放射现象(§80)。
泡利、费密提出中微子理论,二十五年后中微子被发现
(§8I)c
宇宙射线中发现µ介子(§69)。
1939发现铀核裂变c玻尔和夫伦克耳为此提出核液滴模型

(§75~、76)。
1940福辽洛夫、彼得夏克发现重核天然裂变(§78)。1945美国在日本广岛、长畸投下原子弹(§118)。

•298•
  
~,1947鲍威尔在宇宙射线中发现汤川秀树在十三年前预言的
冗介子(§69)。
1954第一个原子能发电站在苏联建成(§118):)1955反质子被发现(§105)。
1956反中子被发现(§105)。
	1957	李政道、杨振宁提出宇称守恒能在弱相互作用中失效、“联合反演“理论,获得诺贝尔物理奖(§110)。
	1959	闪烁计数器的双闪烁证实了二十五年前泡利、费密预言的中微子的存在。

