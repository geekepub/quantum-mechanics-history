\chapter{从原子核到基本粒子}

§84一个新世界的发现

这样看来,什么东西又能比原子核更坚固而稳定呢?高压、极低或极高的温度、强大的电场和磁场都对它亳无影响。自然界最坚固的建筑物就是核结构。这就是物理学家过去的看法。
科学的发展巳经在柜当大的程度上改变了这种观点。大多数重核证明是很不稳定的。甚至在轻的和中等质屈的核里也颇有几个结构是脆弱的。当自然界在核建筑中垒传的时候,核内的质子和中子的比例稍稍失调就将助长不平衡—¬这点是越来越清楚了。
与此同时,为了说明在蜕变过程中,从来不曾存在于核内的粒子能从核内飞出,科学不得不假设核内的中子能转变成一个质子,或与此相反。这便引向中微子的概念。
核稳定性看来归功于一个叫作六介子的新粒子。在寻找这个新粒子的过程中,物理学家却发现了µ介子。
逐渐地,科学家开始明白:那个用来建造原子及原子核的种种砖石,并不象人们想象的那样稳定,那样一成不变。在原子的深处,在原子核的更为幽深的世界里,察观人员遇到的

•201•
  
事件远比鼠子力学所能预期的都奇妙得多。
但拭于力学也能对付这个新局面。它的独特的预见性在基本粒子的新世界中也巳盛开花朵。每当一个新的预见被实验杰出地证坟了的时候,怀疑派总要大吃一惊。在超微观世界里,钉前进一步都是对常识的一个否定,这就更使怀疑派惶惶不安。
常识?!贝的,如果科学家只知逍让一般常识牵着鼻子走,科学到今天还会以蜗牛的步子爬行着呢!
当窃识被弄得神魂颠倒的时候,最重大的发现往往就会做出。许多书物的真正本质并不浮在表面,却隐藏在它们的深处。那些平庸的和显而易见的,往往是具有欺骗性的。当科学深入到奇妙的微观世界之中时,上面所说的“往往”就成了“经常”。
让我们返回到1928年。此时这个新世界才刚刚开始展示它的底皇我们只知道两个粒子:质子和电子。量子力学刚刚三岁。可是它却非恺成功地解答了一连串的谜。我们终于正确地掌握了氢原子和氢分子的形成,而隧道效应也开始帮助我们了解放射性核的a射线。可是对于核及其他粒子,我们实际上仍然一无所知。
这时有个年轻的英国科学家保罗·狄拉克(他们真的都非常年轻:薛定谔38岁,海森堡28岁,狄拉克25岁)抛头露面了。他说:噩子力学的成就将象昙花一现,因为它脱胎于经典物理学,而后者描述的只是物体的相对缓慢的运动。
可是,在玻尔理论中电子每秒钟绕核旋转几万亿次。我

•202•
  
们难道能认为原了中的电子,即使在这个旧理论中,运动得缓慢吗?在轻核中,屯子速度的数堡级是每秒几于公里,在重核中则高达每秒几十万公里。
肯定是不慢了!这意味着我们必须将垃子力学应用到原子杻子的这些快速运动上去。怎样才能做到这点呢?
在此以前二十年,一个正好是描述寻常物休的快速运动的理论出现了。它就叫狭义相对论。作者是阿耳伯特·爱因斯坦。
狄拉克断言:将从子力学扩展到微观粒子的快速运动上去的途径,是把它和狭义相对论结合起来。

§85看不见的分界线

在这本小书里,我们当然不能详尽地描述狭义相对论。
详尽的描述需要另外一整本书。因此,我们只大略谈谈和我们的故事直接有关的地方。
首先让我们弄清楚,什么样的运动将被称为快速运动,什么样的运动将被称为绥慢运动。在日常生活中这是一目了然的:蜗牛的步子慢,喷气飞机快。
主观的吗?是的。这里,快与慢是以人的运动-—-走或跑—一来量度的。
可是你如果从很远的地方来观看一列快速火车,你又觉得它走得多么慢啊。远在天际的喷气飞机也慢起来了。甚至一颗人造卫星看起来也飞得不快。快与慢是相对的概念。事

•203•
  
实上它们也确是这样。
因此物理学家不满足于习惯上的快慢概念。他们需要某种不依赖人类为转移的恒定的址度,以便用它来计算其他所有的运动速度。
那么也许能拿地球环绕太阳的轨道速度来作标准吧9总地来说,倒也不坏。既然人类已经使用望远镜来窥测宇宙的深处,找出一个不与地球、太阳或任何特定的天体相联系的抵度可能更好一些。这样,它将适用于宇宙间的任何事物:它将是普遏的址度。
大自然恩赐了这个作为匿度标准的速度一一真空中电磁波的传播速度,也就是真空中光子的飞行速度。这个速度大约为每秒300,000公里。在所有已知速度中,它是最快的3
再没有北它更快的速度了。相对光足而占的所有运动都是比较慢的3物理学家使用快速一词来描述接近于光速的那些运动。这是一个颇为武断的划分,实际上也是传统的作法,可是此中却有深奥的道理。
当速度接近光速时,物体的特性开始发生显著的、出乎意料的变化,包含着大械粒子的物体尤其是这样。这里,说明这些变化是容易的。
一个显著的变化是:当物体接近光速时,它的质量随着增加。1-:i光速愈接近,质雇增加得愈大。从外部来看,物体开始抵抗心个增加其速度的力。结果是,为了增加物体的速度,施加给它的力也必须越来越大。
但没有一个力量大得足以使它以光速运动。相对论声

•204•
  
明:不可能使任何一个实物以光速运动。这里,所谓实物指的是任何仁够静止的物体(或粒子集团)。光子是不能静止的-这点以陌还耍谈到—一-因此相对论不能应用于它们。
这个观念如果用数学语言来表达的话,就是下面这个著名的方程式:

m=
m,
小-句ca
这里m.,是物体以速度V运动时所具有的质量;m。叫作静质量,即物休不在运动时所具有的质量;C为光速。
从这个关系式可以清楚地看到:随着U接近于c,分母减小,开始减得很慢,然后越来越快。由于m。是一不依赖于速度的恒蜇,因此m也就相应地增大。最后,当u等于c,该物体的质量m,就变为无限大。换句话说,这个物体应有一个无限大的质量。
显然只有无限大的力量才能做到这件事。可是在自然界里,无限大的质量或无限大的力量都不存在。宇宙本身是无限的,除此不存在任何其他无限的东西。
我们已经说过,这个公式不能用于光子。说得更恰当些,如果将光子代入这个公式,它不能给出任何东西。光子不可能静止。换句话说,光子的静质批是零。将这个值代入关系式中的m。,我们发现,对于速度等于e的光子来说,它的质量m斗将会是%。数学说这个值是不确定的,这意味着:它可以取任何一个数值。
事情确实是这样的。光子的质量可以是任一数值,可以

•205•
  
大,也可以小,这一点以后还要谈到。但这个质盘只有当v=c时才存在。所有这些意味着:光子只能以光的速度运动行。
你瞧,这就是光速!任何实物粒子不能具有光速;与此同时,任何光子也不能具有任何其他速度。因此光速就是介乎实物粒子和光子之间的不可越逾的鸿沟。
为什么我们在日常生活中观察不到相对论所预期的质抵增加呢?让我们考察一下以每秒11公里的逃逸速度飞运动着的火箭的质坻增加对它在地面上的静质掀的比率究竟有多大。如果它在地面上是100公斤,则11公里/秒的速度只能将其质猫增加0.35亳克!
可是如果我们将速度升高到250,000公里/秒,则它的质址将增大到静质阰的两倍以上。这就是带电粒子被加速到高速时所发生的现象。这种设备就叫作粒子加速器。因此,质益增加正是设计师建造这种设备时必须牢记的事情。

§86再谈一点相对论

物体速度接近光时,还会出现许多其他不寻常的现象。
质斌在变化,时间过程的本身也在变化。物理学家称之为物体的固有时间。看来好像我们的身体也各有自己的“钟'为这些钟的摆动与我们身体的生理过程的节奏是一致的。

*即第二宇宙速度,火箭具此速度即可克服地球引力,作为行星环绕太阳旋转。一一译者注

•206•
  
另一方面,我们起床、工作和睡觉根据的是一般时钟的时|、乱这些时钟的运转是与昼夜的交替,即地球绕自轴的旋转相吻合的。
“时间过得真快啊产”时间怎么过得这么慢呢!”一一这里说的是找们自己的时间,也就是与我们身体的功能相吻合的主观的时间。但这种时间也有其客观的一面。那就是身体功能的节奏愈快,我们感到时间过得愈快。
相寸论中也存在着与上述情况十分类似的东西。相对论声明:物体运动得愈快,它的固有时间流逝得愈慢,因而从该物体的灼度来观察的普遍时间流逝得愈快。
时钟谜已成为今日茶余饭后的谈话资料,这是因为它关系到远距离的宇宙飞行。在科学幻想小说里,乘坐接近光速的光子火箭的几个宇宙航行员,完成了历时十年之久的空间旅行之后返回地面,他们将发现大地上的朋友已经老态龙钟了。“天上方十载人间几沧桑'\他们会这样说。这是真的,因为在宇宙飞船上,固有时间比大地时间要慢。
我们关于时间所作的描述可以用数学形式表示如下:
t,.=t。v'1一句c2
式中t,是宇宙航行员手表的固有时间,to是大地时钟的时间。方程式中其他的量具有一般的意义。从这公式可以导出:对以光速飞行的光子来讲,时间并不流逝:如果我们能将时钟安置在一个光子上,时间将会从此停滞:时钟不再走动。
柜对论还有许多似非而是的地方,我们就不在这里一一讨论c但还有一个方程往后将起着非常重要的作用。这就是

•207•
  
`
著名的爱因斯坦方程。

E。=m.ic2

这卫,队是具有挣质堪mo的挣态物体的能盄。为了将它和动能或泣能区别开来,我们称之为物体的静能鼠,固有佬量或原能。
我们将要肴到,挣能量是不依赖于物体的速度或位置的。
经典物理学只知逍两种类型的能卧。这一新型能抵在经典物理学中是没有地位的。它是一种非常特殊的东西,这点往后还要讲一些。目前让我们回到相对论上来,看看它是怎祥被引入氓子力学的。

§87最早的困难

这样狄拉克便尝试把二十世纪的两大理论结合起来。在来自敬观怅界的新事实猛烈冲击着量子力学的情况下,这一新合金应对后行起到加强的作用。
薛定谔方程曾是打开自然界各种保险箱的趾子力学万能钥匙。但它对某些书实却感到棘手,因此人们便去寻找改进它的办法。
很快就发现:将这个方程和相对沦熔冶在一起绝非易事。狄拉克所想的第一件书就是:这改进了的方程会给出相对论性不变彴解。但往后的甘实证明,他也不完全正确。天知逍,若不是出了个幸运的错误的话,狄拉克还可能会放过一个重大的发现哩!

•208•
  

、.....
 
相对论性不变,--这些字眼真够吓人。说真的,对所有物理理论来说,这都是一份可怕的宣判书。贴上这个标签的理论就诊扔进字纸婆:它不可能是个好理论。
可是事情的奥秘就在这里。你曾经在船上玩过球吗?设想一下在飞机上玩球和在地面上玩球有什么两样?是的,没有什么两杆。当然,这里有个先决条件:船或飞机必须在作匀速运动3
没有任何方法能将静止和匀速运动区分开来,不管这速度有多大。若没有屁夜的交替,我们就不可能察觉出地球的自转。若没有四时的变化,我们就不会知道地球在绕太阳运动。严格地说来,这两个例子并不十分正砾,因为旋转总是加

速运动。但在我们这两个例子中,加速度是那样地微小,因而我们能够把这两种运动都看成是匀速的。
在接近光速的飞船里,物体的各种运动将与地面上的那些运动没有两样(当然须要假定引力是相同的,也就是假定在飞船内也用某种办法人为地产生出同样的引力)。既然物休的运动不依赖于用以计算其在空间与时间中的位置的参照系的速度一一不管这参照系是地球还是飞船,这些物体的运动
定律也同样不依赖于参照系。
在所有的参照系中,不管这些系统彼此在以什么速度作相对匀速运动,运动的方程必须是一样的。换句话说,相对于不同的速度,这些方程必须不变。
上面的句型”相对于...…不变“用物理学的语言来表达就是”相对论性不变%它的晦涩的含义就是:如果一个方程声

•209•
  
明,在接近光速的飞船上一个球描绘出一个双曲线,而在地球上则描绘出一个抛物线,则此方程是错误的,因而必须被抛弃。
当人们为改进薛定谔方程而作出种种尝试时,就遇到了上面听说的问题。

§88一个意料之外的发现

狄拉克在寻找出路的过程中,曾做了一件不寻常的事:他把四个波函数引入薛定谔方程,以代替原先的一个波函数,结果得出了一个大不相同的方程。这个新的方程给出了极佳的相对论性不变的解。
与方程中波函数的数目相对应,方程有四个解。但我们如何理解电子的四种几率,以代替原来的一种呢?
如果电子的自旋没有在三年前被发现的话,前面两个解的意义在许多年内可能还是模糊不清的。
因此狄拉克方程的前两个解对应着两种可能意义下的相对于电子运动方向的电子自旋。这个自旋就是从解中计算出来的,而计罪数值被证明与实验结果极佳地吻合。
这里有必要谈一点自旋。首先,自旋对应着接近光速的电子的某种运动。当然,如果我们暂且把自旋解释为电子绕自轴旋转的结果(我们说过,这样的概念是完全错误的),我们将发现,电子在这种转动中的速度比光速也只小千分之几。
显然,当我们谈到自旋时,我们所论及的运动与电子在一

•210•
  
般空间中的一般运动毫不相于。电子的自旋完全不依赖于一般运动。不管电子在作快速运动、缓慢运动或是静止的,这个自旋一杆存在,自旋的数值也永远不变。
自旋足粒子的一个内在的、本质的特性,正象其静能一样。改变具自旋,就是改变粒子的类型。这点,我们很快还要谈到。
自旋是怎样体现出来的?我们讨论原子光谙时,曾稍梢涉及到这个问题。远在上个世纪末,人们就发现:如果一个材料被驾于磁场中,它的光谱线便会分裂成许许多多微弱的线。往后证实:所有元素的原子的光谱线在磁场中都会产生类似的分裂。
但这一现象(即塞曼效应)的本质,尤其是一条光谱线分裂成不同数最的伴线的原因,直到1925年两个年青的物理学家乌兰贝克和古兹米特引进了自旋概念之后,才得到说明。
往后的推论是这样进行的:电子具有自旋,归根到底,也就是具有角动量。它的起源尚未使我们感到兴趣,重要的是,它对应着电子的某种运动。但电子的运动就是电流——由单个粒子产生的电流。当然,“真正的“电流是大数量的电子的运动造成的。
人们知道电流的磁作用已有一百余年。换句话说,一个电子可以被描绘成一个微小的永久磁体。如果把这样一个基本磁体放进磁场中,它会在这个磁场中取向。最简单的取向有两种:一种与磁场一致(绝对稳定),另一种与磁场相反(绝对不稳定)。

•211•
  
什么又是稳定性呢?当磁体方向与磁场一致时,它在磁炀中的位能是一个极小侦;在逆磁场情况下,位能是一个极大忙。
这两个能也相差多大呢?这可以哏容易地计鲜出来,并换算成具有顺磁场自旋的电子和逆磁场自旋的电子,在原子中发射不同光子的波长之差。
就这样,在磁场内每条双谱线都分裂成若干条谱线,分裂的倍数恰好与具有相反自旋的电子对数目一致。
下面谈谈那些分裂成三条、四条以及许多条伴线的宽谱线究竟是怎样一回事。自旋的两个取向只能产生一对伴线!而且不能多于一对,因为电磁体会直接跳到最稳定的位置上去。
这里我们应当记得:电子不仅具有自旋运动,而且还有围绕原子核的轨逍运动。实际上这一轨道运动也是一个复杂的准运动。几率云概念不允许我们为原子内电子的运动,想出一个比几率云本身更为生动的形象。
但它仍然是运动,是一种单粒子引起的电流,因此它起的作用也与一个小磁体相似。看来事情是相当复杂的:原子中的电子就像是一个双重磁体。
这样一个磁体在磁场内将起着什么样的作用呢?很奇怪,磁休的取向不是两个,而是三个、四个,甚至很多个。当电子的基本磁体从较不杞定的取向过渡到比较稳定的取向时,它可佬在一系列巾间位置上停留下来。这些中回价詈的能量,是磁体的两个极端位置能讯差数的一个简1'(_1分数。这就

•212•
  
意味着:它们是一些十分确定的能晕,能鼠间距是星子化了的特定数值。原子中的电子磁体在磁场内的确定取向现象,物评学家叫作空间量子化。
且余的就好理解了。一条光谱线分裂成一定数目的伴线`这数目恰好就是电子磁体可能具有的取向的数目。伴线波长差别的计算值也同样极好地与实验结果相吻合。
关于突然出现在狄拉克方程中的自旋,暂且说到这里。
这个方程还有两个解。

§89一个更惊人的发现

正象前两个韶对应着电子自旋的相反方向那样,后两个解也彼此十分相似。
这屯也有阱个对立面:一个对应着芷的电子总能量,另一个对应肴负的电子总能量。你也许会说,这没有什么好大惊小怪的;我们知道,总能屈既可取正号,也可取负号,究竟取什么号,那就要看电子是在自由飞行中,还是与其他粒子有着联系,象在原子中那样。
但狄拉克方程只是为自由电子建立的!
哼!这样说来,狄拉克电子是自由的,同时又是被束缚着的。真是胡言乱阳
狄拉克自己也认为这是胡说。当然,最简单的事情就是把你不需要的东心扔掉,正如在计3和中得出士20平方米的房屋面积朊所做的那杆。这里负值是与常识相抵触的。找们同

•213•
  
样也可以拒绝接受自由电子的负能量,因为它亳无物理意义。
狄拉克并不急于这样做。作为一个英国人,他是很通情达理的。但作为一个科学家,他决心探求这个乖谬的根凉。在他石来,甚至谬论也可能具有某种意义。
狄拉克终于获得了一个激动人心的概念。可能那个疯狂的解不是屈于电子的,而是属于带有与电子相反的电荷的某个其他粒子的。既然电子的电荷是负的,因此这个粒子就应当带有正电荷。当然,这两种电荷在绝对值上应当相等。狄拉克原以为它可能就是质子,但也很快就发现,这个负能显必须属于一个质量恰好等于电子的粒子。质子是肯定不行的,因为它的质昼大约要比电子大两千倍。唯一的可能就是:它是电子的镜巾映象。
当然,上面这个描述还没有觯释这种粒子的总负能。如果这个能晕是负的,那就意味着粒子是束缚于某种东西之中的。电子是绝对地自由的,而其他粒子,根据该方程的解,又与电子相距如此遥远,因而它们之间的电相互作用可以忽略不计。电子孤零零地在无边元际的绝对虚空中运动着。我们又从何而得出这第二个带正电的粒子-~电子在镜中的映象呢?
这时狄拉克获得了一个最最疯狂的概念。除了上面那个孤笭零的电子而外便什么也没有的真空根本就不是空的!与此相反,这个真空充溢着电子!在这并不真的真空里,电子带正电的镜象是一个空穴。
疯狂,彻底的疯狂,但此中毕竟包含着某种不容等闲视之

•214•
  
的胆略与见识。
我们怎样去称呼这样一个空间?在其中不管多么灵敏的仪器也测不出一个粒子!
不要着急。让我们先假设这空间包含着不能与仪器相互作用的粒子e这样,即使空间充满了粒子,你仍然会继续把它叫作虚空。
的矶是这样。但粒子怎样会被剥夺掉它们的相互作用的能力呢?它们的这种能力被剥夺了,可是它们的本质依然存在着,这不是自相矛盾吗?
下这样的结论还为时过早。让我们先用一个较翡的电场来探测一下金属的结构。我们发现了电流,因而说金属充满了自由电子。当然,如果我们只局限于这一个实验,我们将会得到一个关于金属的错误概念。因为金属内还有许多原子,它们的电子不能与仪器,例如电流计,相互作用。这些电子深居于原子的低能级中,即阱中。由于它们没有足够的能匮,因此不能从阱中跳出并与仪器相互作用。
当然你也可以说,只要仪器不同,实验方法不同,也有可能发现金属内的原子,甚至原子核,可是真空却是任何仪器也探测不出来的。因此,真空里没有任何东西,也不可能有任何东西。
以上就是常识的意见。可是狄拉克的意见则不一样。
真空充满了电子。整个的宇宙都参与了这种统一的、茫茫无际的真空的构成。无穷无尽的电子充斥着真空的无限数量的能级,从而形成了一个统一的相互联系着的粒子集体。

•215•
  











  

1.
 
然是个梦想。
但是当电子在真空里的时候,它为什么就不能象在金属里那样,和仪器相互作用?又是泡利原理给出答案。
物体间的每一相互作用都是它门能量的改变。正是通过这种能最的改变,我们才能发现这一相互作用。如果真空电子与仪器相互作用,则该电子将改变其能量而过渡到其他能级。但困难在于:真空中所有能级都充满着电子!这里简直没有空符的地方。
这就是为什么真空电子不能被察觉的原因。它们生存于真空中,但不能彼此相互作用,也不能与任何仪器相互作用。在无限漫长的岁月里,这些电子与我们共处着,我们从未猜想到它们的存在,因为它们也从未以任何形式留下自己的踪迹。

§90"空穴”的诞生

让我们假设由于某种原因某一个电子获得了足够的能量,并从真空中跳了出来。电子自由了,因而它的总能匮变为正的。但真空里会发生什么呢?
一个空穴形成了。电子在真空里的旧址仿佛是离子化了。这个旧址获得r一个数值等于电子电荷的正电荷。
我们记得讨沦半导体时就曾谈到过空穴。在半导体里电子跃进传导职囚加在填满了电子的价带内留下一个空穴。这个空穴在价带内具有负能量。这确实是个醒人耳目的类

•217•
  
比,但类比也就到此而止。在半导体内,空穴确实是个“空址”,引用它的目的纯粹是为了便利于描绘价带和传导带内不同类型的电子运动。
但真空里的空穴则是十分不同的东西。这里,它与电子没介什么不同。它是个真实的粒子,它的真实性一如电子。和电子一样,空穴具有静能昙m。产,这个能黑等于真空位阱的最高能级的深度。
换句话说,电子和空穴只能成对地产生于它们的真空的鸟h之中。用于产生每一个质晕相等、能噩为m。c2的粒子,因而产生对粒了的能量为2m。c2,这点,前面已经说过了。
电子能邀沥于自由世界,然后返回真空。要做到这点,它必须遇到一个空穴,并与之融合。这样,电子就再次成为无法观察的,而空穴也一齐消失。
但书悄还不止如此3在返回真空以前,电子必须交出用来将它从真空中抛射出来的能批,换句话说,即用来产生电子和空穴的能址-—-2m。产。
交出的能屋将以什么形式表现出来?以r光子的形式。
r光子将在电子与空穴融合的地方飞出并将此能量带走。
最后的问题是:为什么要由丫光子来接管这一能显呢?这是因为:电子-空穴偶在坠入乌何有之乡的时候,交出的能星足够与硬丫射线相对应。产生出来的光子数目绝不少于二(但也极少大于二),这是因为进行融合的电子和空穴具有相反的自旋。
这一切都是很自然的:由于真空中电子和空穴的整个动

•218•
  







  
狄拉克的理论即使在作出正电子这一项预见后便停步不前,它也会在物理学中赢得一个光荣的位置。事实并不是这杆。狄拉克打开了物理学家的限睛,使他们看到了微观世界的全新景物。
首先就是关于真空。根据狄拉克理论:真空充满了电子,这些电子不与真空之上的粒子相互作用。当电子离开真空时,正电子立即产生出来。电子和正电子只能一对对地诞生、死亡。
那么我们或许也可以说真空充满了正电子,只是当正电子离开真空时,电子才出现。狄拉克原理在其最初形式中认为二者是同样地可能的。当然,我们宁可认为真空中充满了粒子而非反粒子。理由是我们的周围到处都是电子,而正电子倒真地是我们这个世界里的稀客。从这点就应该得出结论:在这个世界里,正电子要远远少于电子。但狄拉克原理声明:一个粒子只能与另一个粒子-—-它的反粒子一一同时产生。这就意味着:在这个世界里电子的数目应当与正电子的数相等。
奇怪!更奇怪的是我们这个世界存在着,而我们自己也存在着,因为没有任何东西能阻止电子与正电子相碰撞而坠入真空并化为摆脱了驱壳的幽灵一光子。
可是,电子并不经常遇着正电子,事实上这种遭遇是非帘罕见的。因此没有理由担心我们这个世界会变成真空。这样看来,电子确实多于正电子了!可是正电子又往哪儿去了呢?

•220•
  










  
粒子在获得足够的能垦后只能一对对地从阱里出来。当然,最先逸出的是最轻的粒子:中微子和电子。对一个质子和反质子来说,这个能量至少耍比给与电子-正电子对的能屎大两千倍。粒子愈大、愈笨重,使它从真空中出来也就愈困难。

§92再谈真空

当电子-正电子偶消失时,我们知道一对高能丫光子便告诞生。但为什么诞生的恰恰是光子而非其他东西?这点我们现在仍然不清楚。
当弹子彼此砚撞时,我们看到的相互作用是:一个球朝若某个方向飞去,另一个球则继续向前运动。你试用下述办法使一个静止的球运动:让另外一个球几乎擦边而过但又没有接触到它。这简直就像马拉着车又没有套在车上一样。
在这两种情况中,物体的相互作用都由于接触:一个球击中另一个球,马拉着车。
还有另一种类型的相互作用。苹果坠落于地。磁石吸引着铁。带电球相互吸引或排斥。吸引这个词本身就表明:物体已经开始隔着一定的距离相互作用。
可能这个相互作用是通过空气传递的。对此实验早就作出了否定的回答。地球吸引着月亮,太阳吸引着月亮和地球,虽然介乎它们之间的几乎就是真空。原子核吸引电子,虽然它们之间是绝对的真空。所有这些表明:物体不通过任何接触也能相互作用。

•222•
  
一个世纪以前,物理学家给超距作用发生的空间场所取名为“场”。但他们并不准备接受这个事实:介乎它们之间的空间是空的。
作用不可能没有中间媒质而产生。必须要有媒质。这样,他们就想出了一种充满整个虚空的、极为稀薄的东西一一以太。
许多年里,物理学家尝试弄清以太的特性,而这些特性,本书一开头就说过,是真叫异想天开的,甚至是自相矛盾的。最后,在上个世纪末,对光进行的实验埋葬了以太概念。又过了几年,爱因斯坦的相对论表明:以任何方式或形态来挽救以太都是亳无希望的。
以太寿终正寝了,可是又没有东西能代替它。物理学家最后让了步,接受了虚空中的超距作用。但是虚空又怎么能成为超距作用的传递者呢?甚至最伟大的思想家也不能理解。虚空是一无所有的,是个0。
这就是你的想当然。可是我们愈依赖于常识,我们就愈难摆脱它的秷桔。亳无疑问,空间是万物的贮藏所,这难道不是明摆着的吗?为物质占据着的一部分空间叫作物休、粒子,等等。但也有不被任何物质占据的空间。我们把它叫作虚空,空的空间,真空。这两部分是没有任何联系的。真空对物体不起作用,物体也不与真空相互作用。当然,物体之间能通过空的空间相互作用,但虚空这里是无足轻重的:相互作用仅仅起因于物体本身。

•223•
  
§93空间依赖于物体

后来出了这么一个人,他不仅对上述说法怀疑,而且对每件事物都从头到脚重新审查。他就是爱因斯坦,他的理论就叫广义相对论。他的第一个理论,即狭义相对论,讨论的是快速运动的物体,这点我们已经说过了。广义相对论包括远较前者更为广泛的间题。一句话,它讨论的是物体与空间的相互关系。
广义相对论的主要观点在于声明:物质影响着它周围的空间。没有物体的空间是绝对地均匀的(这当然是想象的空间),但只要物体被带进其中,它便会失去这均匀性。
这是怎样发生的?我们又怎样量度这种不均匀性?这件事就让几何来做吧。一无所有的空间的几何就是我们在学校里学的一般的几何一一一欧几里德几何。在这种几何中,两点之间的最短距离是一直线,平行线永不柜交,还有其它的显而易见的陈述叫作公理,而公理就是一目了然的、不须任何证明{顺便提一下,也不可能证明)的定理。
但在上个世纪初,俄国儿何学家罗巴契夫斯基发现其中一个公理(平行线公理)有个瑕疵。他声明:如果抛弃这个公理,则有可能建立起另一种几何,它也如欧几里德几何一样,毫无内在矛盾,但却与常识格格不入。罗巴契夫斯基儿何是那样地不寻常,那样地难以捉摸,以致当时没有人懂得它。年复一年地过去,罗巴契夫斯基的告作在大学图书馆的书栗上

•224•
  
沾满了灰尘。
罗巳契夫斯基同时代的人被他的观点吓倒:根本就没有什么一般的、能应用于所有的世界的几何;每一种几何都是由它所讨论的具体物体的特性来决定的;空间的几何依赖于存在其中的物体或其他什么东西,并且依赖于它们的形状。简直是亵汝上帝!凡人兑敢改变上帝赐给的放之宇宙而皆准的几何!
但在爱因斯坦著作中,这些观念又获得了可尊敬的地位。
正如没有物体的空间不存在一样,统一的、均匀的空间也不存在。在物体四周的虚空中,两点之间最短的线,在一般情况下,不再是一条直线,而是一条名叫短程线的曲线。这段曲线的两个端点和物体愈靠近,物体的质量愈大,则这段曲线也就愈弯曲。
`我们怎么能断定这些观念是正确的呢?就让光线来帮助我们吧。由于物体的存在而产生的空间弯曲是非常微小的,在一般情况下是察觉不到的,我们必须将实验在星际空间中进行,并且选择某个大质量星体,如我们的太阳,作为使空间弯曲的物体。如果我们观察一条被认为是直线的线的弯曲,那自然是最方便的。如果我们相信经典物理学的话,这条直线就应当是光线产生的一条线。这也正是爱因斯坦准备给予驳斥的。
让我们将望远镜指向某个星体,并给它拍照。当它的光线经过太阳近旁时,我们再次给它拍照。笫一张相片是在夜间拍摄的,第二张是在全日蚀时拍摄的。

•22S•
  
根据经骈4物理学,这个星体应当呈现在两张照相底片的同一位置上。不管光线是从太阳近旁经过,还是在远离它的地方经过都不应当产生丝亳差别。可是根据广义相对论,光线经过太阳近旁时,它的途径应当弯曲。第二张照相底片应当把该星相对于第一张照片的位移作为这一弯曲而展示出来。
1919年8月,一个特别的探测队开往阿拉伯沙漠,准备对一个全日蚀进行观察。激动人心的消息!照相底片展示了空间的弯曲。不仅如此,弯曲的程度几乎与爱因斯坦预期的数值完全一木r。
从那时起,物理学家关于虚空的概念发生了根本性质的变化。空间不仅容纳着物体,并且还容纳着场。

§94实物和场

(什么是场?物理学家用这个词来描述这样的空间,在此空间中诸物体呈现相互作用。当然,不发生相互作用的物体是没有的;所有物体最终是由粒子构成的,而其中没有一个粒子对其他粒子是漠不关心的。
由于这个原因,场处处存在,时时存在。不仅物体之间有场,物体之内也有场,因为物体之内也存在着没有实物的虚空。这就是场的第一个,也是最根本的一个特性。从这里立即引出另一个结论:场和实物一样,是真实的,普遍的。
场在下述这一重要方面区别于实物:实物是有实体的,

•226•
  
而场是没有实体的C如电、核、引力场)。但我们不能说场是不可察觉的举苹果坠地为例:从物体的运动就可以明显地看出场的作用。
场还显示了另一现象,那就是光-—-一种独自的、自动的作用。早在上个世纪,光就被确证是一种特殊的电磁场。
爱因斯坦在他的光电效应理论中引入了光子。这是个重要的概念。电磁场被虽子化了,也就是说,它以单个粒子一一场量子的形式存在着。光子就是这些场骰子。
场的历史继续在发展着。1872年斯托列托夫发现光能够起礼实物的作用:它能够将金属中的电子轰击出来。1900年列别杰夫发现了光对物休的压力,就好象光是由具有质队的真实粒子组成的。
这两个重要实验和光子概念不可避免地导致如下结论:电磁场具有实物特性,而场蜇子能具有实物粒子的特性。
这就是在实物与场之间的鸿沟上架起的第一道桥梁。与此同时,德布罗意的假说从鸿沟的另一侧建筑着这座桥。电子能够具有波动特性。这就是说,实物能以场的形式活动。
无边无际的、不可秤的场,也能具有体积和质量。占据有限空间并且可秤的实物,也能被剥夺掉体积和质批。
我们现在是否应当作出这样的结论:实物和场应当融合成一个不可分辨的整体,以代替原来的尖锐差异?否!场的实物特性只是当它的凰子的能鼠很高时才是明显的。而实物的场特性也只是当粒子能最很高时才呈现出来。
在低能股的情况下呢?场还是场,实物还是实物。

•227•

  
原能,它便不再是个电子,正如正电子也不再是个正电子一样。自然,它们的质量不能消失得无影无踪,正如它们的能量也不能化为乌有一样。但这个质址已经转变了性质,转变成非实物的、似场的质昂,同时这个原能也转变成场量子--丫光f的能罩。这忭,贞空中毕竟没有什么真的电子了。不妨说,它们只是概念上池、潜在地存在于那儿。
作这种描绘的理由是:真空或虚空或乌何有一般是不存在的。充满整个空间的是实物和场。狄拉克头脑里的真空仅仅是一个生动的形象,以便利于描述实物粒子和场匿子之间的相互转变过程。
本书的作者井不打算牵着读者的鼻子走。他觉得最好先有一个一般的真空概念,然后再采纳一个非习惯的真空概念,最后再把二者都扬弃掉。不管怎样,这就是科学发展的自然途径。
电子碰到了正电子,二者转化成一对r光子。这既然可能,其反过程显然也应该可能:r光子转化成粒子偶。这确实能够发生,只要光子具有足够的、至少是2m。c2的能隆。
光子可以被观察到,可以被记录下来,而且是颇有实质的。另一方面,除了电子和正电子从真空中跃出的情况而外,箕空却是颇无实质的。我们怎样将二者调和起来呢?
实际上也没有什么须要调和起来的。只要能量不大,光子仍然作为光子而被记录下来。一且能量变得足够使一个光于转化成一对粒f时,我们就开始察觉到光子的真空特性:光子能够消失,而取代其垃置的是电子-正电子偶。

•229•
  
这呈真空一词表示实物粒子和场量子相互转化的可能性。这是我们的基本观点。而从本章开始到现在,我们就一直在讨论这个基本观点。
事帖显然比较消楚了。我们既然在实物和场之间的鸿沟上架起了一座桥,交通便可以往来进行了:实物粒子转化成场星子,场风子转化成实物粒子。重要的事情是必须上桥,但这个桥很高——-它的能扯高度是2m。心对电子来说,这个高度表征几兆电子伏特,对质子来说,则表征几于兆电子伏特。
一句话,真空向场作了让步。我们将继续使用真空一词,这是因为它的形象生动。这样,我们就能方便地将真空想象成一个无边无际的海洋,那里海豚般的粒子正在跳进跳出。

§96鲸鱼歇在什么上

精妙的量子力学栖憩在几条鲸鱼之上,其中一条我们直到现在才能解释清楚。正象往昔一样,鲸鱼共分三类飞普朗克量子假说,爱因斯坦相对论,德布罗意关于粒子的波特性的假说。这里我们将对后一种假说进行讨论。
在前几节里,我们正谈着为宇观世界而建立的广义相对论,忽然又跳到微观物体的量子世界上来,这似乎有点不合悄理吧。问题是我们一再强调:在某个规模的世界中生效的定律,在另外一些规模的世界中至少也是不精确的。因此,我们

*三种鲸鱼是:有须鲸(Mysticeti)、有齿鲸(Ondontoceti)和古鲸(Archaeoceti)。这里以三种鲸鱼来比拟三种重要学说。—--译者注

•230•
  
有什么根据将爱因斯坦的物质与空间概念扩展到微观世界中去呢?根据就是德布罗意的假说,那些巳被可祁地核对、证实的假说:微观粒子具有波特性;它们的二象性处处存在,时时存在。但波是什么呢?根据它不确定的广延性和永恒的变动性笘特征来判断,波的实质朗显地是一个场。因此,德布罗意假说实际上就是声明:实物粒子具有场的特性。在这个意义上,它补充了爱因斯坦的假说:场量子(光子)具有实物特性。
微观粒子的场特性是怎样体现出来的呢?我们已经遇见了很多的实例,其中最典型的就是电子和其他粒子在空间的弥漫。芷象物理学家说的那样,它们是不可定位的。电子在此一瞬间既在这里又不在这里*。如果企图准确地测簸它的运动速度,我们便不再能够说出它的位置。这正是场的特征。由于场是无往而不在的,因此确定场的位置是不可能的。
如果我们增加电子的速度,则当它接近光速时会变得更重。它从哪儿得来这一额外的质鼠?电子照例是由电场来加速的。在加速过程中,电场仿佛进入了电子,并将一部分能量赋与了它。由于电子的能量增加了,因而根据爱因斯坦关系(§85),电子的速度和质量也应该增加。
但场将质且泵入粒子这一过程不能无止境地进行下去。
质匮将极迅速地积累起来,最后粒子的动能将与其原能相等

*请苤考«反杜林论>)第117页“十二、辩证法。虽和质芞”运动本身就是无盾:••....物体在同一瞬间既在一个地方又在另一个地方......。l'-译者注

•231•
  
(当粒子速度大约达到光速的80%时,这种情况便会出现)。此刻一个新的过程介入。在这个新过程中,粒子的波或场样的特性居主导地位。处于这种状态下的粒子将一股脑地摆脱其积累起来的能量和固有能量(原能),转变成场匿子。
粒子质愿随速度而增长的原因是大自然赋与它们的一种自我保存本能\粒子不愿丧失其个性,并愤怒地抗拒任何能量的积累,而当它们愈是接近向场的转化,这种抵抗愈是剧烈。
粒子永远不能以场的传播速度运动,而场也永远不能以别的速度传播。

§97粒子更换装束

直到现在,关于粒子的转化我们只谈了电子(当然还有正电子)。中子被发现后,人们认识到中子也能转化,但与电子不同,它不能转化为场量子,只能转化为其他粒子。
首先,中子能转化为一个质子,一个电子和一个中微子(在fJ衰变中);但只有自由中子才能发生这样的转化。在核伪,中子转化成一个质子和一个冗介子。往后发现:中子的笫二种转化与笫一种转化并没有很大的区别。一个自由冗介子衰变成一个µ介子(其质星相当于前者的3/4)及一个中微子。这个µ介子又衰变成一个电子,一个中微子和一个反中微

*请参看斯宾诺莎«伦理学》第三部分第VII命题:“每一物为保存其自身
	俨	而作的努力正是此物的本质。”一一译者注

•232•
  
子。这样我们就有:自由中千虳衰变:
中子-质子+电子+中微子
`核'中子的衰变:
中子一质子+亢介子
元介子-,i介子+中微子
µ介子-电子+中微子+反中微子
结果:中子-质子+电子+2中微子+反中微子
当然,用这样一个账单来说明两种转化的相似总的来说是行不通的,理由是:兀介子在核内并不衰变。我们知道,作用于核内粒子的是电力以及强大得多的、保证核的稳定性的核力。
既然有新型的力,就意味着有一种新型的场。既然有新型的场,就表明有新型的量子。电磁相互作用的传递者是光子。以此类推,核相互作用的传递者就必然是冗介子(找们说过,µ介子与核的相互作用是微弱的,因此它不能是核场的盘子)。
那么,总起来说,冗介子是核场的量子。但不同f光子,这些量子具有静质噩,而这个质量在微观世界中是相当可观的,它几乎要比电子的质量大三百倍。由于这个原因,兀介子不能以光速运动。嘿,这也叫量子!与其说象量子,不如说象粒子。但它们是星子。物理学家刚刚为场与实物的相互关系描绘的协调的图画忽然义被搅乱了。
看来兀介子是二象性的极端情况:它们是实物,这在于

•233•
  
它们具有非零伯静质鼠;它们又代表着一个场,这在于它们的自旋是零。
让我们沉思片刻吧!问题的关键是:自从量子力学兴盛以来,物理学家在实物粒子和场昼子之间又划了另一条新的、朗确的分界线。这个分界线就是二者在自旋上的差别。人们发现:真正的实物粒子只能有一个相当于1/2普朗克常数*方(或h/4心的自旋,而场量子必须有等于零或一个普朗克常数月(或h/2分的自旋。
自旋之所以体现出粒子和场本质上的深刻区别是有充分理由的。人们发现:自旋对微观物质的品性起着本质的影响。
回忆一下泡利原理:一个集团中不能有两个电子存在于绝对相同的能量状态中,不仅对电子来说是如此,对质子、中子以及所有的具有1/2自旋的粒子都是如此。
可是对于自旋为0或1的粒子,这个原理便失效。例如对光子世界(实际上是整个宇宙!)来说,可以有任何数目的光子处于同样的能量状态下,即具有同样的频率和自旋方向(光子的自旋等于1)。
附带提一下,根据自旋的这种划分可以推论出:曾使物理学家一度感到棘手的µ介子不可能是核场量子。它具有1/2自旋。但所有的冗介子都具有零自旋,因而能充当场篮子。可是它们的非零值静质量!......

拿普朗克常数可用h或忙表示戊它们的关系是h=2冗,0-一译者注

•234•
  
§98两面派的冗介子
对物理学家来说,冗介子的非零值静质垦简直是件奇闻。
让我们把它弄个水落石出吧。
可能中子就是一个质子和一个冗介子紧密结合而成的吧。否。简单的贺术会使我们相信这一否定的答案。中子和质子的静质雇分别近似地等于1839和1836电子质拯,而带电冗介子的静质量为273。这就是说,一个中子发射出一个元介子后应当减少273电子质匮,可是实际上它只减少3。
当一个自由中子蜕变时,这个矛盾并不出现。中子失去一个电子,这样它就减少一个电子质量。此外,它还赋与电子和中微子双倍电子原能,其结果,蜕变后的质量便与质子相等。可是当中子发射出一个冗介子时,它的质虽损失约为自由中子衰变的质量损失的100倍。可是由于某种原因,这样大的质屉损失又不会被察觉。中子生下一个~介子后,_并未见消瘦下来,体重也未减轻。这究竟是怎么一回事呢?
看看下面这森电影吧。中子割舍一个负介子,并将它向质子射去。质子接过介子并立刻转化成中子。射出介子的中子变成了一个轻最级的质子,而捉到介子的那个质子却成了超重中子。说时氐那时快,超重中子又射出了这个介子,这样它就又变成了一个正常的质子;而轻量级质子接过这个介子并变成一个正常的中子。这场球赛包括不对等的两局,第一局是全部巳知物理学定律所绝对禁止的,而第二局倒是可

•235•
  
以允许的。
禁令出于下述理由:任何粒子的质量不能小于它的静质量,可是这里却出现了不够赋量的质子。这也可以用另外的语言来表述:中子这个前锋不能踢一个比3个电子质晕还要重的球。为了发射一个介子,它必须从核内某处找到一个相当r270个电子的质屎-—-一个真够大的质晟。这并非对能昼守恒定律的直接违反。从整个球赛来说,它并没有开守恒定律的玩笑,但第一局却似乎难避此嫌。
当物理学家发现这种情况时,他们倾向于这样一种想法真是个很坏的想法,那就是:在微观世界中,能量守恒定律只能作为平均值生效,而对单个事件则会失效。当然,往后的科学发展证明:这个定律继续有效。对经典物理学来说,这场球赛仍然是个不解之谜。
当然,只要我们记住,我们是在和粒子的量子特性打交道,这团迷雾就会被驱散。物理学家给那些被经典教条所禁止的过程取了个名称:等效过程或虚过程。
一个粒子系统或一个单个的粒子,能以不同方式转变成另一粒子系统或另一单个粒子。我们可能不知道这些转变途径C实际的情况也往往是这样),但我们可以对这些转变过程进行随心所欲的描述,只要我们使用的那些中间过程符合当今的计算。在目前,虚过程是一些方便的、形象的概念。
另一间题是:介子能否在质子与中子间交换,就像电子在氢分子内交换一样?如果确是这样,事情就简单得多了,因为电子不会发生任何转化,而原子的结合却因电子交换得到

•236•
  
实现。也可能有一个负冗介子循环在两个质子的外围。
不,这是行不通的。理由如下:最近科学家成功地将µ介子引进原子云中,以代替电子,结果这项交换工作µ介子也干得同样出色。首先,正象电子所做的那样,µ介子将两个氢原子结合成一个氢分子,即所谓氢的介子分子。由于µ介子的质匾大约为电子的200倍,因此介子的几率云也相应地更加贴近核,也就是说,两个原子靠µ介子结合成的分子在体积上要小200倍。
当然,存在于氢介子分子中的不是一个冗介子,而是一个µ介子,而且起作用的力不是核力,而是电力。后者要远远弱于核力。
冗介子不能象电子那样逗留在原子中,因为它与核起着既强烈而又非常特殊的相互作用。这里,特殊的意义在于:元介子能将一个中子转化成一个质子,也能将一个质子转化成一个中子。

§99揭露介子交换之谜

我们姑且认为冗介于在核粒子之间循环。当然,这种循环不是在粒子外围而是在粒子内部进行的,方法是:一个粒子发射一个介子,而另一粒子将它俘获。但这种发射和俘获过程是与上面所说的守恒定律相抵触的。可是这些过程又存在着。这些过程在等效述进行着。
一般说来,等效过程或虚过程并不是新东西。回忆一下

•237•
  
微观粒子是怎样渗过位垒的。用经典理论的观点来看,一个粒子在位垒外的出现表明它跳过了位垒。但薛定谔方程指出:阱内粒子在未获得任何能批下,也具有脱离位垒的几率e这似乎是和能显守恒定律相冲突的。实际上,如果一个粒子自发地越过位垒,它必须从自身吸取能量,随后这个能量又必须消失。

前面我们曾借助微观拉子的波特性来解释这个谜。让我们伲耍地温习一下。
根据悔森堡关系,每个粒子都有能敬(包括动能和位陷)的测不准昼。如果企图俘获一个正在穿越位垒的粒子,也就足说在垒璧内友现这个粒子,那就会使其能低成为不确定的。其结果,这个能朵将变化到这样程度,以致允许粒子以经典的合法方式眺过位垒。
严格地说,能量守恒定律在经典意义下遭到了破坏。但阰f力孚证实;这个定律根本就没有被违犯。
我们也能用同样的方式来解释核粒子是怎样发射并吸收冗介子的。关速是:海森堡关系(能量与时间的关系式)也可以应用于粒了的原能。这样,中子发射一个负冗介子而消瘦,或质子反射一个正冗介子而丧失体重,以及粒子因吸收介子而肥胖起来,-—-这些都可以被认为是粒子原能的某种测不准,而后者又是与它们的质鼠的某种测不准相联系杆的。
很清楚,这些粒子原能的测不准擅在数值上不能小于冗介了的原偕~/;;=m飞心这里m兀为1l介子的静质槛。让我们恨据这个关系式求出能屈的这个测不准将持续多久。换句

•238•
  
话说,质子和中子之间进行球赛的一个周期需多长时间。
根据海森堡关系:
.t:i.EX.t:i.t"'h
我们罚出

k
b.t"',.旷
m兀c2

把冗介子的质暨m六,普朗克常数h以及光速e代入上式,我们得出At""10一23秒。
时间真够短促啊!在这样一段时间里,冗介子又能走多远呢?显然有个极限。元介子能够具有仅仅小于光速的速度,因此,冗介子被核粒子抛出后所能走的最大距离为R=CXAt.这个数值大约为10一13厘米。但这一数值在数量级上与核力的作用距离相符合!太出奇了!它证实了我们的推理。
这样,俘获一个被核粒子违法地发射出来,然后又被其他粒子违法地吸收的兀介子,是不可能的,其理由正如我们不能侦察出正在穿越位垒的电子一样。我们刚一开动头脑里的测量仪器,它立即就使参与冗介子交换的质子及中子的能显浒
~
加到如此的程度,以致这种交换在经典方式下成为可能。
这样看来,隐藏在虚过程背后的实质还是微观粒子的波特性!核力的作用距离是有限的,理由正是:核场蜀子(冗介子)具有一个非零值的静质掀。
冗介子只是在核内执行任务时才表现出稳定的品性。在自由状态下,这个粒子的行为则十分两样。一旦居于核外,亢介子就在很短的时间内衰变,这个时间的数量级为一亿分之

•239•
  
一秒。一个正兀介子转化成一个芷µ介子,一个负兀介子转化成一个负µ介子。衰变过程中,一个中微子被抛射出来。
不久以后,第三种冗介子被发现—一一个不带电的粒子。
这个中性介子的衰变要比它的带电兄弟快十亿倍。临死的时候,它生下两个丫光子,但这种光了的能匮耍远远大于电子·一正电子遣遇时产生的光子的能属。
正是元介子的这种不稳定性使得它们这祥地异于光子。
光子能够改变它的能匮,甚至能把徒址全部交出并作为粒子而消失净尽,但光子永不衰变。从来没有人观察到一个母亲光子分裂成若干带有较小能盐的女儿光子。
元介子把物理学家惨淡经营的场-实物相互关系的美丽图画弄得一塌糊涂了。冗介子的确是迄今所知的粒子和址子的最大的二象性混血儿!

§100相互作用的秘密

在科学已知的全部场中,电磁场被研究得最多。关于电
""

场和磁场,我们已经很熟悉了。电场既可以是静止着的电荷产生出来的,也可以是由运动着的电荷产生出来的。但磁场却只能由运动着的电荷产生出来。由于带电粒子的每一相互作用都与运动相联系着,并且体现于运动中,因此可以作这样一个一般的陈述:每一相互作用都包含着复合的电磁场。
为使问题简化,让我们暂且对磁场置之不理,而对电(更确切些说,静屯)场进行一次更细致的观察。从上学的日子

•240•
  
起,我们就记得同性电荷相斥,异性电荷相吸。教科书是这样来解释这个秘密的:电荷在它的周围产生一个场,这个场排斥进入其中的任何同性电荷,而吸引进入其中的任何异性电荷。
这个解释并不比说”一个人死了,因为生命力离开了他“更高朋一些。
在学校里产力”字被换成“场”字。这个新字眼被引进来了,但并没有得到解释。场的特征也被给出(场强,电力线等),但更多的便没有说。
是的,经典物理学确实引进了场的概念,可是它并不能够赋与场任何特定的、确切的意义。场被证明是那样的复杂,以致直到今天它大体说来还在物理学家的认识范围之外。
但忱子力学在这方面却获得了若干十分重要的进展。让我们看看它作出了哪些成绩。
物理学知道两种电荷-—-正的和负的。质子具有正电荷,电子具有负电荷。这些(以及它们的反粒子)是仅有的、绝对稳定的电荷携带者e现在我们将讨论电子。质子看来却复杂得多,因此留在以后讲。
这样,所有的负电荷都是属于电子的。让我们取两个电子,看看它们是怎样打架的。首先,它们必须摸清彼此的位置。
首先须要考虑的是:每一个电子都使自己周围的空间变得弯曲,正象爱因斯坦对一切物体-—-不管它们是多么大,或多么小一一所要求的那样。于是,每个电子将沿着一条环绕

•241•
  
另一电子的曲线运动着。这就好象一个球在一张纸上运动,而这张纸却被静止其上的另一个球压得弯曲了。
当然,这个弯曲是质鼠而非电荷引起的。因此,这里又出现了另外一个场一一引力场。
弟二件须荌考虑的事是:电子打乱了自己周围的真空的均匀,「L,因为如果真空被认为充满尚未诞生的电子的话,那么找门的真空之上的电子忱势必要排斥那些真空电子。如果这个屯了又遇到一个对手,后者也将以同样的方式作用于真空。
但真实电子和真空电子间的排斥是相互的排斥。对于第二个屯子来说,这个真空屯子也将一视同仁。这祥真空电子对两个贞实屯子的排斥就休现为两个真实电子间的相互排斥。
当然,如杲我们想得再深入一些,这个推理看来还是有点工病彴。我们想要说明排斥,因而引入真实电子与真空电子间的排斥而又未对此加以任何解释。
尽管如此,粒子通过真空相互作用的概念还是有用的。
我们叽-须裴似定的书悄就是:一个电子能自发地发射光子')
一个电了能发射光子。原了云中电子的眺跃能产生光于,这点我们已经知道。但在这一过程中,电子改变了它的能益状态。确实是这样。如果一个自由的、居于定态的电子在发射一个光子后,立刻将它收回来,则情况又将如何呢?这样,屯了的能闷仍然保持不变。但这一过程本身是经典力学所

•2-12•
  
禁止的。可是我们知道趾子力学是允许这一过程的,唯一的条件是:电子必须被纳入测不准关系的框框里去。
屯了友射并收回光了的频率应当仅仅依赖于光子的能从3光于的能凰愈大,电子完成这一动作也愈快。
当然,在光子居于电子之外的这段时间里,它将从容地环绕母亲的身旁游飞一番。这里所说的身旁,范围究竟多大呢?无限大。必须记住:电子能发射任何能益的光子,甚至能最最小的光子。而这些能证最小的光子,能够离开母亲游向任何可以想象的远方3当然,对于那些具有十分确定的频率的光子来说,它们的作用距离属于光子波长的数星级。对于可见光线的光子来说,这个距离的数星级是一微米的十分之

__..一
。

光f自然不行仅仅作为观察者而活动着。如果一个电子发射的光子与另外一个电子相遇,它们将彼此猛烈地碰撞。其结果,可能其中某些光子将永远不会返回母亲的怀抱。它们可能,巴如说,枝母亲的对手吞食进去。
这卫看来好象能戳守恒定律遭到了确实的而非等效的破坏。事实并非如此。电子能噩的变化批恰恰等于那些回不来的光子所带走的能乱。这样,两个电子将彼此远离:它们彼此离开得愈远,它们相互作用的能量也就愈小。
尽管如此,这个解释也不能令人十分满意。这个场不知何故被拴在它的创迅者身旁了,但我们知逍光子是能够完全独立行动的物质c
为了更好地Ll圆J且心人们又引入了另外一个虚过程,而

•243•
  
这个过程也是实际上遇到过的。一个被电子发射出来的能撒足够大的光子能够在非常短暂的许可寿命中,转化成电子-正电子偶。
这样,在一瞬间一个电子将被两个电子和一个正电子所代替。在另一瞬间,这个电子将还原成自身。但两个电子中究竟是那一个将与正电子融合而湮没?那是确定不了的,因为两个电子完全一样。
多么有趣啊!遗憾的是我们不能观察到单个电子是怎样开放出一束粒子之花的。这一切都是在那样短暂的一瞬间渡过的。
但无论如何也得让我们检验一下。根据海森堡关系而作的简单计算表明:这个短暂的时间待续大约10一21秒。在这段时间里,光子能够生出第二代电子和一个正电子,——它们构成对偶;而对偶的出生地距离笫一个电子约10一11厘米。
这个数批恰好代表着电子在空间的最小弥漫。10一11厘米正是接近光速运动的电子的德布罗意波的波长。
这幅情景是多么奇妙啊!它说明电子(当然也是其他所有粒子)波特性的实质是相互作用,是它的场。电子的弥漫是因为它跳进了真空又在近旁从真空中跳了出来,而这样的跳进跳出在一秒钟内要进行数不清的次数。
物理学家把具有这种奇妙品性的电子称为颤抖着的电子。这个形象是非常逼真的。在这个过程中,电子能够振动,而它的位罩可以是许给它的区域内的任何一处。这个区域将白电子抛射的光子后者能产生电子-正电子偶--的能

•244•
  
晕,以及由此而导出的光子的波长来确定。

§101虚效应王国

这样,一个电子虚发一些光子。这些光子又虚化为一些电子-正电子偶。这些偶又融合成一些光子。这些光子又被电子吸收。整个万花简般的转换在以神奇的速度进行着一一每秒几百亿亿次。
某个电子发射的光子可以被别的电子俘获。但电子都是一模一样的,因此也无法分辨出究竟是那个电子俘获了这个光子。
这个交换的结果不是虚的了,而却是十分真实的:诸电子彼此尽量远地分开。即使它们司的距离超过了真空弥漫几倍,光子也能追上它们,并将双方推开得更远。但电子间的距离愈大,高能光子追上它们就愈困难,这就意味着,在光子交换中,光子赋与电子的能量也就愈小,因而电子的电排斥也就愈微弱。这正符合库伦定律的陈述。
电子的相互作用是无往而不在的。为了简单起见,我们假设只有两个电子参与这个相互作用,实际上宇宙间的全部电子都已参加进来了。我们可以说,这两个电子所形成的无边无际的电磁场,达到了无限宇宙的每一角落。
电子和正电子的相互作用,电子和质子以及一般说来所有异性电荷的相互作用,都具有相同的等效(虚)性质。只是在这种悄况下,光子交换的结果不是粒子双方向后退缩,而是

•245•
  
相互靠拢。
自然界是二重性的。它既要经管对立的统一,又要经管统一的对立,或统一的对比飞具有相反电荷、相同质量的两个粒子,即互为镜中映象的两个粒子,一且从镜中跳出相互遭遇,就会将双方的电荷勾销而转化为起笞相互作用的炀址子。

§102虚的变成真的
物理学家在给事物命名的时候,并不总能想出最妥当的词。虚,意味礼实质上,而非事实上,足真实的,或不十分真实的。可是虚真空能够突然地变得非常页实。
匝想一下原子中电子的过渡**产生出光i礼我们说过,电子从一个状态过渡到另一个状态,只有当电子在这些状态下的几率云,在空间的某处彼此煎叠时,才是可能的。
氢原子也有这样的两个状态,它们的电子几率云完全融合在一起。这两个状态都属于第二壳层(此壳层从锥原子开始装其)。但氢原子中还存在着属于第一壳层的另一个状态,也就是最低的、最稳定的能态,氢的电子在正常条件下就居于

这一状态中。

*参石«毛泽东选渠»第五卷笫319页:“世界上一切事物郎是对立统一。”320页:"`一阴一阳之渭逍'。不能只有阴没有阳,或者只有阳没有阴。”-----详者注
**或跃迁。——详者社

•246•
  
对应第一壳层的状态和第二壳层的状态-也就是原子大厦的一楼和二楼一一是彼此从来不重叠的球形云。还有我们记忆犹新的第三状态:它是联接一楼和二楼的一个不方便的纵贯楼层。
第三状态只是在锥原子中才以纵贯楼层的形式出现,而在氢原子中它必须与二楼住宅相吻合。因此,在锥原子中所能观察到的电子过渡在氢原子中则观察不到。在一般情况下,原子的房客并不直接在两层楼之间跳上眺下,它们宁愿先进入纵贯楼层再说。
实际上也正是这样。从来没有人在氢原子中观察到两层楼间的直接过渡。如果一名房客由于某种原因被从一楼抛到二楼去了,它将会十分孤独地呆在那儿、直到某个不合法的机会将它带回一楼(这样一种过渡几率小到可以完全被忽略)。
在十五年前左右,物理学家发现电子成功地迥避了这一清规戒律,轻而易举地从二楼溜回一楼。真有点象是乘电梯下来的。
这一违法事件很快就得到解释。这里需要的只是丰富的想象力,物理学家确实是富于想象力的。回忆一下真实电子排斥那些未出生的真空电子的虚过程。在那个描述里,电子仿佛在和自己的影子博斗着。
事实上也正是这样。电子和真空的相互作用,即电子的“妀抖,,'赋与电子自身非常真实的——虽然是很小的——一点衔外能星。可姑这点微不足道的能氐(比原子内的电子能

•247•
  
量小得多)也足够使氢原子中的两个交织着的状态分开,从而使电子能够从这一个状态过渡到另一个状态,即从二楼过渡到目前已经变成现实的纵贯楼层,然后经过这个楼层过渡到
一层。
当然,实际可能观察到的只是从二楼向纵贯楼层的过渡。
这已经足够使人满意了,因为其余的将自动进行。
但真空给氢电子增添了多少能量呢?如果我们使用普朗克关系并将其转化为频率,我们将会看到,增加的能量子不属于T射线,甚至也不属于可见光线,却属于高频无线电波波段。
这就是为什么这一重要现象不能被常规的光谱方法发现的原因。第二屯催界大战后高频无线电振荡器制造成功,因此氢原子可以用高频电磁波来照射。其结果,氢原子立即对符合于真空给它增添的能址的频率作出反应。一个很深的衰减出现在氢无线电光谱中的这一频率的位置上:氢原子积极地吸收着这一频率的卧子。
不久以后,第二个真空效应被发现。我们已经讲过了两种电子磁体。一种是电子绕原子核运动而形成的,另一种则是电子的自旋运动造成的。在磁场中,这两种磁体结合成一个具有确定数值的统一磁休。
物理学家以极高的准确度测量了这一基本磁体的力。结果发现,这个力比两个磁体的合力略大一小点。间题就在于这一小点。物理学家最后不得不承认:统一磁体磁力的这一小点增应是由于电子与真空的相互作用引起的。

•248•
  
这个解释与我们曾经作过的解释相似。原子中运动着的电子浩途排斥着真空屯子分仿佛一只静止的船只能排水,而一个航行着的船却迫哫水跟着运动。将运动从电子转移给真空,结果就在真空中产生出一个真空电子流。这个虚电流的磁效应,便增添在真实电子运动引起的磁体上。
浸透格各种虚效应的扇子力学不仅能解释这些重要现象,而且还能将它们计筛出来,同时计算结果还十分淙亮地与实验相吻合!
这样看来,物理学家就是要有想象力。虚(等效)过程毕竟应该岚得人们的尊重。

§103寻找新的粒子

物理学家一旦承认了微观粒子世界不寻常的性质一一它们之间的相互关系以及它们和场之间的相互关系,一个猎取新粒子的活动便真地开始了。每一新粒子都是微观世界的一个新的方面,对其特性的每一新的发现都是知识道路上的一个进步。
一批批带着复杂的装置、设备和仪器的探险队在紧张地工作着。在很长的一段时间里,宇宙射线一一那些从宇宙深处向地球飞来的粒子流一一是新粒子的唯一供应者。新仪器发朗出来了,老仪器得到了改进。一批批探险队开往高山之颠,深入那更加贴近苍穹的洁净的气层中,另外一些探险队驶入海洋,还有一些将火箭送进更高的太空。结果,捷报象雪片

•249•
  
似池飞来,打年发现的新粒子达数十种之多。
对此首先作出反应的是理论家。他们实在感到困惑不安,因为,他们说,根本不可能有那许多不同的粒子。实吤人员用一种极为苛刻的目光检验自己的成果。这样,新粒f-¬个接看一个地消失了。它们的下台比早先的登场还要快。
但胜利果实是显著的。首先是冗介子。五十年代初,质晕大于质子和中子的粒子被发现;他们叫作超子。宇宙射线给物理学家赠送了一项非常贵重的礼品:一群K介子(很快我们就会看到为什么它们是这般贵重)。
当一系列巨型加速器开动以后,质子袚加速到接近光速。
俩个新粒子被发现了,这就是反质了和反中子。这样,狄拉克理论的预期便得到了证实。
今大,{;贞观粒子表册贞叫人惊叹不已J这里,大约有三十种不同类驴的粒子,二十五年前还只打四种。
让我们看看这个表册。第一件引人注目的是粒子的质阰范围很广:从一个电子质点到S超子的2500个电子质区。粒子的质酰分布是颇不均匀的。两个或三个质遠近似的粒子形成小组。
可是电他和自旋却及有呈现任何E花八门的迹象。粒子的屯荷只能具有三个数值:+1,0和一1。这里-1就是电子的电荷。自旋也具有三个数值:1,1/2和0普朗克单位(h/2式)最后,表中大部分粒子是不和定的:平均起来,它们的片命从,a介f的百万分之几秒,到芷介子的亿亿分之几秒3这两种办命屈极端情况。介乎二者之间的粒子也是不稳

•25n,
  
粒子表

粒子			I质愤		,自旋		
粒子
	,/'.,,	弥符~(贞屯丘D子	电荷(普朗	寿命(秒)	衰交力式
种失	,/'.,,
	弥符~(贞屯丘D子
	--克-肘·一单2..一元位.)..	寿命(秒)
	衰交力式

	兑于	了	。	。	I	稳定	
	电子	e	1	-]	1/2	同上	
	正电子	e+	1	+l	1/2	同上	
轻子	中负子1和2	I'	。	。	1/2	同上	
	反中微子l	J)		。	。	1/2	同上	
	和2							
	µ一介了	,.,,-	206.i	-1	1/2	2.2x10-•	µ一->~-+V+V
	µ十介子	忙	Z06.7	+l	1/2	2.2>(10一6	沪一千e++II十D
					..'			
	冗一介子	汀-	273.2	-1	。	2.6)(10刁	元--µ-+ii
	冗十介子	兀+	273.2	+I	。	2.6)(10-s	叶一忙+v
	亢介于	亢	一、	264.2	。	。	2.2x10-1'	万0--.2r
	K一介予	K-	%6.5	一l	u	I.2)(10-•	K一一玩-+叶~;
介子									或2矿+tr
	!(+介了	入十	9(i6.5	+I	。	1.2x10-'	x+-2叶+冗一;
									或2:rc丿+叶
	k介`.	K'已	974,2	。	0r~	1.0X10-10	K"-穴++汒'
	k介`.
			974,2
	。
	0r~
	1.0X10-10
	1或2矿'
	反K'介i	k	47·1.2	。	0K~	6.1x10-•	K~-3矿
									
	质子	p	1,836.12	+l	1/2	稳定	
斗	反质子	j5	1,8,6.12	一1	1/2	同上	
泌	中子	"	1,838.5	。	l/2	l.0)<103	“一p+,.-+ii
	反中子	"	1,838.5	。	1/2	l.OXIO'	ii一-ft+e++V
	,10	,1''	2,182.8	。	1/2	2.5)<10一10	/1o_..p十,了或n+n:o
	反,1<	j·	2,182.8.	。	1/2	2.5x10-'•	AD->-p+丑.或n+7fo
	I;+	z,;十	2,327.i	+I	1/2	8.1X10一ll	工+勺五噙叶汒
重了	反l;+	E+	2,327.i	-1	1/2	8.1)<10一1J	豆仁汀+亡或P+沪
	J::0	~,--,	2,331.8	。	1/2	<ID一11	.E0-il0+丫
归	反立	t		2,331.8	。	1/2	<10一II	2。一A"+r
朗	工一一	2:.-	2,3~0.6	一l	1/2	l.6Xl0勹0	1:--11+::c-
	反江	E-	2,340.6	+l	1/2	1.6)(10一"	i;--ii十冗十
	x·o	二一"'	2,565	。	1/2	1.5)(10一IO	已勺一,lo+::ro
	反X仁	巨二	2,565	。	1/2	1.5)(10一to	忌。一,10+沪
	x-	己-	2,580.2	-1	1/2	i.2x10-'•	己--..,lo+兀-
	反x-	仁;一	2,580.2	+1	1/2	1.2)(10-10	豆一一J•十元十
  
定的,它们的寿命从一亿分之几秒到一百亿分之几秒。
不要错误地将一个粒子的寿命和它在我们这个世界里的生存时间混淆起来。为了说明间题,让我们举正电子为例。正电子不衰变成其它位子,在这个意义上来说,它是稳定的。但它在我们这个世界里活得并不长久,它一旦与电子相遇,便照例消失。另一方面,自由状态中的冗介子是不稳定的,可是在核内它却从不衰变。
再看看表的最后一行。在这些不稳定的粒子的衰变残骸中,最常见的是哪些粒子呢?各种介子和中于衰变后将出现电子和中微子。在超子衰变残骸中,我们总能找到核子和冗介子C

§104清点胜利果实

上面我们对微观世界的成员作了一个普查,并从中得出了一些初步的结论。现在的问题是搞清楚微观世界中粒子的生活条件究竟如何。
为什么粒子的质撬是那样地参差不齐?质最的极限又是什么?重型E一超子的质匾到头了吗?为什么具有近似质量的粒子总是两个、三个或四个结成一群?为什么粒子的电荷只有三个数值,而自旋只有两个(如果不把光子算在内的话)?为什么大多数粒子是不稳定的?同时为什么又有稳定的粒子?为什么粒子的衰变方式只有一种或两种,而可供选择的、可能的衰变方式却十分繁多?

•252•
  

.,
 
在穷根究底之前,必须说明:量子力学对这些问题中的大多数并未给以回答。有些问题虽然给了回答,答案大抵也只是说朋如何,未说明为何。而后者也是相当重耍的。
可以清楚地看出这个表是按质量来分组的。同一组里粒子的质戳彼此非常接近。与此相较,两个邻组在质最上却存在很大差异。对这样分组的解释是很有趣的:一组中的各种粒子,实际上只是以不同的乔装打扮起来的同一种粒子。
让我们以冗介子为例来说明这件事。元一介子和元十介子的质噩是相等的,但不同于不带电的芷介子。带电冗介子之所以具有较高的质量,可能是由于它们具有电荷。
我们说过:场是与粒子的一部分质量相关连的。由于兀介子是核场的扯子,而核场要远比电磁场强大,因此有理由假设:冗介子质址的主体起因于核场,电磁场(与电荷相联系着的)给它增添的质益只占一小部分。由于这个原因,带电冗介子要较中性冗介子略重,后者的质量当然完全是由核场引起的。
看来这也能解释为什么轻粒子不能形成三粒子组。核场与电磁场的区别在于:核场量子具有非零值静质量,而电磁场量子是具有零值静质凰的光子。电子、正电子以及两种µ介子都有一个明显的非核电磁本原。这就说明为什么它们没有中性粒子。对它们来说只存在两种可能:要么是正粒子,要么是负粒子。它们构成双粒子组。
但上面所说的对K介子不适用。中性K介子的质趾要大于带电的K介子9看来这里的电磁场好象是从带电K介子的

•253•
  
咳场中分割出来的。
这就是为什么物理学家倾向于把这种说明当成一种纯粹的似想。具有核本原的、不形成三粒子组的超子似乎证实了这种说明是假想。恰当的解释仍待入们去作。

§105反粒子开始活动了

直到1955年,核子组也还只包括质子和中子。这个组也贞希奇:它是由一个带电粒子和一个中性粒子组成的二粒子距这个秘密看来终于得到解决:带有负电的反质子被发现。这样核子也形成了一个正常的三粒子组,正象丁介子组
一样。
可是这野又出现一件头疼的扛:中性的中子比质子及反质子更重一些,而不是更轻一些。看来电磁场又是从核场中分割出来的。当然,最重要的是:质子和中子本来就是以两种不同形式出现的同一种粒子。顺便说一下,物理学家早就猜到了这一点,根据是:这两种粒子能同样顺利地在核内相互转变这件牢已经弄清楚。
反质子被发现后一年,反中子也被发现。这就是核子组的第四个粒于。反中子不愿被纳入这个小组的相框之中。但这里还有另一条出路——核子组可以被认为是由这样的两对构成的:质子和中子构成一对,它们的反粒子构成另外一对。可是质子和中子又是两个不同的粒子。核子组包括四个粒子的秘密真是一个硬欠果,直到今大人们也未能把它敲碎。

•254•
  
乌核子组类似的是四个K介子。这点以后还要专门来谈y沁后,我们看到,超子是成对地出现的。
粒户的分组有没有规律可寻呢?很可能有,但我们尚不知道。对i改观世界的普查已经作过;哪些粒子该归属哪一组,都已女甘妥当。尽管如此,最后的结论还是作不出来。
现在让我们尽晁把粒子和反粒子之间的差别搞清楚。我们知道,最早形式的狄拉克理论声明:区别在于电荷的符号。对于电予和正电子,质子和反质子,两个,ll介子以及一般说来对所有带电粒子,这确实是正确的。
对门子和反中子,狄拉克理论又将说些什么呢?这两个粒子都不具电荷,它们的质晟也相等,正如所有的粒子-反粒子偶一样。看来区别在f磁矩上。
这样说来磁矩是否也可能具有两种相反的性质?当然,我们知道在原子里电子总是成双地居于同一能态中,这意味着它们具有相反的自旋。尽管这样,它们还都是电子,二者之中并没有一个变成了正电子U同理,核中子也能成对地占据着同一伲级一一这点,我们在讲核的壳层模型时曾经说过__可是也不曾有反中子诞生出来。
原子中每对电子具有相反方向的自旋,但这只意味石:这两个电子在朝相反的方向运动着。当然,如果电子被描绘成几率云的话,则它们不同的运动方向认很难想象得出。对于一个自由原子来说,两个电子在能屈上没有差别。但电子的自旋方向肯定是与电子的运动方向相联系着的。假设电了在向右运动,我们可以说它的门旋,暂如,向上成某一角度。

  
为什么冗介子蜕变为两个粒子,而µ介子衰变为三个粒子?回答是简单的:完全由于自旋。儿个女儿粒子的自旋之和必须等于母亲粒子的自旋。
!l介子具有1/2自旋,电子也是如此。由于电子不能带走µ介了的全部质鼠,因此需要一个中微子以运动形式即能量形式把剩余质匾拾起来。但中微子的自旋也是1/2,这样新生粒子的自旋之和将大于母亲粒子的自旋。中微子的这一多余的自旋必须要由另一粒子的相反自旋抵消掉。这个粒子就是反中微子。结果是:三个粒子。
在冗介子的衰变过程中,一个中微子(或反中微子)即足以抵消新生µ介子的相反自旋。这两个自旋相抵消,结果自旋为零,这也正是原来的亢介子的自旋数值。
对超子来说,衰变的最终稳定产物往往是质子。此外,超子还发射兀介子e这样就有两个粒子世界,和两种限制性转化型式:在轻粒子世界中,最后转化成电子;在重粒子世界中,最后转化成质子。两个世界各有自己必然的衰变副产品:轻粒子世界中出现中微子,重粒子世界中出现元介子。
现在要问:有没有这样一个定律,它将指出在许多衰变方式中哪一种C至多两种)方式将被选用?
我们已经知道这种选择的某些特点。为了和经典物理学一致,让我们把这些特点叫作守恒定律。观察表明:粒子衰变前后的总电荷和总自旋是守恒的。但这些守恒定律也还给衰变方式的选择留下了一点自由遥旋的余地o
应该还有另外一些衰变定律,它们将衰变途径限制到这

•257•
  
样狭窄的程度,以致沿此途径不稳定的粒子只能转变成物质的最稳定的建筑砖石——质子和电子。

§107物理学家将相互作用分类

让我们从一个比喻开始。这里有许多不同方法摧毁一座山。可以是某种爆炸,如火山爆发。但也可以是一个较雨的、较慢的过程,如地宸。最后,也可以是一个最慢的过程,如风化——在这一过程中水、风、寒、暑都对这座山起着剥蚀作用。爆炸只须几秒钟即可完成这项破坏工作,地震须几小时,可是风面的侵蚀却须要几千年之久。
对佽观拉予破坏过程的研究表明:这种过程可以按不同的强度和速度分为三种类型。
第一种、也就是最强的一种破坏来自核粒子的撞击,即核内的相互作用。物理学家把这些扑件称为强相互作用。强相互作用的特征是能屈大——属于或大于冗介子的原能的数量级,以及与此相应的(根据测不准关系)极短寿命。我们知道,这里的时间因子屈于10-23秒这个数匮。
其次就是电磁相互作用,它的强度和进行时间仅次于前者。在电磁相互作用过程中,电子-正电子相遭遇并产生两个y光子。上面所描述的中性冗介子衰变为两个r光子也属于这一类。这一过程进行时间的数最级为10一17秒。
第二种砐坏过程最涡,延续的时间最长。物理学家把它叫作为相互作用2微观粒子表中的衰变绝大多数屎f这一过

•258•
  
程C~]相互作用能说匪压、元、K介子、中子以及超子的衰变。从该表m可以召出:影响着各组粒子的这种普遍的、具有破坏性的相互作用延续时间在10-10秒以上。
彴研究粒子组的过程中,有件事引起了人们的兴趣。各种K介臼对于一组,各种超子属于一组,而这两类粒子的分组力法不同了其他粒子。
这俩种粒子不愿迁訧于其他粒子的分类方法。”奇怪,,'物理学家说,懊恼地把这些调皮的家伙叫作“奇异粒子”。他们甚至引进了一个特殊的最来定量地描述它们与正常特性的歧离程度。这个量就叫作“奇异性”。
人们发现,奇异粒子只能通过缓慢的弱相互作用衰变成一般粒子。在一般粒子的碰撞过程中,奇异拉子只能成对地诞生,因为只有在成对诞生的条件下,它们奇异性的总合才等于零,正象碰撞前原来的一般粒子那样。
换旬话说,在强相互作用和电磁相互作用中,奇异性不变。这个规律就叫作奇异守恒定律。但在弱相互作用中这个定律不生效。
定律太多了吧?那样一个统一的、普迫的定律又在哪儿1y臼我们又怎样能够解释这许许多多的定律呢?
可惜的是,我们一直在谈论着的种种规则,至今还没有得到一个有说服力的解释。物理学家通过这种或那种途径把这些规则结合起来,但隐藏在事物深处的本肵却仍然把握不住。当然,根据这些守恒定律的祯算,我们伲够彴出粒子的实际衰变方式()所有这些守恒定律综合起来,允许粒子实际上只有一

•259..
  
种,最多两种衰变方式。
对K介子衰变的研究使微观粒子物理学家在发现真空效应之后,又作出了另一项袁动整个科学界的重大发现:“宇称不守恒气

§108K介子的秘密

大约在十年前,人们在宇宙射线里发现了K介子。宇宙射线在照相底片上留下了一堆光怪陆离的径迹。在这些径迹中,物理学家以机警的目光分辨出了某些质量大约为电子一于倍的新粒子的径迹。
K介子看来有三种类型:正的、负的和中性的。自旋也被计算出来:等于零。K介子家族最初看来与冗介子这个较轻的家族(除质旦外)并没有很大区别:自旋都是零,都是三粒子组,只是中性的K介子比它的兄弟稍重而已。
物理学家审查了K介子在照相底片上留下的径迹。带电粒子留下的一般径迹经常会中断,接踵而来的是较细的径迹。这意味着K介子已衰变成较轻的粒子。对派生径迹的研究表阴:它们是属于亢介子的。
中性K介子的衰变就更困难了。物理学家惊异地发现:K介子飞行路线的终点时或出现两条,时或出现三条径迹。这些径迹,和上面所说的一样,是属于兀介子的。
这样,中性K介子有时衰变为三个,有时又衰变为两个兀介子,而其地所有的粒子总以一种方式蜕变为固定的女儿

•260•
  





  
但它却触犯了一个我们尚未说过的禁令。这个禁令是量子力学建立的,它的名字就叫宇称守恒定律。

§109左和右有什么两样吗

让我们回顾一下激发状态下的原子是怎样发射光子的。
电子先居于一种状态,然后跃至另一种能量较低的状态。当时我们感到兴趣的只是电千的能垦,以及电子前后两种状态的几率云是否正叠。
这种重香似乎和宇称(或奇偶性)有看十分本质的联系。
如果重数一下原子内的电子楼层,我们将发现:电子这个房客只能从偶数楼层迁移到奇数楼层,或者从奇数楼层迁移到偶数楼层。从第十层通过一次跳跃而迁至其他偶数楼层,例如第八层,是没有可能的。
这个定则早在1924年就已由实验所确立,以后又获得了量子力学的解释。为此物理学家引入了波函数的宇称概念。从这点出发,宇称概念又扩展到波函数所描述的状态本身。
我们知道波函数就是薛定谔方程的解。至于宇称这个概念,这里还须进一步作点解释。
多少人看到自己的相片时都要说:"哦,这张相片一点也不象我气他们认为这都怪照柜师。但实际上往往不怪照相师。
你不妨照照镜子吧!你看到的并不是一个准确的挛写。
如果你的鼻子有点向右边歪,那么镜里的员子就会偏向左侧。镜子里,右与左已经调换位置。
 
首先就是光子。以后其他粒子也都获得了符号。例如电子就被证明是奇数粒子。
我们曾经说过,电子的自旋肯定是与电子的运动方向密切相关的。如果电子向右运动时的自旋,譬如说,是向上的,那么屯子向左运动时的自旋便应当是向下的了。现在让我们想象一下电子在镜中的映象。当电子向右运动时(在镜里便是向左运动),它的自旋在镜里的映象仍然是向上的,因为镜子只能使映象左右调换,不能使映象上下倒置。由此可见,镜中电子的自旋在正常电子中是不存在的。这就说明,电子肯定是个奇数粒子。如果电子是偶数粒子,它的镜象必然与实物一致。
冗介子是奇数粒子。
物理学家从电子发射光子开始类推,将奇偶性的划分扩展到不稳定的粒子,确证:原来粒子的奇偶符号必须与衰变后各个粒子的奇偶符号的乘积相同。截至目前,没有任何粒子违反过这条规定。这条规定就叫作守称守恒定律。
可是现在偏偏出了个中性K介子!中性K介子可以衰变为两个亢介子,从这点看来它是个偶数粒子(负乘负得正)。但中性K介子又可以衰变为三个冗介子,从这点看来,它又是个奇数粒子(负乘负乘负仍然得负)。实际上它倒底是个什么粒子?是偶数的?还是奇数的?
显然我们这里涉及的是同一个粒子,而非两个粒子,因为T介子和O介子在质量上是完全等同的。这样看来K介子是一个具有双重奇偶性的粒子!这样的假设似乎也很难接

•265•
  
受。这个K介子确实使量子力学陷入了困境。

§110道路终于被发现

怎样办呢?如果说中性K介子在衰变过程中不遵守宇称守恒定律,那就意味着自然界这面镜子有毛病,因而左异于右。这也就是说空间本身是不对称的!这真是个可怕的结论。
物理学在其存在的漫长岁月里,巳经习惯于认为空间在各个方向上都是等同的。在相同条件下,向左的运动无异于向右的运动Q真的,物理学的全部定律都暗示着所有方向的等价性。这就叫空间各向同性。
放弃这个观点就意味着拒绝承认全部最最基本的物理定律。这一点真是连想都不敢想。
年轻的物理学家李政道、杨振宁从这一团境中找到了一条引人注目的出路。他们大胆宣称:在K介子的衰变过程中,一般说来在一切溺相互作用(这种蜀相互作用引起介子的衰变以及核中子的8衰变)中,宇称守恒定律能够失效!
李、杨还作了足以确证这一惊人事实的实验飞这实验是值得在这里描述一番的。
计算表明,如果宇称守恒定律失效,那么在核P衰变过程中,电子大多数要沿养与核自旋相反的方向飞去。在正常条

*这个实验实际上是又健雄作的。一一评者注
•	266•
  
件下,从整体来石,核自旋是杂乱无章的,因而电子将同样地沿看所有不同的方向飞去。
因此第一件书就是要使原子核整齐地排列起来,从而使它们所有的自旋都具有相同的方向。其次便是在实验过程中,要始终保待原子核的这种整齐排列。为此,一块p放射性物质被放置在一个能够使核自旋磁体整齐地排列起来的强大磁场之中。然后使温度急剧下降(至绝对温度百分之五度),以消除由于核的热运动而引起的畸变效应。
在上述装置的周围安置了一系列的电子计数器,并使这些计数器与核自旋方向及核自旋的镜象方向成一微小角度。计数器接通后,很快就发现:沿着前向飞去的电子数目要少于沿着镜向飞去的电子数目。李、杨的预言得到了证实。
这是否意味着空间是自然界的一面哈哈镜呢?物理学的基本定律是否从此乱了套呢?这里,李、杨以及独立进行工作的苏联科学家朗道,作了一项重要声明:空间与上述现象无
	关,毛病出自粒子本身。	一
你还记得,当电子反映在镜里的时候,所得到的映象是一个具有相反自旋的、实际并不存在的电子。现在已经证实,这种粒子确实存在,不过我们必须“反映“(倒转)它的电荷符号才行。这祥我们得到的将是电子的真正映象——我们听熟悉的正电子。
这样看来,自然界的这面镜子还是正常的。但它是某种二重镜:当一个粒子反映在其中的时候,得到的总是它的反粒子!屯子得到的反映是正电子,中性K介子得到的反映仍

•267•
  






  
§111世界和反世界

我们已经叙述过这样一个事实:在我们的世界里正电子是一些稀客。这就意味着粒子世界并不是对称的:具有左螺旋性的粒子大大超过具有右螺旋性的粒子。
其实这也不足为怪。更仔细地瞧瞧宏观世界吧。蜗牛大多具有左螺旋性:蜗牛壳的螺纹大多是左旋的。化学里我们遇到立体异构分子,这些分子彼此互为镜中的映象。在它们的世界里,左向异构体要多于右向异构体。最后再看看人类本身。人类的心脏也位于左侧,虽然我们也会偶然遇到所谓镜中入,他们的内脏器官位置完全和正常人的相反。人们中间左报子很多,大多数人倒是使用右手的。
.因此,在更广阔的宇宙里,我们不要因为可能发现反世界而大惊小怪。在那里事事都来了个大拧个。在那里,反原子将具有这样的结构:构成原子核的是反质子和反中子,环绕着它们的是正电子。在那里,如果有任何生物的话,它们的器官将和地球上生物的器官在镜中的映象一样。
如果两个世界所处的条件完全等同,反世界所遵循的规律与我们这个世界所遵循的规律将毫无两样。不同的只是两个世界的符号完全相反而巳。正因为如此,我们永远也不会察觉反世界的存在,即使它就在我们的身旁。
我们可能发现的,最多不过是我们的世界和反世界之间的分界线。在这条分界线上,两个世界将怀着最大的敌意相

•269•
  
遗遇。在遭遇的一瞬间,所有的粒子将灰飞烟灭,转化为以光...的速度驰向四方的高能T光子,或转化为冗介子。光子和立介子将形成这两个世界的中间地带,以便警告任何粒子不要越过这个危险区域而进入对方的世界。可是截至目前,科学家还没有在太阳系范围内,或其他更广阔的星系范围内,观察到具有上述性质的分界线。

§112粒子内部在发生着什么

我们从一个从来没有被回答过的问题开始:微观粒子的体积到底有多大?这些粒子究竟有没有确切的体积?
这个问题真间得怪!当然每件东西都有某种休积。当我们在微观世界里有了一些阅历之后,便知道市情并不尽然如此。
在漫长的岁月里,物理学家不能正确处理这个问题。部分的原因是:粒子一且被赋与体积,呈子力学的数学运算总要归于失败。另一方面,正如我们在§41一44中看到的那样,对于它们的体积根本就无法进行实际晕度。造成这种状况的原因是:粒子的波特性将它的空间位置涂抹得模糊不清。
这些波特性是粒子和它的场之间的相互作用的外在表现。换句话说,一个电子的弥漫是庄于它与其他粒子(包括电子)的相互作用引起的。
我们知逍相互作用的现代描述是怎样一回才比一个电子以虚方式发射出一些光子,并与其他粒子发射的光子相互作

•270•

.
  
用。其结果不是粒子的相互排斥就是它们的相互吸引。看来好象这个电子被裹在它自己发射并吸收的虚光子云海之中。而这个云海是茫茫无际的:这里总有那样一些低能盈光子,它们在海森堡关系的许可下,能达到距离发射它们的电子任何远的地方。正是光子云使电子在空间中的位置桢瑚不清,也正是光子云不允许我们说出电子的确切休积。
当然,电子云愈接近核心收缩得愈紧凑。在足够短的距离内,虚光子所具能量足够形成电子-正电子偶。在这样的距离(属于10-II厘米数抵级)以内,我们将看到一个所谓“颤抖着的“电子。即使是这样一个电子,它也仍然被涂抹得模糊不清,只是这一涂抹在空间中占有较小的区域而巳。即使足这样,电子也没有确切的体积。
也许准确地测量一个摆脱了它的光子以及电子-正电子云的赤裸裸的电子的体积是可能的吧?否。在自然界里,无相互作用的电子根本就不存在,也不可能存在。粒子和它的相互作用是统一体——-不可分割的统一体一一的两个方面。
留给我们设想的唯一一件事就是:在这重重云雾之中有某种象物理学家取名为核心那样的东西。这个核心究竟是什么样子的?核心内部在发生着什么?我们直到今天也弄不清楚。
物理学家也在尝试去作另一件类似的工作,那就是描绘另一个基本粒子——质子一一的构造。质子(以虚方式)发射冗介子,后者的能抵自然不小于它的静能最。这就是为什么冗介子的寿命是那样短促。这意味着兀介子不能达到距离发射


§113旧观念使人裹足不前

实物和场永恒的、普遍的关系使物理学家面临着这样一项工作:解释这种关系,并提出统一实物-场的新概念。这里,具有已被确认了的观点的量子力学巳经有些保守了。
当量子力学诞生的时候,它从自己的前辈一一经典物理学一一那里继承了用于宏观世界的全部概念,并将这些概念带进微观世界里来。薛定谔方程就是沿着经典的波方程的思路建成的,唯一的区别在于:它描述的不是一般的波,而是体现微观粒子在时空中运动规律的几率波。最初的结果是令人完全满意的:微观粒子驯服地遵守着这些规律。
当然,在量子力学刚刚出世的时候,人们就发现:那些旧观念在新物理学中不会很好地发挥作用。测不准关系明确指出:关千粒子的准确的位置和速度、能量和时间的早先的观念,在微观世界中只能在一个受到很大限制的范围里得到应用。
微观粒子一旦获得了足够进行相互转化的能量后,量子力学给人带来的不很满意就让位于很不满意了:此时,用以确立粒子在时空中运动规律的批子力学方法便完全失效。
设想这样一幅情景吧:原来只有一个粒子,后来又出现了另外一个甚至儿个粒子;或者一些粒子被一些光子代替。很自然,波函数对描述粒子转化是无能为力的。根据量子力学,转化应当在空间某一定点并在某一瞬时发生的。转化的结果,

"'273卢·
  
我们获得了另一个粒子或光子,一一对此早先的波函数已经不再生效。
面收这种从而,忱子力学将笘么办呢?在粒子抟化的现场上,探子力于将新、老运动规律结合起来,在这样做的过程中,它充分地利用符能僵和动量守恒等人所熟知的定律。
可是这样做的结果,转化过程本身便被置之脑后。首先,由于转化是在时空的某一定点发生的,因而粒子在转化的一瞬间便不再处于寻常意义下的运动之中。其次是因为一种类型的粒子消失,而另一种类型的粒子出现,可是运动方程的描述对象还总是一个单一不变的粒子。
这意味着:在砃究微观世界现象时,经典方法通过习惯的时空概念而被引人抵子力学是有明显的缺点的。这样的理论不能反映微观世界的最根本的本质:粒子能相互转化,位子伲转化为场且子,反之场量子也能转化为实物粒子。目前的问题是如何决定转化的实际过程。但这须要在描述方法上来一个根本性质的变革。
匾子力学靠引进我们说过的虚过程来实现这一变革。但虚过程也失败了,而且不能为这个问题提供最终的解答。现在需要的是一个更深刻的研究方法。在这种研究方法中,经典的时空概念可能会获得一个根本性质的变革。

§114修改时空观念

这个新的研究方法将从何着手呢?有人说:放弁我们一

•274•
 
有其他思维活动的话,我们绝不可能决定物体离开我们倒底有多远,以及这物体倒底有多大。这里触觉帮助了我们。我们接触了这个物体,从而知道了它的体积(自然是相对我们自身来说的)。
如果没有物体,我们也就不会有空间概念。在夜挽我们看不见物体,因而也就失去了空间的感觉。
帮助我们建立有关周围世界的各种概念的感觉器官实际上就是仪器。它们甚至灵敏到足以记录量子事件的程度。世界的构造是如此地复杂,以致在每一瞬间须要记录几十亿次那样的事件。结果是我们的感觉(以及概念)被平均化了(或者如物理学家所说的那样,成为经典的了)。当这些事件逐个地加以研究时,量子规律的特殊本性便会显示出来。
在我们的头脑中,空间不是具有物质本原的唯一东西。
如果在我们所处的环境中,周围事物不发生任何变化-譬如在极深的地下,或在长年远离星体而运动的宇宙飞船内-—我们将会失去全部时间知觉,因而也就会失去任何时间概念。
我们曾经说过这个事实:原则上时间有两种,一种是物体的固有时间,它是由该物体中的物理(和化学)过程来决定的;另一种则是由许多物体构成的许多巨大集团来决定的普遍时间。其结果,正如不存在摆脱了物体的空间一样,也不存在摆脱了事件的时间。
时间的过程是由事件,即因果连锁,来决定的。在物体的某系统中,事件愈活跃,它们就愈迅速地接踵而至(换旬话说,该系统中的相互作用愈强烈),因而该系统中的时间也就流逝

•276~
  
得愈快。
我们说过,这个结沦甚至被我们自己的经验所证实。充满事件的一天好象一眨眼就过去了,而清阱的日子却在拖着脚步走。隐藏在这个主观印像后面的是非常深刻的客观实际。

§115时空量子

爱因斯坦的新的时空概念到现在也没有被所有的物理学家接受。不仅如此,这些概念至今还没有在实验上被证实。
这种时空概念是四十多年前出现的,到现在也没有普遍地流行开来。当然,很多物理学家相信,它们包含着某些真理。
有关空间、时间与微观世界中物体的存在及其运动之间的相互关系的基本命题可以大致表述如下:既然微观粒子及其运动具有鼠子特性,空间和时间也应量子化。假如事情确实如此,则经典概念的最后堡垒就要崩溃。空间和时间将失去其连续性,并分裂成微小的、分立的部分。
这意味着:应当存在着某种所谓细胞一一-空间与时间量子。这些量子的大小可能是由微观粒子的质量、能量、动量(可能还有其他一些特性)来决定的。很自然,这些细胞势必成为所有数量中的最小数量。
可是直到目前,我们也不具备有关任何那样的长度元或肘间元的知识。这可能意味着,这个长度元和时间元,已经超出测昼微观世界长度和时间最精确的现代方法的灵敏程度。

•277•
  
这些现代技术的极限是核力距离——10一13厘米。在时间方面则屈于核时间的数星级一一10一23秒。某些物理学家相信,长度星子如杲存在的话,比上述数值还要短几百甚至几千倍。
非常有趣的概念。这样看来,我们从未注意到时空量子的存在是可以理解的。它们简直太小了。任何计时器也不可能测虽出那样短的时间,如一亿亿亿分之一秒。对长度来说也是如此,一厘米的一亿亿亿分之一也是测不出来的。
即使我们能够测出那种奇小的空间和时间釐子,我们也永远不能将它付诸实际应用。仪器是很粗笨的东西,它一旦对微观世界进行探测,便立即改变这个微观且界。最后,请回顾一下:在微观世界中,我们的长度和时间的经典概念是受到限制的,也就是说它们只在一定范围内有效3这些限制造成微观粒子的二象性:实物-场。可是构成这些限制的又恰恰是我们一直在讨论的空间和时间匾子。
引入时间细胞或批子又有什么意义呢?这样的细胞或量子还能继续反映我们关于空间和时间的日常概念吗?
这话问得很有道理。我们反复说过,知识的每一新层次不是从真空里冒出来的,而是建立在知识的原有层次的基础之上的。发展新概念极为艰苦的过程也不能在一夜之间完成。这个过程是缓慢的,而且新概念也总包含着前辈的痕迹。新概念总是在阵痛中诞生的。
量子力学的早年遭遇就是如此。现在,面临着重重困难的昼子力学又在阵疼中挣扎着。它将胜利?或将失败,让一个新的、更强有力的理论来代替?
