\chapter{量子力学的初级阶段}

\section{热和光}

严冬的夜晚,坐在暖烘烘的火炉的近旁,闲听炉里劈劈拍拍的响声,沐浴着炉火的温暖,该是多么美好啊!可是我们为什么会感到温暖呢?为什么火炉近旁就温暖呢?离火炉一定远的距离内,人们就会感受到它的热,虽然炉里的火连看也看不见。

炉子发射着某种看不见的、给人以热的感觉的射线。这种射线就叫热射线,或红外线。

略微仔细地观察会向我们说明:热辐射在自然界中是一件很平常的事。一支蜡烛,一团烈火,以及硕大的太阳,都既发射着热,也发射着光。甚至那些远得出奇的星星,也将热射线送到地球上来。

如果一个加热的物体在发光,它肯定也在发出热射线。光和热的发射实际上是同一个过程。这就是为什么科学家把看来是加热过程引起的物体的全部发射——光的发射和严格意义上的热辐射——都叫作热辐射。

在 19 世纪,物理学家发现了热辐射的基本定律。这些定律现在已为我们大家所熟悉。这里回顾一下两个定律。

第一,物体越加热,它发光越亮。每秒的辐射量随温度的改变而急剧地变化。如果温度增加到原来的三倍,则辐射增加将近一百倍。

第二,发射的颜色随着温度的增加而改变。观察一下放置在喷灯火焰里的一根铁管吧。开始的时候铁管可以说是黑的,后来就带一点暗红色,然后变得鲜红,之后又变成橙色、黄色。最后炽热的金属开始发射白光。

一个有经验的炼钢工人能够根据发光颜色,十分准确地估计炽热金属管的温度。他会说:暗红色意味着约 500°C 的温度,黄色约 800°C,明亮的白色则达到 1000°C 以上。

可是物理学家是不会满足于这种粗略的定性分析的:他们要求精确的数字。对物理学家来说,“天很冷“这句话并不比“他有张大脸”更有意义。需要的是他的特征:鼻子、嘴唇、前额长得怎样。

物理学家碰到过形形色色的物体,以及各种各样的热辐射条件。但热辐射的多样性使他们很不满意。他们需要拿某种标准物体作衡量的尺度,以便在这个基础上建立起加热物体的辐射定律。这样其他物体的光发射就可视为对此标准的偏离。想像一下这样的描述:这人的鼻子比标准的鼻子长一些,额头窄一些,下巴稍稍向前突出一些,眼睛比一般的略微绿一些,小一些。说也奇怪,这幅图画倒使物理学家高兴了。

下面我们就谈谈为什么会是这样。

\section{比黑还要黑}

取若干颜色好像是一祥的物体,仔细地观察它们,以便发现它们在颜色上究竟有何差别。

仔细观察发现它们的颜色是有差别的,这个颜色淡一些,那个颜色深一些,浓一些。这一差别是由于:落在物体上的光线一部分被吸收了,另一部分被反射掉。很自然,这两个分量之间的关系可以在极广的范围内变动。以两个极端情况为例,一个是发亮的金属表面,另一个是一块黑色的天鹅绒。金属几乎将落在它上面的光线全部反射掉,而天鹅绒却吸收了绝大部分的光线,也就是说,几乎没有反射任何光线。

魔术师可真会利用天鹅绒的这个特性,因为一个不怎样反射光线的物体实际上是看不见的。在舞台上的黑色背景前面,一口覆以黑色天鹅绒的箱子是几乎不会被察觉的,这样魔术师就可以用手绢、鸽子,甚至他自身来变各种各样的戏法,让它们忽隐忽现。

物理学家也同样发现了黑色物体的这一十分可贵的特性。在寻找标准物体的过程中,他们决定采用黑色物体。一个黑色物体吸收的辐射最多,因而这一辐射能将黑色物体加热到较其他所有物体都要高的温度。

相反地,当一个黑色物体被加热到高温并成了一个光源时,它的辐射强度较在同一温度下的任何其他物体都要大。因此黑色物体是一个可以用来建立热辐射定最定律的非常方便的辐射体。

当然,人们发现;不同的黑色物体是以不同的方式辐射的。例如黑烟灰就可能比黑天鹅绒更黑一些,但也可能没有后者那样黑,这全看它来自什么样的燃料。黑天鹅绒的黑度也不一样。这些差异不大,但对于标准黑色物体来说,仍以去掉这些差异为好。

于是物理学家想出了一种最黑的黑色物体——一口箱子。它是用来吸收热辐射的一种非常特殊的箱子。箱子的内壁装有若于肋状墙,整个箱子内部覆以黑烟灰。光线从一个小孔进去后便再也出不来了:它一且被捕便永无自由了。物理学家说:这口箱子把进来的辐射能量全都吸收了。

现在让我们把这口箱子当作一个光源:这实际上也正是制造这口箱子的真正目的。箱壁充分加热后就会变成白炽的,并开始发射可见光线。正如已经说过的那样,对于一定温度而言,这种箱子的热与光的辐射将大于任何其他物体,因而后者就被叫作灰色的,以示区别。

所有热辐射定律都是以这种最最黑的箱子为标准而建立起来的,因而这种箱子就获得了一个通称:“黑体”。只须作少许变化,这些定律也可应用于灰体。

\section{要准确的定律,不要草率的近似}

让我们用物理学的语言更精确地描述我们的定律。

笫一个定律说:黑体的辐射能力,即它每秒以光和热的形式发射的能量,正比于它的绝对温度\footnote{绝对温度是以摄氏温标零下 273 度为起点来计算的。}的四次方。这个定律是两个德国科学家斯蒂芬和玻耳兹曼在 19 世纪末发现的。

第二个定律说:随着黑体温度的升高,对应着它所发射的光线的最大亮度的波长将要变短,并向光谱的紫色区移动。这就是以奥地利物理学家维恩命名的维恩位移定律。

物理学家现在掌握了两个无例外地适用于任何物体的热辐射的普遍定律。第一定律正确地描述了物体被加热而发光时亮度的增高。可能乍看起来维恩的定律与观察的事实不甚符合,因为随着温度的增高,物体发射越来越多的白光。白的,而非紫的。

但是让我们更仔细地考察一下。维恩定律只说了对应于光辐射的最大亮度的颜色,此外则什么也没说。不言而喻,除了这一辐射而外,仍有较长的波长(即不同颜色)的辐射,而后者在较低温度时就早已发生。当物体继续被加热的时候,它的辐射使光谱范围变宽,也就是说它开辟了许多新的光谱区域。结果是,当温度升到足够高的时候,我们就获得了一个完整的可见光光谱。

也可以将上述情况和一个乐队的演奏作比较。在这个乐队里越来越多的乐器参加进来,奏出越来越高的音阶。这样,整个乐队就会发出一个强有力的、越来越高的合声。这个合音从许多大喇叭的深红的低音开始,然后上升到许多短笛的最高的尖紫音。白光是光谱的总体效果。维恩定律是有效的。但大自然却从另一个角度打击了热辐射的研究者。

\section{紫外灾难}

物理学家都有一个寻求普遍定律的癖好。他们一旦发现同一现象可以用几个定律在不同方面进行描述,便立即尝试将这些定律结合成单一的普遍定律,使之能同时概括所有的方面。

英国物理学家瑞利和金斯就作了结合热辐射定律的尝试。他们所获得的统一定律表明:热物体辐射强度正比于它的绝对温度,而反比于所发射光线波长的平方。

这个定律看来与实验结果很好地吻合。可是后来忽然发现:只是在可见光谱的长波部分,即绿、黄、红部分内,这个定律才能很好地与实验一致。当接近于蓝、紫、紫外光线时,这个定律便告失效。

从瑞利-金斯定律得出:波长越短,热辐射强度就应越大。但这点不能为实验所证实。不仅如此,一件很不愉快的事情是:随心波长的缩短,辐射强度应无止境地增大。

这种现象当然不可能出现。波强度的无限增长是绝对不可能的事。如果某个物理定律引导到无限性,则它注定要失败。自然界有许多大的东西,很大的,甚至大到难以想象的东西,可是,除自然界本身以外,还没有任何无限的东西。

辐射理论中出现的这个荒唐局面就被称为“紫外灾难”。这是 19 世纪末发生的事。那时候任何人也没有想到,这不仅仅是某一个颇为特殊的定律的灾难。它是摧毁这定律全部理论基础——经典物理学——的灾难。

\section{经典物理学处于困境}

在那些日子里,有的物理学家并不认为经典物理学道路上的这个辐射理论障碍有什么要紧。但任何障碍都是一件严重的事。因为理论中所有方面都是相互关联着的。如果某一点是错误的,我们便不能信赖这个理论对其他现象的描述。如果这个理论连这样一点小障碍都不能克服,那还能希望它去克服什么大障碍吗?

为克服辐射理论的困难,物理学家曾作过一些勇敢的尝试。今天,这些尝试看来在逻辑上是不一致的。但人们还能期待些什么呢?当理论处于危机之中的时候,它就像在一座着了火的住宅里的猫一样,只有一条出路——跳进河里去。可是这只猫却在房子里横冲直闯,连想也没有想到该跳进河里去,因为那样做是违反猫的全部本能的。

类似的事情现在却降临到科学家的头上了:他们被困在着了火的房屋里,而这所房子对他们是多么宝贵啊!他们毕生在里面工作着,他们已经非常习惯于居住在这所房子里。他们竭力想扑灭这场火灾,可是却不愿意扔下这所住宅逃命。

当然那些更锐敏的科学家清楚地意识到:经典物理学已经进入绝境。热辐射理论并不是唯一的死胡同。那些年代还目睹了以太学说的崩溃。

崩溃来得这样迅猛,以致许多科学家都处于束手无策的境地。还有什么事可于呢?

如果实事不符合理论,那是事实有毛病。大自然不想遵循任何规律了。大自然是不可知的!这就是那些神经衰弱的人作出的反应。

唯物地思维首的科学家对此作出的反应则不一样。如果事实不能被理论说明,那是理论不中用。理论必须在新的基础上重建,并立即重建。

历史再次证阴,伟大的需要产生伟大的人物。普朗克和爱因斯坦终于从具有万古不变的教条的经典物理学所处的困境中找到了一条出路。1900 年普朗克引进了量子概念。1905 年爱因斯坦提出了相对论。

\section{出路}

什么是普朗克的发现?

乍一看,很难说它是一个发现。这里有两个关于发热物体的热辐射定律。分别说来,两个定律都有效,但当两个定律捏成一个时,它便面临着紫外灾难。有点象两个思想方法差不多的人遇到一起了;交谈一会儿后,两个人都成了真正的疯子。

此时,普朗克已年逾四十。多年以来,他一直在研究热辐射。在他眼前,热辐射理论已经陷入绝境。和他的同事一样,他在寻找一条出路。他检查了推理的所有环节,最后确信推理上没有任何错误。普朗克继续深入研究,可是他走的道路却指右另一个方向。

后来他追述着:在 19 和 20 世纪交替的年代里,他以从未有过的青春活力和激情工作着。最最不可能的事在他看来成为十分可能的了,而普朗克就像个狂热分子一样,以无比的毅力建立了、又推翻了一个又一个的理论模式。

起初引导他的是一个十分简单的想法。瑞利和金斯将两个热辐射定律结合成一个,因而在短波长范围内引出了荒谬的结果。如果通过另一途径将斯蒂芬-玻耳兹曼定律和维恩定律结合起来,也许有可能获得某些合理的东西。

为了便于整理自己的实验资料,普朗克尝试找出一个不与资料相矛盾的普遍公式。经过一番探讨以后,他找到了那样一个公式。这个公式相当复杂。它所包含的表述式没有明显的物理意义,它纯粹是一些不相关联的量的偶然结合。可是非常奇怪,这个杜撰的公式却与实验极好的吻合。

不仅如此,普朗克能从自己的公式中推导出斯蒂芬-玻耳兹曼定律和维恩定律。总的来说,这个公式不具有任何无限大值。物理学家将会称它为一个正确的公式。

这是个胜利吗?是一条出路吗?不尽然如此。作为一个真正的科学家,普朗克是倾向于怀疑的。

胡乱地敲击钢琴的琴键 20 次也许能敲出个调子,但怎样能够证明这样做会奏出一个旋律呢?这个公式必须能从某些理论中推导出来。任何法则的赢得者,在他的法则被科学承认之前,都不得不受到批判。法则的嬴得者必须能将他同自然界决赛的每一步骤都加以证明,否则他的胜利是不会被记录下来的。

可是这里普朗克却失败了。这个公式不能从经典定律中推导出来。可是它又奇迹般地吻合着实验数据。

这就是普朗克所处的富有戏剧性的境遇。他将接受经典理论的观点来反对事实呢?还是站在事实一边去和旧理论作战呢?普朗克决定站在事实一边。

\section{能量子}

在经典物理学中,究竟是什么东西使普朗克公式不可能被推导出来呢?那无非就是经典物理学最基本的前提之一:能量是连续的。对当时的物理学家来说,这一陈述是十分普遍而且不可移易的。

乍看起来,上面这一陈述似乎与经典物理学的精神相违背,因为从一开始,它就把不连续性看成是隐藏在事物深处的原则。道理似乎是一目了然的。如果承认世界上存在着空无一物的空间的话,那么所有的物体都是孤立着的,都具有各自的边界。一个物体不能以连续的形式进入另一个物体之内。每一物体都终止于某个点上。

可能事物内部的状况不一样。否。事物内部看来也没有任何连续性。19 世纪末,经典物理学迫不得已承认了分子的存在,以及分子间的虚空的存在。分子也具有明显的边界,只有它们之间的虚空才是连续的。

顺便提一下,分子能够以某种方式通过这种虚空相互作用。从法拉笫那时起,经典物理学就已尝试以某种中间媒质的存在来说明这种相互作用:分子相互作用的效应是通过中间媒质传递的。

那么能量又如何呢?人们确信当分子彼此碰撞时,能量能以每一可以想象的数量进行交换。这个交换严格遵循弹子球的规律。一个运动着的分子撞击一个静止的分子,失掉自己的一部分动能,然后两个球沿着不同的方向飞开。在正面碰撞的情况下,主击分子甚至能停下来,而被击分子将以主击分子原有的速度飞开。分子在不断地交换着能量。

另外一种形式的能量也被发现。它与分子运动没有明显的联系,它是波动能量。由于麦克斯韦证明光是一种电磁波,因此光辐射(热辐射的一种)的能量必须遵从所有的波所遵从的定律。

不仅如此,这种能量是连续的。这种能量随着流水似的运动着的波一同传播。一定数量的能量连续地被消耗掉,就像水源源不断地注入一个容器一样。

当我们切下-片黄油的时候,我们并不去考虑这小片东西的连续性。我们总以为愿意切多小,就能切多小。当分子的概念被引进科学以后,人们清楚地认识到:比黄油的分子还要小的一块黄油是不存在的。

至于能量,则根本不存在分立性那样的概念。看来物质的原子结构并不要求能量也是由许多”小块”构成的。

只须环顾一下周围,就可以相信事情确是这样的。蜡烛的光将房间充满了均匀的辐射能流,正像太阳永远保持着它的永不间断的光线长河一样。我们同样还看到下山的火车、正在坠落的石头是怎样平滑地增加着它们的速度的(与之俱增的也就是能量)。

不妨想象一下能量也能一小部分、一小部分地被获得或被丢掉。回想一下早年的跳动的电影吧:蜡烛的火焰忽地腾起,忽地低落;太阳就像在放连珠炮似地把自己的光能以乍隐乍显的闪光形式倾泻出来。下坡的火车颠簸而行;石头在空中跳动,直至坠落于地。

“胡说八道!”这极可能就是普朗克最初的建议所遭到的评价。他的建议是;辐射能量(正像物质本身一样)也是由一个个能量原子构成的;能量的释放和获得不是连续的,而是一小份,一小份的。普朗克把每一小份能量叫做一个量子。这个字是从拉丁文 quantum 而来的,它的意思就是数量。要是普朗克知道这个数量将产生什么样的质量,他该怎么想啊!

在普朗克公式中,量子是个极其重要的角色。如果没有量子的话,他的公式将要遭到惨败,最后与许多缺乏实证的公式一齐被送进尘封了的科学档案堆。

这些能量子成了普朗克公式的坚实基础。可是这个基础本身几乎无处栖息,因为它在经典物理学中是没有任何地位的。这的确使谨慎的普朗克感到困惑。要摆脱一辈子的习惯可不是一件容易的事。

\section{行踪飘忽的量子}

光量子是一份极小的能量。最小的尘粒也有几十亿个原子。一个小小的萤火虫释放的能量也包含几十亿个量子。

现在让我们看看每份能量究竟有多大。对于不同型式的辐射来说,每份能量的大小也不一样,这是普朗克所做的极为重要的发现。光的波长越短,也就是说它的频率越高(换句话说,光线越紫),则每份能量也就越大。

如果用数学形式表示出来,那就是著名的频率与量子能量的普朗克关系:

$$E=h \nu$$

这里$E$是这量子的能量:$\nu$是量子的频率;$h$是一个比例常数。对于已知的所有形式的能量而言,这个常数都是一个定值。它就叫作普朗克常数或作用量子。它的数值是 $6 \times 10^{-27}$ 尔格·秒。这个数值尽管很小,可是对物理学家来说,它的价值却非常大。

正是因为量子的数值是这样地微乎其微,因而在我们看来蜡烛或太阳总是发出稳定的光芒。为了说明问题,让我们计算一下 25 瓦电灯泡每秒发射的量子数。假设发出的光是黄色的,根据普朗克关系,我们得出量子数为 $ 6 \times 10^{19} $,也就是每秒发出六千亿亿份能量。这么多份能量就是一个小小的 25 瓦灯泡在一秒钟内发射出来的!

很显然,对那样大小的能量来说,人的眼睛看来是不够灵敏的。可是事实却不是这样。眼睛是一个极其灵敏的仪器。苏联物理学家瓦维洛夫的实验令人信服地证实了这一点。一个观察者在黑暗中先呆上一段时间(以增加视觉的灵敏度),然后接通一个特别微弱的、每秒钟只发射很少的量子的光源。眼睛几乎能将光量子一个个分别地记录下来。

问题不在于量子的能量的大小,而在于它们是极快地接踵而来的。我们已经说过,甚至一个小小的灯泡每秒钟也能发射几千亿亿个光子。而人类的眼睛就像任何其他仪器一样,是以某种时滞工作着的。它不能从某种快速接续的事件系列中将各个事件分别地记录下来。正是眼睛的这种视觉保留特性使电影成为可能。这样我们就把银幕上的景物看成是一个连续的事件系列,虽然我们知道整个电影是由一帧帧不连续的画面构成的。

光源发射的能量子快得多地接踵而至,因此人的眼睛就把光线看成是一个连续的光流。

瓦维洛夫的实验是在 30 年代进行的,这时普朗克的量子观点已被普遍承认。可是普朗克本入却没有能够用直接的实验来证明自己的发现。

一个公式能被实验证明,但是又无理论依据,——乍看起来,这样的事情是很难理解的。尤其是导出普朗克公式的推理简直与传统观念格格不入,因而这样的公式就更加难以置信。这就是为什么普朗克在柏林科学院提出他的报告的时候,科学界并没有对此表示出很大的热情。科学家也是人,他们需要时间来消化那些不合常规的事情。

普朗克本人充分意识到他对经典物理学发动的进攻大胆到什么程度,因此他焦急地在为这样的进攻寻找理由。当然,他绝对想象不到仅在几年以后,就要出现使整个物理学革命化的巨大发展。

20 世纪最初的几年——也就是 1901、1902、1903、1904 这几年悄悄地过去了,人们对量子理论仍未给以任何重视。曾经发表过的有关文章也是屈指可数的。

\section{一个无法解释的现象}

到了 1905 年,瑞士专利局的一个默默无闻的职员阿尔伯特·爱因斯坦在德国杂志《物理学观点》上发表了他的有关金属光电效应的理论。

爱因斯坦开始从事这项研究的时候,光电效应已经有了许多年的历史。早在 1872 年,莫斯科大学教授斯托列托夫就已发现了这个现象。以后德国物理学家赫兹和雷纳德也对此现象进行了研究。

斯托列托夫在-个抽成真空的瓶子里放置了两块金属片,并将它们与蓄电池的电极连上。很自然,没有电流通过这一没有空气的空间。但当水银灯的光线射向其中一块金属板上时,电流立即开始在电路中流动。灯光一关闭,电流便停止。

斯托列托夫得出一个正确的结论:电流的运载者(电子)已在真空瓶内出现,而且只有当金属板被照射时,它们才会出现。

很显然,这些电子从被照射的金属板中抛射出来,很像分子从被加热的液体的表面跳入空气中那样。当然,这里“很像”二字实际意味看“颇不同于”;金属发射电子是一种根本不同的现象,它的性质在当时还没有被认识。

从头说起,光是一种电磁波。很难想象波能将电子从金属中轰击出来。这里并没有致使分子从液体表面轰出的高能分子间的碰撞。

另一有趣的现象也被注意到。对每种用作实验的金属来说,入射光线的波长看来都有某个极限值。一旦波长超过了这个极限值,不管照射的光线有多强,真空瓶里的电子就会顿时消失,电流就会顿时停止。

这是件很奇怪的事。很显然,电子从金属射出是因为光线以某种方式将能量传递给它们了。照射的光线越亮,电流的强度也越大。金属接受的能量越多,被轰击出来的电子也越多。

但不管光线的波长如何,金属如要发射一定数量、一定速度的电子,则应当接受完全一样的能景。当然,随着波长的增加,能量将要下降,因而从金属中被轰出的电子数也将减少,但仍应有一点电流在流动。可是实验证明电流已经完全截止。这样人们自然会想到电子已经停止接受辐射能。

但电子为什么对给与它们的能量食粮这样挑剔呢?人们很难想象出来,而对物理学家来说,这事也是很难捉摸的。

\section{光子}

爱因斯坦从一个不同角度来考查光电效应。他尝试描绘光线将电子从金属中轰出的真实过程。

在正常的条件下并没有电子云缭绕在金属表面的上空。这意味着电子是被某种力量拴在金属之中的。因此要把它们从金属中轰击出来需要一小点能量。在斯托列托夫的实验中,这一能量是由光波给与的。

但光波具有一定的波长,它的数量级为十分之一微米,而其能量则好似集中在一个电子所具有的微小体积之内。这意味着在光电效应中,光波表现出一个微小“粒子”的品性。它轰击电子并使之从金属中释放出来。

这显然就是一个光粒子;用牛顿的话来说就是微粒,因为牛顿不把光当成波而当成粒子流。可是那样一个粒子该有多大的能量呢?计算表明这个能像是很小的。因此,为什么不能假设这个能级恰好等于五年前普朗克想象出来的那个量子的能量呢?

因此爱因斯坦宣称:光无非是一束能量子流,而同一波长的所有量子是完全一样的,也就是说它们携带着相等的一份能量。以后,这些光量子又被取名为光子。

到此解释已经是很圆满的了。携带一小份能量的光子以足够的力量敲击一个电子,并使它从金属中脱出。

另一方面,如果光子能量不足以克服金属对电子的束缚,则电子不会被敲出,因而也就不会出现电流。根据普朗克公式,量子的能量是由它的频率来决定的,而光的波长越长,其频率也就越低。因此显然光电效应具有确定的极限。也就是说,如果光的波长太长,光子就不会具有足够的能量将电子从金属中释放出来。

此外,光线的强弱是没有关系的,也就是说,不管是一个个光子还是两个光子在照射金属并轰击它的电子,后者是不予理睬的。但如果光子具有足够的能量,则情况就要发生变化。在这种情况下,光线越亮,每秒钟进入金属的光子也就越多,因而被抛出的电子数目也就越大,相应而产生的电流也就越强。

这样光电效应终于找到了解释。可是和普朗克的假说一样,它动摇了经典物理学的基础。在经典物理学中,光被认为是一种电磁波,因而在任何场合下它也不可能是这种独出心裁的光子。爱因斯坦的理论又重新掀起了历时两个世纪之久的关于光的本质的争论。

\section{光是什么}

实际上这一争论从末停止过。当经典物理学还处于萌芽时期,这个问题就已出现,到现在它已阅历了一个暴风雨般的生涯。今天它又处于进退维谷的境地:光是什么?是波还是微粒?

物理学中这两种观点几乎是同时出现的。牛顿说:物体发光是因为它射出光粒子(微粒)流。牛顿同时代的荷兰人惠更斯则认为:物体之所以发光,是因为它在脉动并在周围的以太之中形成波。

每一理论都有自己的信奉者,而从一开始他们彼此就冲突起来。一百多年以来,这一斗争在激烈地进行着,此一时这一边居优势,过些时候另一边又占了上风。

最后,在 19 世纪初杨氏、夫累涅尔和夫琅和费的实验结果使波动学说获得了一个似乎是决定性的胜利。新发现的光的干涉、衍射和偏振与惠更斯的原理极好地吻合,但从牛顿的观点来看,则是不可理解的。

光学仪器开始发展。杰出的光学理论得到了发展,复杂的光学仪器也被建造出来。最后,麦克斯韦证明了光波的电磁性质,从而使光结构的描述达到完善地步。波动学说至此获得了彻底的、无可争辩的胜利。

将近 50 年的光阴过去了,光的微粒说再次复活。波动学说解释不了光电效应——这是一个多么恼人的瑕疵啊,不然光的波动结构将是尽善尽美的了——可是它却被敌对理论以令人惊异的方式阐明了。

百年的争论又重新点燃。但现在的论战却在一个新的基础上进行着。敌对的双方都已厌倦,并准备接受某种妥协。物理学家逐渐恍然大悟:光既是波又是微粒,这就是令人吃惊的,但又是不可回避的观点!

但光为什么不同时将两重属性一股脑地呈现出来呢?有时候光只作为微粒出现,而另一些时候它又以波的形式呈现出来。这个重要问题我们留在后面讨论。

随着爱因斯坦理论而来的第二个问题也并不简单。在光电效应中,电子不能对赋予它们的任何一份能量作出反应。这份能量必须大于等于一个非常确切的数量,不然光能将得不到反应。

事实上如果没有任何力量将一个电子与其邻近的诸电子束缚在一起的话,这个电子便不再挑剔能量的大小,而会对各种能包作出反应。但如果这个电子居于金属之中,它的脾气就会变得古怪,它再次要求特定份量的能量来解放它。这一现象大约在 20 年后才得到解释。

\section{原子的名片}

与此同时,-位青年的丹麦物理学家尼尔斯·玻尔尝试将新的量子概念应用于可尊敬的光谱科学。进入 20 世纪以后,讨论光谱学的文章已达几百篇。光谱分析以矫健的步伐前进着,很好地服务于化学、天文学、冶金学以及其他科学。

光谱的发现应归功于具有多方面才能的牛顿。但只是一个世纪以前光谱分析才与世见面。1859 年著名的德国科学家本生重复了牛顿的实验:在太阳光线经过的地方放置了一块三棱镜,从而将光线分解成光谱。但在本生的实验中,太阳光是由浸以盐液的布条燃烧时发射的光线来代替的。牛顿的发现是:太阳光线被展开成许多颜色的光谱带。但本生在其实验中却看不见光谱带。如果布条蘸以食盐(氯化钠),则光谱将展示出很少的几条窄线,此外便什么也没有。这些线中有一条是明亮的黄线。

本生的实验引起了另一个著名的德国科学家克希霍夫的兴趣。他们两人都正确地得出结论:三棱镜的作用只在于将入射光线根据其波长整理出来。太阳光谱的宽阔的光谱带表明:太阳光包含所有波长的可见光。当光源是一个燃烧着的布条时,光谱中出现的黄线表明:食盐光谱具有一个单一的、特定的波长。

氯化钠的分子式是 NaCl。但这条黄线究竟是属于哪个元素的呢?是属于纳?还是属于氯的?这点可以很容易地检验出来。钠可以用氢来代替,这样就成了氯化氢。这种化合物溶于水中就是盐酸。把蘸有盐酸的布条放在本生灯火焰中观察它的光谱。黄线消失的无影无踪了,这就说明它是属于钠的。

这点还可以用另一种方法来证实。钠还保留着,而氯被替换(如苛性钠 NaOH)。那条熟悉的黄线立即在光谱中出现。再没有任何怀疑的余地了。不管钠躲藏在什么化合物中,明亮的黄线便立即指明它的存在。黄线光谱线就是钠的名片。

后来发现钠并非具有特殊光谱线的唯一元素。每一元素都有它自己的独特的光谱。总的来说,其他光谱要比钠光谱复杂得多;它们有时是由为数极多的光谱线构成的。但不管这个元素存在于什么化合物或材料中,它的光谱总是明显的,正如一个人的相片一样。

可以用挨个检查身份证的办法,从人群中寻找某个人,正如化学家用化学分析方法从岩石标本中寻找某个元素一样。但比较容易的办法是掌握它的相片。借助光谱分析来寻找一个元素,就恰恰和对照相片来寻人一祥。有时元素存在的池方”验身份证”的方法行不通,例如,在太阳上,在远星中,在炼铁炉里和在等离子体中。

需要的是全体参与者的相片。今天已发现了一百多种化学元素,几乎所有的元素都已根据它们特有的光谱进行了分类。

\section{为什么物体会发光}

光谱分析的成绩是伟大的,但这里也还存在着一个根本的缺陷。光谱学大厦是建立在热辐射理论基础之上的,因而,带着这一理论的根本弱点的全部痕迹。这个根本弱点就寓于它对下述问题的回答中:为什么物体加热后会发光?

光是怎样发射出来的?显然,光是物体的组成部分——原子和分子——发射出来的。温度的上升使分子的运动加快。这样,互相碰撞就越来越猛烈、频繁;这样,分子就会急遁地振动起来,直至它们开始发光。这就是旧物理学的观点。可是物体的分子在室温下也在运动,为什么它们不发光呢?这个问题旧物理学却解释不了。
  
1898 年,英国科学家汤姆生建立了笫一个原子模型,发光的秘密似乎就会被揭示出来。在这个模型中,原子是些带正电的云。在这些云中负电子在浮动着,而负电子的电量恰好和正电荷抵消。电子被带正电的云吸引着,因而它们的运动也就受到阻滞。

根据经典物理学,带电粒子减速时必然会发射电磁辐射。显然,这种辐射即物体加热时发射的光。乍一看,这一解释是颇有说服力的。物体越是受热,电子在原子中的运动也就越快,这样,由于正电荷云的吸引,电子所获得的减速度也就越大,其结果辐射也就越强。

如果电子在辐射时不消耗能量,上述说法便可能成立。可是当电子辐射光时,它们就必然极快池减速。在极短的一瞬间,它们就会陷进带正电的云雾中,就好像葡萄干陷入布丁里一样。

肯定哪里有毛病。几年以后,汤姆生原子模型在其他方面的问题也明显地暴露出来了。得不到解答的问题真是太多了。比如说,电子为什么不直接和带正电的云融合在一起,并将它们的电荷中和掉?从这个模型中获得的极少的几个解答,一般说来也都与实验发生尖锐的矛盾。

1911 年,著名英国物理学家卢瑟福提出了一个新的原子模型。卢瑟福用新发现的放射性物质的 $\alpha$ 射线去轰击各种物质的原子。那时人们已经知道,$\alpha$ 射线是由带正电的粒子构成的。

卢瑟福研究了原子对 $\alpha$ 粒子的散射后,不得不作出一个意义深远的结论:$\alpha$ 粒子在其散射过程中,好像没有受到汤姆生原子模型中的整个正电云的排斥,却受到几乎集中于原子中心的一个极小部分的排斥。原子的全部正电荷看来集中于这一极小的中心部分之内。

卢瑟福把原子的这一部分叫做核心(核)。但电子又在哪儿呢?老观点认为电子亳无疑问是被电吸引力束缚于原子的电荷之上的。但由于电子存在于离开核一定距离的地方,因此必须有某种力能够抵消核与电子相互吸引的电力。

佷显然,这样的力必须时刻在起作用。原子在足够长的时间内生存着,因此这个起抵消作用的力,显然也必须像电子与核之间的电吸引力一样,恒定不变。

把这个力想象成一个离心力似乎是合理的。看来电子似乎在围绕着原子核旋转。可以计算出来多么大的力才足够使电子不致坠落于核上。计算表明;如果电子以每秒几万公里的速度,在与核相距一亿分之几厘米的地方绕核旋转,便能产生足够大的离心力。

这就是卢瑟福的原子模型。甩动系在绳子末端的球,使它绕着握着绳子另一端的手旋转,间接地给牛顿以行星引力概念的启示。现在这个概念又引导卢瑟福形成一个聪明的、完全正确的、有关原子的行星结构的概念——这点将被未来的事实证明。

现在我们能够回到物体为什么会发光这一问题上来了:我们将在这新的原子模型中寻找答案。电子围绕核的运动是加速运动(电子沿着密封曲线运动),因此必然辐射电磁波。无论是对汤姆生的原子模型来说,还是对卢瑟福的模型来说,经典定律都同样地适用。但遗憾的是,得到的结果也没有两样。电了由于辐射光,会把它的能量消耗干净。这样,在百万分之几秒的时间里,电子的速度就要缓慢下来,以致最后不可避免地坠落于核上,正如在大气层中减速的卫星必然坠落于地面一样。电子的命运应当与卫星的命运完全相同。在这种情况下,原子将很快地毁灭。

但原子却生存着。电子不应丢掉能量,也不应当发射光。但物体加热后却发光!

\section{玻尔写下的原子传记}

经典物理学再次陷入绝境。而这一绝境比可以设想的还要坏。经典物理学不能说明加热物体的发光,也不能解释光谱的本质。

你还记得蘸以氯化钠溶液的布条吧。这种盐的光谱只有一条黄线,这就意味着这种盐原子的辐射只有一个波长。

即使假定这条线是一个电子在原子内减速时发射的,我们也会立即面临另一个困难。经典物理学的定律声称:这样的电子应当发射的不是一条光谱线,而是一个由所有不同波长的光线构成的全光谱,同时这种光谱不呈现任何间断性。一个电子所产生的光谱不应区别于太阳的光谱。可是事实上我们面前只有一条黄线!

玻尔认识到肯定某个地方有问题。但什么地方有问题?可能卢瑟福的原子模型有毛病?不可能。拒绝这一模型还为时过早。玻尔的老师卢瑟福也持有同样的见解。人们认为应当这样改进这个模型,从而原子中的电子能围绕核旋转,发射光线,但又不致坠落于核上。

那是在 1912 年。爱因斯坦光子理论在物理学家中引起的轰动记忆犹新。而仅仅在此三年以前,又是爱因斯坦轰动了世界——他完成了相对论。很自然,所有这些对经典物理学的攻击不能不激励着年青的物理学家,使他们的思想更加大胆。

玻尔继续在这问题上沉想冥搜,最后终于得到了一个概念。为什么原子内的电子要连续地发射光线?是因为它总是以某个加速度在运动吗?让我们拒绝这种概念,并且宣称:即使原子内的电子在作加速运动,它也不需要发出光线。

但这怎么可能呢?电子必须沿着特定的路径绕核运动,也就是说,电子必须在轨道里运动,而不能任意地运动。如果电子不发射光线,则这个电子会无限期地生存在这个原子中。

但经典物理学却不可能对这种描述表示赞许。不仅如此,这种描述没有其他任何理论依据。玻尔不能对此加以证明,因此他谦虚地称它为假设。附带说明一下,玻尔在他的理论框框内也永远无法证明它。证明来自 10 年之后,而且来得出乎意料。这点我们将在后面讨论。但一个电子能沿着多少可能的轨道运动而不发射光线呢?玻尔的计算表明:这个数字是巨大的,可能大得没有限制。那么它的突出特征是什么呢?这就是它与核的平均距离:有离核近的轨道,有离核远的轨道。但问题还不在于距离,而在于电子在轨道中所具有的能量。这是可以理解的,因为电子离核越近,它就必须越快地运动,以免于坠落在核上。反过来对较远的电子也是一样:核对它的吸引力不是那么强,因而它能较慢地运动。

因此结论是:对应于电子的不同能量,电子的路径(轨道)也就不同。只要电子继续在它的轨道内运动,它就不会发射光线。

接着玻尔又提出了第二个假设。让我们设想某个轨道中的电子突然跳到另一个能量较小的轨道之中去了。多余的能量往那儿去了呢?能量是不会化为乌有的。

在原子外部去寻找这个多余的能量吧,玻尔说。

这个能量以量子的形式从原子中抛射出去了。这个量子也就是爱因斯坦称为光子的那个光能量子。

发射了光子的电子进入另一个轨道,并且不再发射光线。这个光子是电子从一个轨道跳到另一个轨道的那一瞬间发出的。

此时这个光子穿越其他的原子,并终于从这物体中逸出。它能进入我们的限睛。它能透过光谱仪的三棱镜,然后被拍照下来。光子包含的能量转化了多次,最后我们看见了它的现实的形象——照相底片上的一条黑线。

这条线可以说明许多事情。测量它在底片上的位置就可以发现光子的波长和频率。根据光子的能量与频率的普朗克关系就可以决定光子的能晕。这个能量正好就是原子中老轨道和新轨道之间的能量差。底片上这条光谱线的浓度表示射到这条线上的光子数目:数目越大,线越黑。光子越多,发射光子的物体也就越亮。

对光谱的这种解释是多么简单而且精彩啊!

对某一种物质来说,所有的原子都是一模一样的。这样,电子完全处于相同的条件之下。因此,当电子跳跃于两个轨道之间时,电子发射出的光子也就完全一样。电子在两个轨道之间进行的全部过渡终于产生出一条独特的光谱线。

我们已经说过,原子中有许多那样的老轨道和新轨道。一个电子能轮流居于其中任何一个轨道之内。

从较高能量轨道向较低能量轨道的每一跳跃都伴随着一个光子的诞生。由于不同轨道之间存在着不同的能量差,因此光子也就具有不同的能量和频率。于是照相底片便展示了一系列的狭窄的光谱线。气态氢的光谱恰好就是这样的。它具有几十条不同波长的光谱线。

总地来说,像钠那样只具有一条光谱线的光谱是罕见的。光谱一般具有几十条线,往往甚至有几千条线。有些化合物的光谱图样是那样地错综复杂,以致看来没有希望理出这些光谱线。但也有一些能使这项工作变得容易的规律可以遵循。

在玻尔理论提出以前,物理学家为了剖析某些复杂的光谱绞尽了脑汁。但当玻尔证明光谱是原子——更确切些说,是原子的电子——的传记时,这项工作被大大地简化了。我们只须将原子中的不同的电子轨道结合起来研究,就可获得观察到的光谱线。

与此相反,考察一个光谱就能得出有关原子中电子的不同存在条件的结论。实际上,我们关于原子的电子壳层的全部知识,都是通过对光谱的艰苦分析取得的。

\section{能量从何而来}

玻尔已经解释了原子是怎样发射光线的,现在让我们问一个为什么。为什么只是当物体处于高温时才发光?为什么在室温下它们不发光?

在回答这个问题之前,让我们稍扯远一点。我们刚才描绘的这幅很有说服力的原子图画必须上下颠倒过来。不是这幅图画有什么问题。不是的。只是电子轨道顺序必须颠倒过来。

我们本以为距离核最近的轨道是能量最大的轨道,因而电子从近轨道跳向远轨道时发射光子。实际情况恰恰与此相反。

让我们这样来描绘此事:在地上挖个洞,把一个球放置在洞底,在洞口的地面上再安放另外一个球。这两个球中哪一个球的能量较大?

一个有知识的人会说:这个问题不清楚。首先,你问的是什么能量?是位能还是动能?其次,从什么高度的平面去考虑这个位能?如果取地平面作起点的话,球在地面上的位能就可以被看作是零,因而洞里的球的位能就要小于零,这就是说,它具有负位能。但如果我们将位能从洞底计算起,则地面的球就要有一个大于零的位能。由于两个球都是静止的,因而在两种情况下它们的动能都等于零。让我们试用第一种参照系。

现在假设球在运动。这样,除了位能而外,还要加上动能。首先,如果球不从洞里跳出的话,它的两种能量之和,即总能量显然仍是负的。其次,如果球跳了出来,并沿着地面滚动,它的总能量将成为正的。

这个冗长的解释可能使读者感到厌烦了,但不论现在还是将来。它将有助于弄清许多事情。关键是:从能量的角度来看,原子中的一个电子就好像洞里的一个球。而一个自由的、独立的电子就好像地面上的一个球。物理学家已经同意把一个自由的但是静止的电子的总能量看作是零,用以计算其余电子的能量。

当然,电子和球并没有很多共同之处。可能唯一共同之处是:二者都在自己的运动中受到制约。球不能自动地离开洞,而电子也不能离开原子。这正是为什么原子能够存在的理由。

球越接近洞口,它离洞底就越远,它的总能量就越大(这就是说,它的负能值越低)。对电子来说也一样,它离核越远,它的总能量就越高;它离核越近,它的总能量就越低(当然,它的负能值越大)。

概括起来说,当电子跳至离核较近的轨道时,它的能量就会减少,因此在这样的过渡中,光子就会被发射出来。在另一方面,轨道离核越远,电子就越接近于从原子中“逃逸“出去,此时电子所具有的能量也就越大。现在让我们回到本题上来。

\section{激动的原子}

我们又要和那个球打交道了。它为什么不会坠落?这是个愚蠢的问题,因为它无处可坠。

对低温下的原子中的电子来说,情况也是类似的。它无处可跳。电子位于离核最近的轨道内;唯一可以坠落的地方就是核,但这是不可能的,正如球不能穿越地壳间下坠落一样。

电子的能量此时最低。它没有任何能量可失了。因此,它不能发射任何光线。

显然,电子必须首先居于离核较远的轨道,这样它就能过渡到离核较近的轨道之中。但问题是:电子怎样才能进入另一个临近的轨道?这和一个球升到梯子的顶端一样,比如,球被我们安放在那里,也就是说给它一些能量。

对电子来说也是如此。为了把它放置在较远的轨道上,我们必须给它一定的能量。说得更明确些,我们必须赋予电子的一份能量,至少相当于两个轨道的能量差。

可以通过不同的途径赋予这份能量。一个常见的途径就是原子的热运动。如果一个原子以足够快的速度撞击另一个原子,则它将把要求的那份能量赋予后者。在室温下,撞击虽然是寻常的,但能量都太低。当温度上升到几百、几千度时,撞击的结果引起能量的大量交换,电子跳至新轨道,而当其返回原轨道时,光线被发射出来。

能量是赋予电子了,电子居于一个外轨通之中。于是又将发生什么呢?核不允许电子在外轨道停留,哪怕是最短的一段时间。核将它拖回到较近的一个轨道之中,而当电子向内跳跃时,光子被发射出来。我们的眼睛察觉到了这个光子,于是我们说物体在发光,或在放射光线。

现在物体正放射着光线。让我们将温度升高,看看会发生什么。原子的热运动变得更有力,碰撞来得更频繁、更猛烈。电子在内轨道仅停留极短的时间。原子越来越频繁地进人物理学家称为“激发”的一种状态,然后返回“正常”状态,几乎跟着又离开正常状态而再度进入激发状态。

此时,每秒产生千百万的光子。随着温度的升高,光子数目增加,其来势之猛犹如冰山的崩落(请回忆斯蒂芬-玻耳兹曼定律)。

可是不仅仅是光子的数目在增加着。电子跳跃的距离也在增加着。电子最初只是战战兢兢地往返跳跃于邻近轨道和原来轨道之间,可是现在却朝着离核遥远的外轨道作着创记录的跳跃。再从那些轨道跳回时,电子将产生非常强有力的光子。我们知逍,光子的能量越大,它的频率越高,它的波长越短。随着光子数目的增加和光子频率的增高,发射的光线也就变得更明亮,更紫(回忆一下维恩位移定律)。

这样玻尔理论便能够一举解释热辐射理论和光谱学的基本定律。玻尔理论取得了这个伟大成功之后,光和原子过程的量子性质便昭然若揭了。很快这个理论便为大多数科学家所承认。

\section{玻尔理论的最初挫折}

说玻尔理论已经取得彻底胜利还为时过早。在往后的 10 年里,这个理论获得了巨大的发展。这个理论涉及的现象范围极为广阔。它们包括原子发射并吸收光的最最微妙的过程,以及原子和分子的精细结构。1914 年,考塞耳奠定了量子化学的基础,他的理论现在已被包括在每一种有关此专题的教科书中。1916 年,索末菲提出了有关原子光谱起源的更为确切的理论。到今天这理论还在帮助人们分析复杂的光谱。不仅如此,这个新理论能够解释新发现的原子和分子的磁和电特性。

与此同时,玻尔的理论却遇到了越来越多的困难。它不能够解释许多新事实,其中有些还是玻尔理论的产婆。

笫一个困难就寓于玻尔理论帮助解释的光谱之中。问题在于这种解释是不充分的。

我们已经说过,光谱线的特征不仅在于波长,而且还在于亮度。根据玻尔理论,我们能够得出电子轨道能量阶梯的各个梯距(也就是说能够得出电子从一个梯级\footnote{正式名称为能级,见下文。——译者注}向另一个梯级,即从一个轨道向另一个轨道跳跃时产生的光子的波长)。至于对有关光谱线的亮度作出解释,这个理论却无能为力。也就是说,这个理论搞不清楚怎样才能计算出光谱中的光子数目。

认为玻尔理论已经战胜了经典力学显然为时过早。虽然玻尔在开始时抛弃了经典物理学,可是后来又不得不皈依了它。这表现在他终于采取了所谓对应原理\footnote{或译为并协原则。——译者注}。

这个原理可以扼要地表述如下。经典物理学能够计算出光谱的亮度,但不能说明光谱的起源。量子力学能够解释光谱的本质,但不能计算出光谱线的亮度。玻尔的结论是:两种理论都必须使用;两种理论在它们多少互相吻合的领域内,应被套在一辆车上。

但在哪些领域内它们能够吻合呢?根据经典力学,在环绕原子核的轨道内运动的电子会越来越贴近核,最后便坠落其上。

根据量子力学,原子内的电子辐射分立的光谱线,也就是说,辐射不连续的光谱。两种光谱有何共同之处呢?

电子轨道的能量阶梯具有不同高度的梯级。轨道离核越远,相邻梯级之间的距离越小。原子能量阶梯有些像一个长梯子从一端顺着它看去时那样:越接近于远端,梯级显得越密集。就梯子而言,这只不过是一种错觉;而在原子中,实际情况就是这样。

但能级之间的距离对应着光子的能量或它的光谱线的波长。由此可知,光谱的长波谱线对应着电子在远方两相邻轨道之间的跳跃,因而这些长波谱线就必然比较密集,以致看上去几乎像是一个连续的光谱。

这样看来,在长波区域内量子光谱与经典光谱并没有显著的差别。在这区域内,物体受热后开始产生的光谱的亮度可以在经典物理学的基础上进行计算。然后我们再将这种计算扩大到整个“量子“光谱上去。这些就是对应原理。

这的确是个高明的想法,只是在实践中物理学家感到很失望。实验给出的光谱线亮度与理论亮度有出入。

一般说来,也很难期待更好的结果。如果一个理论还要求助外援,这个理论不会是很牢固的。如果一个人不得不向他的新对手求援,他一定是很虚弱的。

英国物理学家布拉格曾经说过,把经典物理学引入量子力学中,就像每周的双日传布“经典宗教”,单日传布“量子宗教”一样。有时科学也要信奉两个神,而且这样会有好处,可是这实际上只能说明它在理论上的软弱。

对新原理更细致的观察表明:对应原理还不是玻尔理论的唯一漏洞。实际上从开头起,它的全部基本前提都带着经典物理学的鲜明痕迹。

玻尔拒绝了关于电子运动的经典观点,但却将电子轨道概念引人原子中。他坚定不移地相信:电子绕原子核旋转正像地球绕太阳旋转一样。

玻尔禁止电子在轨道中辐射,可是为此他又找不到任何恰当的理由。玻尔理论对原子中光子的起源作了正确的解释,可是他所描述的光子产生过程仍然是个谜。这种描述不能从玻尔理论的任何假设中推导出来。

玻尔理论的这种双重性很快就显示出来了。不能纳入玻尔理论框框的新事实大量出现。当然,玻尔理论有它的优点。在认识原子世界上,这个理论向前跨出了一大步。但这个理论也有它的缺点。它虽然对许多不可能理解的事情作出了解释,而且这些解释超过了经典物理学力所能及的范围,可是几乎还有同样多的事情仍然没有得到说明。

采取新步骤的时刻到来了。新的步子终于迈出了。迈出第一步的就是法国物理学家路易·德布罗意。