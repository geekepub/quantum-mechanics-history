\chapter{从玻尔理论到量子力学}

\section{一个重要文献}

1924 年,英国《哲学杂志》9月号刊载了一位不知名的物理学家路易·德布罗意的文章。作者在这篇论文里阐述了有关物质波的可能存在的主要论点。

物质波?它是否就是人们所熟知的声、光以及其他实实在在的、能被入们的感官所察觉到,或者能被仪器记录下来的各种类型的波?

不是这样。德布罗意头脑里的波是一种迵然不同的波。德布罗意的观点离经叛道、玄而又玄、难以捉摸。与 25 年前普朗克提出的能量子观点相比较,德布罗意的物质波具有同样的独创性。这不仅在于两种理论对物理学来说都同样重耍,而且还在于两种理论都受到了同样的待遇:许多物理学家对它们表示出公开的怀疑。

可是,物质波究竟是些什么呢?

在深入这个问题之前,让我们看看当时已被研究透彻了的“通常”的波。

\section{谈谈通常的波}

把一个石子扔进池塘里,然后观察水面上的波的运动。这里附带提一下,水面波实际上是能被直接观察到的正在运动中的波的唯一形式。

乍看好像池水在和波浪一齐向前推进,但事情并非如此。观察一下小孩在他的玩具船的后面投掷石子,希望这样能把船引回原岸。可是波浪却在船腹下推进,而船却仍在原处起伏运动。这意味着水并未流向远方,只不过在作上下的运动。大石头所掀起的大波浪能使水获得一小点向前的运动,但这种运动也绝不可能达到任何较远的距离。

巨大波浪的这种“运载“特性在一种骑浪运动中得到利用,这种运动在澳大利亚等地是常见的。运动员站在一块大木板上,随着向岸边推进的有规律的巨浪起伏运动。在他站稳一个浪头之后,便能随着这个浪头以极高的速度向岸边飞驰。但只要他的动作有一点失误,他便会从波峰堕入波谷之中。

在这项惊险的运动中,负载着运动员的波浪能引导他驰向岸边。记住“导波”这个术语,往后我们还要谈到它。

在上个世纪,物理学家懂得了声音也是一种波动。人们发现声音能在空气、水和固体中传播。声波里究竟是什么东西在振动呢?那就是声音借以在其中传播的媒质质点:空气和水的分子,固体的原子。

一般说来,如果没有空气、水或其他物质作为媒质,声音将会消失。真空中没有声音。未来的宇航员可能在遥远的没有空气的行星上观察到火山的猛烈爆发,可是却听不到一点声音。他们唯一能感受到的只不过是脚下大地的震动。在月球上,宇宙飞船将在绝对的寂静中起飞。在那里根本没有我们在地球上所惯听的火箭发动机的隆隆声。

19 世纪,物理学家也弄懂了由于电荷的运动而引起的电磁波的性质。

遥远的星体或星云发射出的光和无线电波现在才到达地面,可是它们的旅行在几千年甚至几百万年以前就已开始。它们的旅途大部分是在广阔的几乎是真空的星际空间渡过的。在月球上,宇航员将在绝对的寂静中观察从飞船底部喷射出的绚丽的火焰。

在真空中,人能看得见,但听不见。这就是电磁波和其他包括声波的机械波之间最根本的区别。电磁波的传播不需要中间媒质,不仅如此,任何媒质都只能降低它的传播速度。

\section{让我们认识一下物质波}

让我们回到物质波这个问题上来吧。

德布罗意坚信物质波产生于任何物体的运动。这里所说的任何物休包括大至一个行星,一块石头,小至一粒灰尘或一个电子。正如电磁波一样,这些物质波也能在绝对的真空中传播。因此它们不是机械波。另一方面,它们却产生于所有的物体——包括不带电的物体——的运动。因此它们也不是电磁波。

在当时,物理学家并不知道其他类型的波,因而这些物质波的确是一种新的、尚未被认识的波。对此,许多年老的物理学家不禁嗤之以鼻。

这些老物理学家确信所有可能存在的波都已被发现。可是这位年青的德布罗意却谈论什么物质波,而这样的波既非机械波,又非电磁波。当然,没有物质也就不会有什么波,可以说也就不会有任何东西!

诚然,德布罗意并没有为自己的波想出一个很好的名称。

可是他又能做些什么呢?新事物总是先获得名称,然后科学家才会逐渐正确地理解这些名称。

对德布罗意来说也正是这样。他的物质波是那样地错综复杂,因而直到目前物理学家对它也仍有争论。我们还要更仔细地讲述德布罗意波,因为这种波构成了现代量子力学的基础。

\section{我们为什么看不见德布罗意波}

这可能是物理学家向德布罗意提出的第一批问题之一。

可是,一般说来我们怎样才能察觉出波来呢?不仅仅凭借我们的感官,因为感官毕竟是一些很不完善的仪器。人耳只能听到频率介乎 20 ~ 16,000 周/秒的声波。这些频率对应的空气中的声波波长为17米至2厘米。人眼能够反应的光波波长介乎 0.4 ~ 0.8 微米。这些就是自然界赋予我们通向波的窗户(当然,这里我们姑且将水面波撇开不谈)。

物理学家使用特殊仪器将人类感觉沌围以外的波转换为两个肉口以内的波长。这样便大大地扩展了我们对波现象的研究领域。无线电接收装置能接收到从宇宙深处射向地面的米或厘米波段的无线电波,从而使我们能对这些波进行研究。闪光计数器\footnote{闪光计数器使用一种特殊晶体来记录核粒子和丫光子。当一个粒子或一个光子轰击该晶体时,一个闪光便会出现并被灵敏的仪器记录下来。}使我们陷够发现原子核发射出的r射线。这些电磁波的波长只有一亳米的万亿分之几。

现在已经清楚,已被研究的德布罗意波波长范围是非常广阔的。既然如此,我们为什么一直没有能够发现这种波呢?
间题在于:如何去发现?机械波(如声波),波长有几米便能被人耳察觉。但一个收音机,即使调谐到这声波的波长,也不能接收到它。收音机只能接收无线电波。从另一角度来私无线电波不能被人耳或其他机械装置接收到,即使它的波长约几米。
包种接收器只能对某种特定一类型的』旦巴巴包三~旦凶察觉直严严罗意波呢又因为德布罗意波既不属于声波这一类?也不复旦二
还实际上这正回答了本章一开始就提出的那个问题。这点往下还要细说。
如果我们尝试确定物质波的波长,我们将能从另一角度来回答上述问题。德布罗意获得了一个结合这种新波的波长与运动物体的质量和速度的关系式飞这个关系式就是
h
入=--
mv

式中儿代表德布罗意波的波长,m及,分别代表物体的质匾和速度;h是我们的老朋友,普朗克常数。
这个常数有很重要的意义,因为它表明德布罗意波具有噩子性质。这个问题往后我们还要讨论。这里,让我们先看翟根据德布罗意波,我们周围物体的运动是和怎样的波长相对应的。让我们粗略地计算一下行星、石块和一个电子在运动时的波长吧。
我们一眼就能看出这些物质波的波长是非常短的,因为分子是普朗克常数,它的数值是极小的:6.6X10寸尔格·秒。
以地球为例。它的质量是6X10万克,环绕太阳的轨道速度约3X106厘米/秒。把这些数值代入德布罗意关系式中,得出地球波的波长为
;l.=
6.6X10一27
=3.6XlQ-61厘米6X1027X3X106

这个数值小得出奇。任何现有的以及未来可能拥有的仪器都不可能记录下这样小的数值。看来我们也不能通过任何

•	1905年爱因斯坦发表光电效应理论:光子的能尽E=加,动昼P=h/入.密布罗意的基本观念是把光子的基本规律推广到粒子的运动上去。参看”。--译者注

•60•
  
比较来说明这一数值究竟小到什么程度。
让我们再看看一块石头虳波长吧。设这块石头重100克,飞行速度为每秒100匣米。根据德布罗意公式,得出
6.6X10勺
	).,=	=6.6XIo-,i厘米
100X100
这个数值也并不比地球的德布罗意波长更能说明间题。
发现这样的物质波是绝无可能的,因为它的波长要比一个原子核的线度小一百亿亿倍。而原子核的本身已经远远超出任何显微镜的观察范围。
现在再以电子为例。它的质昼约为10-21克。如果电子在1伏特电位差的电场中运动,它将获得每秒6X107厘米的速度。将这些数字代入德布罗意公式中,得出
6.6X10寸
	).,=	=10一7
6X107X10一刀
厘米

这个数字就大不相同了。10一1厘米差不多相当于X射线的波长,而后者是可以被测出的。因而,在理论上,我们应该能够测出电子的德布罗意波。

\section{物质波被发现了}

但怎样被发现的呢?德布罗意波在理论上是存在的,可是看来似乎没有任何方法能用仪器将它测出。但波毕竞是波,不论这种波的性质如何,它总要显示出自己作为波的某些现象。入们尝试通过衍射实验来捕获德布罗意波,理由是衍

•61•
  
射是一种彻底的波现象。衍射的本质在千:当波遇到障碍时,它会绕过障碍前进。这样波的路线将稍稍偏离原来的直线,而进人障碍物后面的阴影区域。
绕过一个圆形障碍物,或通过不透明屏障上的圆形孔眼所得到的衍射阳样,将是一组典型的明暗相间的环。例如,通过一面沾有灰尘的玻璃来观察路灯,就可以看到这样的图案。在降霜的夜晚,月亮将有若千明暗相隔的环围绕着它,月光由于通过漫天飞舞看的冰结晶而产生衍射。
衍射是波的存在的铁证。十九世纪恰恰是光的衍射的发玑成了光的波动学说的最有说服力的论据。
但光波的波长要较电子的德布罗意波的波长大几百甚至几千倍。所有用来产生光的衍射的装置,如狭缝、屏、衍射栅,都显得太粔柲。为了观察一个波的衍射,所设障碍的线度必须相当于或小于这个波的波长。因此,对光波能够使用的方法,对德布罗意波来说则根本行不通。
1924年还没有过去,人们就已知道应该使用什么物体去测定德布罗意电子波的衍射。在此十二年前,德国科学家劳厄就已经发现X射线通过晶体时的衍射。劳厄在暴露于通过晶体的X射线的照相底片上发现了一组组阴暗光点。几年以后,德拜和谢勒用细小的晶休粉末制成的光栅重复了劳厄的实验,并得到一组组衍射环。在这种情况下,衍射之所以可能,是由于晶体内原子的间距(对X射线来说,犹如不透明的“屏“上所开的许多窄缝)与X射线的波长属于同一数量级,即10一8厘米。

德布罗意波的波长也恰好居于这一范围以内!这就意味看,如果这种波确实存在,则电子通过晶体后,也必然会象X射线一样,在照相底片上产生同样的衍射图样。
德布罗意提出他的新颖见解后几年,美国科学家戴维孙和革末以及苏联科学家塔尔塔科夫斯基用一种晶体作电子的衍射实验,证实了他的见解。
当然,只凭”电子射线”和X射线之间的相似是不足以完成这个实验的。它必须具有自己独到之处。
X射线通过晶体时几乎是畅通无阻的,但当电子通过几分之一亳米厚的晶体薄片时,则将全部地被吸收掉。因此需要的是极薄的晶片或金属馅。也可能须要用障碍物来代替缝隙。在这项实验中,让电子射线与晶体表面构成一个很小的角度,这样电子便能在这个表面上掠过,而不致深入到晶体内部或被弹回。其结果,电子只对晶体的最外层原子产生衍射。产生衍射的电子在照相底片上被记录下来飞
水电子如可见光及X射线一样,能使照相底片形成雾鬻。—一原注

•63•
  
塔尔塔科夫斯基将电子射线送向一个由许多细小晶体构成的薄笛。暴露时间达几分钟之久。
底片冲洗后显现出真正的衍射环轮廓。这些底片一一它们要比相同重谥的金子还贵重一被送往世界上几所最大的物理学实验室,在那里这些底片受到了仔细的检验。没有任何怀疑的余地了。德布罗意有关物质波的大胆假设终于被实验出色地证实了。电子显示了微粒的特性以及波的特性。

\section{具有双重属性的粒子}

甚至在进行这些具决定意义的实验之前,科学家就已经尝试寻索德布罗意波的真实意义。人们应该怎样去理解粒子(如电子)活动的这种双重性质昵?
在那些年代里,物理学家知道什么是电子,那就是一个很小、很轻、携带着一个微小电荷的物质粒子。在一个很长的时期内,并没有入去问,这个粒子究竟是什么形状的,或者在它的内部还发生着什么3直接观察一个电子是不可能的,更不月说弄清楚它的内部结枯了。但如果电子是一个粒子,它显然还必须具有粒子的特性。这样,电子又怎能拥有与粒子绝然不同的波的特性呢?
最先尝试解释物质波的便是德布罗意本人。这清楚地表明:当物理学家刚刚进入微观世界时,他们由于习惯,总是继续沿用有形的模型。在玻尔-卢瑟福的理论中,原子就像一个行星系统,在这系统中电子行星环绕核太阳旋转,唯一不同的

•64•
  
是:行星总保持一定的轨道,而电子却经常在变换自己的轨道。
继此而来的便是光匿子,即光子。正如爱因斯坦所示,光子也具有波粒二象性。显然,这样的二象性物质是不能用图象来表征的o
这样,物理学便面临着笫一个不能表征的存在,而现在,由于德布罗意的发现,这种不可想象性又扩展到实物粒子,扩展到小至电子大至巨大星体的一切物体。这的确使人望而生畏飞
人们怎能想象飞向一个障碍的电子,由于衍射的结果,会绕过它并达到它的背后?不能。波与微粒是两种势不两立的存在。一个东西要么是波)要么是微粒!
可是德布罗意波确实存在。它不再是“非此即彼气而是“亦此亦彼"飞这样就势必要把不相联系的联系起来。这不仅局限于产生衍射的电子这一特例。假如电子具有波特性,必然世界上所有物体,从最小的到最大的,也都具有波特性。
德布罗意建议从导波这个概念开始来进行这一不寻常的综合。

\section{导波}
让我们重温一下骑浪这项运动。骑浪人占住了巨浪的浪

*恩格斯«自然辩证法»:"辩证法...…不知道什么无条件的普遍有效的`非此即彼',它.....除了`非此即彼勹又在适当的地方承认`亦此亦彼'I"参看«自然辩证法))人民出版社1971年版190页。一译者注

•65•
  
头,这巨浪把他带到岸边。这里巨浪充当一名向导。
德布罗意的想法是:物质波以类似的形式在引导运动羞的实物粒子。一个粒子仿佛坐在一个波上,随波而驰e
德布罗意说:这种波的波长可能很长。当电子运动的速度甚低时,电子波的波长要较屯子本身大好几千倍。随着速度的增加,粒子仿佛将波收进其自身之中,因而波就变得较短了。但电子即使在作高速运动,它的波长也仍然大于电子本身的线度。
I究竟谁引导谁一是电子引导波,还是波引导电子一一一是无关紧要的。重要的是:波与电子是紧密地而且是永恒地联系着的。只有当电子停止运动时,电子波才消失。在这一瞬间,德布罗意关系中的分母变为零,因而波长变为无限大。换句话说,电子波的波峰与波谷相距如此之远,因而电子波也就不成其为波了。
德布罗意的图画是十分生动的:电子骑在自己的波上。
但波又是从哪里来的呢?它与粒子俱存,即便后者是在绝对的真空中运动着。这就是说,波是粒子本身产生出来的。但这又是怎样发生的呢?
德布罗意的假说并未对这点作出任何解释。退一步说,也许这个假说能够解释粒子与其波之间的相互作用;波如何与粒子一齐在运动;当粒子与其他粒子和场相互作用时,例如,当粒子遇到障碍或照相底片时,波将如何分担粒子的命运。否。在这些问题上,德布罗意的假说也未提供任何有说服力的觥释。

•66•
  
为了寻找一条出路,德布罗意甚至想把粒子完全丢掉。
难道不能没想波本身就是粒子吗?换句话说,可以将粒子想象成波的紧凑结构一一波包,就像物理学家称呼它的那样。一个波包是由几个相当短的波构成的。当两个或两个以上的波包相碰撞时,它们应当象粒子一样地活动,正象短波光子将电子从金属中轰出时那样。但不管这个波包是多么紧凑,也不管它多么象一个粒子,它毕竟是由波构成的。这当然意味着:必然存在着某些现象,在其中波包能将自己的基本波性质显示出来。
可是自然界也同样地拒绝了这种说法。事实上不论波包是多么紧凑,它也不能形成一个粒子。这是根本不可能的。关键是:即使在绝对真空中,这些波包也将随着时间迅速解体。转瞬间,波包在空间中将变得模糊不沾:,换句话说,将错就错的解释反而使原来紧凑的粒子丧失了形状。可是我们知逍粒子是十分稳定的,没有任何迹象表明它会随着时间而弥漫开来。
这个模型也必须被抛弃。将波与粒子这两个实质互不相容的机械池结合在单一的形象中没有获得成功,也不可能获得成功。但这是后话。德布罗意仍然不想放弃他那个头是粒子身体是波的“半人半马的怪物"*0
两年的时间过去了。1927年夏,全世界的物理学家来到布鲁塞尔,参加索耳威大会。在大会上,德布罗意关于波及粒

*引自希腊神话。他是被绑在地狱车轮上的受罚神伊克赛恩的后裔,性格暴烈。--一译者注

•67、
  
子之间的关系的表述遭到了全体一致的否定。往后许多年内,对此关系的一种截然不同的表述在前带路。这表述是由两位年青的德国物理学家一一悔森堡和薛定谔—一在大会上提出的3

\section{衍射一一一是群体还是个体的现象}

海森堡和薛定谔埋葬了德布罗意的想法。他们谈得那样头头是逍,从而决定了往后杂子力学的全部发展。
德布罗意关于波与物体的运动相关联的基本思想很快就被许多国家的科学家采纳。德布罗意的第一篇论文发表后不到一年,饱国物理学家圾恩便提出了自己的有关德布罗意波的观点。
玻恩的学生海森堡这时刚开始自己的科学生涯。他对这个问题也产生了兴趣。与此同时还有另外一组物理学家,其中包括薛定谔,也对德布罗意的研究工作展开了热烈的讨论。
下面我们将不按照历史事件的先后来谈一下主要事实。
衍射相片一出现便使人意识到它的重大意义。这相片是最后的插曲,但它却大大提高了戏剧效果。
回顾一下证实电子衍射的实验。电子射线冲击一片晶休(或一片很薄的金属箱)。由于晶体中原子的障碍而产生衍射的电子束,射击在照相底片上,使其形成雾器,并留下衍射环。
这里补充说明一下,由炽热的金属丝产生的电子射线是

•68•
  
用特殊的方法形成的。在电子源和晶休之间安插了一个开有小圆孔的隔衱九这祥,电子射线穿过隔板上的小圆孔J元应该具有固定的横截面积?
设想实验刚一开始便停止下来,此时,比如说,只有几十个电子射出,我们将获得什么样的结果?照相底片冲洗后,我们将看到,它很象一个满是弹痕的靶子,而弹痕的布局说明射手是个初出茅庐的新手。黑色斑点对应看单独的电子击中的部位,而这些部位十分漫无目标地分布在整个底片上o
如果实验继续进行下去,我们就会发现,电子射在靶上的部位将逐渐展示出规律性。在几千发射击之后,底片将呈现出轮廓清晰的明暗环,这就是科学家实际观察到的衍射环。
这是件有趣的巾实。显然,当参加衍射的电子数目很少时,波特性显示不出来。只有当电子的数目很大时,这种特性才会呈现。换句话说,粒子的波特性似乎只能由群体表现出来o
为了找出答案,我们再次进行实验。同祥还是电子衍射实验,但做法却不相同。我们先采用一个强电子源,并给底片以很短的曝光时间。这样衍射环将很快地形成。与此相反,我们再采用一个闭电子源,同时却延长曝光时间。假设在两种情况下射击在底片上的电子数目相同,则所获得的两个衍射图样也将绝对相同。
这是非常重要的。在第一种情况下,电子对晶体的衍射

*或光阑。一一译者让

•69•
  
是同时产生的,也就是说它是某种群体性质的东西。在第二种情况下,电子是一个个地射向晶体的,因而群体的概念很难运用。设想有这样一心队怢路工人,今天这个工人来焊接,明天那个工人来动一动某个螺栓,过了一个月以后,第三个工人再将这个螺栓旋紧,那还算个小队吗?
不管是成千个电子同时完成衍射,还是一个电子接着一个电子去完成它,得到的图样是一样的。结论是明确的:每个电子不依赖于其他电子而独自显示其不寻常的特性,就好像其他电子并不存在似的。

\section{访问靶场}

让我们先来看看布满弹痕的靶子。这些痕迹是为数很少的电子留下的。乍一看,电子好像是漫无目的地射向底片的。
但有一件事引起了我们的注意。我们量度一下允许电子通过的隔板小孔的尺寸,并照着它将小孔的轮廓描绘到靶上。看起来,所有的电子都好像应该落在这个轮廓线以内,而事实上电子在底片上留下的痕迹是十分杂乱的,许多痕迹甚至远远地超出了界线。
另一件事也很有趣。仔细观察一下这个靶子,就会发现,电子向底片的射击完全不是漫无目标的。即使射向靶子的电子为数很少,也存在着全然没有击痕的空白区域,而在另一些区域内则密集着击痕。如果通过击痕密集的区域划一条线,则一个个小环将会出现。

•70•
  
当然,这些密集的击痕所形成的图象开始并不明显,但随着射向底片的电子数目的增加,图象便越来越清楚。
让我们耍个小花招吧!取一个步枪靶子,按照相底片上的电子击痕在靶上打出孔眼。将这个靶子拿给老练的射手去照瞧,看看他的意见如何。
”这个射法真奇怪。你看,第10环里有那么多的弹痕,而第8环或第9环却一个没有。是有意这样射的吗?所有的弹痕都集中在第10、7、4、1这几环之中。”
我们一言不发。过了片刻,这位老射手说道:“胡闹!任何人也不可能在靶上射出这个样子,不管他多么想这样射。道理是这样的:如果他是个新手,他射出的弹痕将是杂乱无章的,这些弹般将均匀地分布在整个靶子上。一个有经验的射手的靶子则完全两样:大卧的弹痕集中在靶心周围,只有少谥的弹痕分布在外环。让我们数一下每一靶环里的弹痕总数,并作出一个图。
“在横轴上我们标出环的序数(让横轴代表离开靶心的距究,也是一样);在纵轴上标出两环之间的弹痕数目。我们将获得一个随着到靶心距离的增加而衰减的平滑曲线,再看看你那个靶子。从中心到两侧,这个曲线在一起一伏地振荡着。这个曲线的衰减,与我们那个曲线相比较,形式完全不同。
“对老射手的靶子来说,机遇法则是有效的。我们所获得、的曲线就叫作无规则失误曲线,或高斯曲线。你那个曲线也有某种不规则的地方,但它遵循的是另一种不同的、对靶场来说是颇为新奇的法则产


现在让我们再来看看自己的靶子吧!

\section{几率波}

真的,波形曲线从来没有在射击中遇到过。电子不是子弹。子弹的质匿太大,从而不能显现其波动特性。
被晶体反射的电子在照相底片上留下痕迹。这些痕迹形成的分布曲线,玻恩建议称为德布罗意波。
且慢纸上的波和真实的波,二者之间有什么关系呢?真正的波与电子一齐运动,而我们的波却停留在纸上。
当然,它们之间还是有关系的,照相底片上电子击痕所形成的图象并不是幻想的产物。它反映的是与运动着的电子相

•72•
  
联系的真实的波的存在9但这利波的意义与德布罗意所申明的意义是迥然不同的。
牛顿的经典物理学非常肯定地认为,从隔板开口处逸出的电子在遇到晶体以前应作直线运动。然后电子将被晶体的原子反射,正象弹子从弹子台的凸沿弹出一样。最后,电子远离晶体,射向照相底片,并在它上面留下痕迹。
这里并没有射手的颤动的手或疲劳的眼睛,也没有风或从地面升起的热气流,来影响瞄准过程。条件是理想的,因而准确度也应当是理想的-~换句话说,电子应当在照相底片上准确地复制出隔板孔眼的原来形状。如果隔板孔限是一个小圆润,则照相底片上除了复制出一个小圆斑而外,不应该留下其他任何痕迹。
但屯子拒绝服从经典定律。除了一个小圆斑而外,还出现了一系列的明暗环。这并不是射击的不准确造成的。即便我们认为确实是这样的,电子的分布也应当服从高斯定律。但事实上它们却按照与高斯定律截然不同的波定律来分布的。
\屯子在照相底片上的分布曲线在形状上是波样的。光以及X射线的衍射囡样的强度曲线也具有相同的波形状,而光与X射线肯定无疑是波。
因此,电子的波特性的表现方式较德布罗意想象的更加难以捉摸。电子波不是一架飞机,里面乘坐着电子旅客。电子波决定电子射中照相底片上的某点的几率。因此玻恩建议给它取个更恰当的名称一几率波。、

•73•
  
\section{几率进入物理学}

在经典物理学中,我们从未遇到过几率这个名称。每一质点或物体运动都被认为是由作用其上的诸力严格地决定了的。只要我们知道作用于一个物体的诸力,该物体的位置以及开始进行计算的参考时间,我们便可以准确地预言在任何一瞬时一一一下一秒钟或一百万年以后-—-该物体的位置和速度。
上世纪中期,物理学对气体的内在运动进行了研究。人们几乎立刻明白:牛顿的公式不能直接应用于气体分子的运动。
你不妨设想一下,即使在极小的体积内,气体的分子也是数以亿亿计的。如果要准确地描绘气体分子的运动,应该写下并解出每个分子的运动方程。分子从来不会静止下来:它们与其他分子经常相互碰撞,把有些分子弹射出去,又撞上另一些分子。每秒钟这祥的事件要发生几百万次。
即使幻想写出所有这些分子的牛顿方程也是荒谬绝伦的。仅仅写下这些方程也要花几百万年的时间。再解出这些方程还要几万万万年的时间。可是到那时候,这些运动早已被其他运动所代替了。
在探索一条合理的道路的过程中,物理学家认识到:他们不应当对以极高速度碰撞其他分子的各个气体分子的运动感到兴趣。相反地,他们的兴趣应当寓于气体的总体状态:

•74•
  
它的惊度、密度、压强以及其他特性。
及有必要算出各个分子的速度。气体的状态的全部特征都应当归属于作为一个共团的整个分子系统。因此这些特征便主要地由气体分子的平均速度来决定。平均速度越高,温度也就越高。在这一过程中,如果气体不改变其容积,则压力将随品度的增高而上升。
为了准确地掌握这些关系式,必须设法求出分子的平均速度。这里便须引用几率理论。
几率理论声明:不可能设想每一瞬时所有的气体分子都具有相同的速度。与此相反,它们具有不同的速度。不仅如此,这些速度由于碰撞在经常地改变着。可是,尽管这些速度变化具有不规则的性质,在给定的条件下,每一瞬司都存在着分子的某种平均的、稳定的速度。对于一个分子而言的不规则性,当应用于大数匮分子时,则转化为规则性。这就是大数泣的几率法则。一般容积内的气体分子数目的确是非常大的。实际上这个数巨是如此地大,以致我们可以毫不迟疑地应用这一法则。
这样物理学家便开始根据几率理论的法则,用统计方法计筛巨大的分子集团的活动状况。但在某一方面,他们不想与几率原理取得一致。他们坚信,分子运动中不存在不规则性;每一次碰撞,每一个分子的个别运动都可以用牛顿定律表述出来;如果有人愿意去求几亿亿个方程的解,他便能够绝对精确地表述出这些运动,而无须求助于任何平均值。我们当然不会这样做,但原则上这是行得通的。因此,当人们用儿率

•75•
  
法则、统计法则来描述气休运动时,确信深藏在上述法则后面的是牛顿力学的精确定律。
可是经典物理学却有点过于自信了。将牛顿定律推广到各个分子的运动是没行任何根据的。物理学往后的发展证明了这点。分子不是弹子。它门在运动着,并在相互砬撞着,可是却迫从一些全然不同的法则。

\section{谨慎的预言}

电子、原子和分子遵从的是些新的定律。这里首先造反的是电子。它们不想迁就经典物理学的框框。它们不朝向照相底片上应当射击的地方射击。电子的这种叛逆性虔动了某些科学家,他们戚道:"屯子竟然使用自己的自由意忐,它们竟然为所欲为!”
哲学上软影的物理学家很容易被引入歧途。如果电子真有什么“自己的意志”,那就不会有任何能为它遵循的定律,这样它便是一个真正的无政府主义者了。如果事情真是这样,我们又何必需要科学}因为科学寻求的是法则,而法则并不存在!他们这样地推理:上帝允许电子(因而也就允许世上所有的东西)自由活动,为所欲为,除了下面这一条法则而外可以不受任何法则的约束,这就是上帝自身存在的神圣法则;可是科学并不去审查这条法则,科学以绝对的信念把握着它。你看,问题是多么简单:从电子的自由意志出发坠入了彻头彻尾的唯心主义。

•76•
  



  
击中灰区的可能性次之,击中亮环简直没有什么可能性气
这话好像过于谨慎了吧。科学一向要求被认为是准确的,现在竟说出这样的话,听起来页叫奇怪。这样的话甚至不象是科学。与此相较,经典物理守的绝对准确的预见该有多么好和可是,只要认真想一想就会发现,那样的预言实际上是吹牛皮;它只暴霹了自己的傲慢与尤知。
当然,科学才刚刚开始探索一个无限复杂的世界,对其中发生的事件几乎一无所知,乌此同时还要做出如此明确的判断,我们怎能苛求于它呢?
可是,对于经典物理学,我们可能过于苛刻了。在日常事物尪围内,经典物理学还是有成效的。只是在此以前,它从未接触过什么景子、粒子的波特性以及其他许许多多的惊人的习l物罢了。
是的,每门科学都力求获得有关自己的研究对象的精确面完备的知识。这显然是基本鹄的和座右铭。但也绝对不会有这么一天,那时每件平悄都被认识了,科学再没有什么中情可做了。
所有这些谨慎的科学预言,所打这些可能和或然的意义就在于此。谈论或然正意味着我们对于某一现象的知识还并非绝对地完备而准确飞

*玻姆认为目前篮子力学谈论几半是庄于还存在石某种未被发现的“隘变数”。揽子力学不适用于小于10一13厘米的领域(参看本书\section{5),在此领域内,波函数平方不再是几率密度,因而原则上可能同时精确漪定粒子的位置和动愤。参看周世勋著«觅子力学动仔404页。--译者注}

•78•
  
人们可以容易地想象,如果气象员作出下面这样的预报,他将显得多么愚蠢:“明天天气热,无雨,上午9点温度23.8屯;12点,29.6"C;下午4点,27.4°c。下午1点乌云将在面积为多少平方米的某某区域的上空出现,停留多少分钟。下午5时乌云将以每小时12.3公里的速度向东北方向移动”。
有几十个因索参与气候的形成。气象学在目前状况下不可能考虑到并精确地描述构成完善的天气预报的这许多因素。远远做不到这点!
对僵子力学来说,事悄还要困难得多,因为它研究的对象是一个无比复杂的微观世界。

\section{粒子波和波粒子}

让我们再回到德布罗意波上来。这种波决定着电子运动。但它是以几率形式,而不是完全准确地,决定着电子的运动的。在电子衍射实验中,电子波指明照相底片上哪些区域电子将以最大的几率射击。
但玻恩把德布罗意波看成是几率波。他是否搞错了呢?德布罗意波会不会是一种全然不同的东西?如果真是这样,那也是很容易检验出来的。
回顾一下德布罗意关系。从该关系式中可以看出,随着电子速度的增加,它的波长应当缩短。物理学家已经知道:X射线越硬,它的波长越短,衍射图样也就越紧凑。对不同速

•79•
  
度电子的衍射也进行了研究。实验也同样肯定地证实:随着电子速度的增加,衍射环相应地变得更为紧密。
这样物理学家便能根据波长求出衍射环之间的距离,反之亦然。计算表明:如果根据环间距离计剪出电子波长,结果与从德布罗意关系中得出的数值准确地符合。
,没有任何怀疑的余地了c几率波即德布罗意曾经预言的物质波。这种波不仅仅当电子受到晶体的衍射时才出现,它是普遍存在着的。在任何时候这种波都是与电子或其他实物粒子的运动联系在一起的。`
但发现这种波并不总是可能的。随着粒子质量和速度的增加,德布罗意波的波长将要很快地减小,以致超出了我们仪器的灵敏度。在这种情况下,粒子只能呈现其微粒特性。
勹回顾一下我们对波动特性的讨论。波(例如电磁波),在其波长没钉小到一定限度以前,不显示任何微粒特性,而只显示波动特性,如于涉、衍射等等。但当波长变得足够短时,它们便开始象粒子一样池运动,如能将金属中的电子轰击出来。
	最好的例子是丫射线	一种在所有已知电磁波中最短
的波。这种波能够很容易地敲击出实物粒子,显示出真正的微粒特性。
德布罗意的发现将世界上所有的物理现象结合于一个统一的整体中,这样就在两个对立的、看起来甚至好像是互相排斥的存在-一-微粒与波——之间架起了一座桥梁。统一性被发现了,可是没有根据认为对立面已经消失0)

_.80急
  
对立似乎深深地隐藏在事物之中,决定着侬观世界的奇妙特征。下面我们将要详细地讨论这个世界。我们将要知道许多奇迹般的事情,这些事情在微观世界中是可能的,并且是可以用“儿率波”阐明清楚的。

\section{探讨波定律}
几率波描述微观世界中电子和其他粒子的运动。现在不妨问一下,应当怎样理解描述一语?对一件事物或一个现象,既可作定性的,又可作定量的描述。日常生活中我们经常只作前者。当我们听说今天要下雨,我们就会拿雨伞,一般不去问,云在什么高度出现。
可是科学,尤其象物理学那样的精密科学,是很难满足于一个定性的描述的。需要的是数据,而且是准确的数据。
上面我们只以定性的方式描述了照相底片上的衍射图样,如明睹相间的环。我们也可以对它进行定量描述,如量度出底片不同地方的暗度,并绘制一条曲线,正象我们在靶场上绘制的那祥。
不要以为我们能够对这个现象提出理论,便万事大吉了。
这里还有其他间题须要解释。科学不能孤立地为每个现象建立理论。
事实上,现代科学的力量恰恰在于:它建立的理论能包
	括几盲示面五云云面五五广谅存荷勹百广--—	—---
______,_--
	、	力最的理论正是那
	一	一,一--...'"""'"""''"一一剿	~
_应用范围最广、包括的现象最多的理论。


  
在物理学中建立新的、内容广泛的理论,经常是以寻找一个亟要的公式开始的。这个公式就叫作运动定律。人们熟悉的牛顿第二定律就是这样一个公式,它把物休的加速度与作用于物休的力的大小和方向联系起来了。但实际上我们看不见这个力和物体所获的加速度。我们能够观察到的只是在力的作用下物休在空间与时间中的移动。但正是牛顿的定律使我们认识了这个运动。加速度是运动的速度对时间的变化。而速度是物体的位置对时间的变化。归根结底,牛顿的定律将力与物体的实际移动联系在一起了。因此,解出牛顿方程我们便可得到物体运动的类型。这可以用物体的位置相对于时间的某种曲线来表示,这个曲线就叫作轨道。
物理学中还有一个很普遍、很广泛的定律,它描述的不是物体的运动而是波的传播。如果用数学形式来表达,那就是所谓波动方程或以发现它的十八世纪法国著名数学家命名的达兰贝尔方程。
无论牛顿方程或达兰贝尔方程都不是从更普遍的定律推导出来的。这些方程也并非凭空膛造出来的。牛顿和达兰贝尔的先辈已经作了大量的实验和观察,并作了理论上的概括。牛顿和达兰贝尔方程正是从这些理论概括中提炼出来的。
天才并不是一个只会搜索枯肠、凭空腌想的人。天才必须能够透过错综复杂的事物看到某种隐藏着的力撬,某种定律,从而将它从一些累赘的形式中,从一些偶然的、无足轻重的、枝离破碎的现象中解放出来;然后还要切磋琢磨,使之精练而有条理,在精密科学中这就成了公式。这个新定律就像

•82•
  
是一颗知识明珠:棱角分明,面面光润。
但什么定律够得上是量子力学大厦的奠基石呢?当然,蜇子力学的新的定律既想取代经典物理学中的牛顿和达兰贝尔定律,就必须具有至少是相等的普遍性和概括性。不仅如此,这一新的定律还必须独自描述具有二象性的微观世界,从而代替早先的两个定律3也就是说,这个定律必须同时描述粒子的运动和波的传播。
不管怎样说,牛顿当时的事情好办得多。他拥有极其丰富的实验事实,而此时此刻却没有一项实验作依据。时间是1925年,大约在具有决定意义的电子衍射实验前三年。摆在面前的只有一个德布罗意关系,可是它只描述了粒子的波长,而没有涉及到粒子的运动。
虽然如此,理论物理学家坚信他们走的是正确的道路:不去等待任何实验来证实德布罗意的假设,就着手于建立新理论。
可能要从改变牛顿方程开始,以便让粒子的波特性也被包括进来?否。历史朝另一条道走去。继德布罗意之后,物理学家都在致力于改变波方程,使其同时也能反映波的微粒特性。这种作法证明是比较简单的。
在这方面,薛定谔和海森堡首先取得成就。两人的方法颇不相同。此外,这个人可能还不知道那个人在于些什么。只是过了一些时候,两人的论文都已发表,薛定谔才能证明:对间题的两种解答在物理意义上是完全相同的,尽管从外形看来二者亳无共同之处。

•83•
  
闷森堡发明了所谓量子力学的矩阵表述。这种数学形式非常复杂,远远超出本书的尥围。另一方面,薛定谔改变了波动方程,这样就使德布罗意波的微粒味道也被包括进来。这个新的方程就叫作薛定谔方程,它是量子力学的最常见的公式。

这样,波定律便成了掀子力学的基本定律。

\section{轮到测量仪器了}

现在让我们再回到德布罗意波上来。根据玻恩的解释以及薛定谔方程的最终形式,这种波是通过照相底片上电子击痕的波浪形分布显示出来的。但我们知道,必须有大量的电子才能产生一个清晰的图样。
对于单个的电子,德布罗意波又有什么意义呢?我们同.样也知道:对于一个电子,这种波也要使之偏离经典的轨道。
如果没有偏离,也就不会有任何衍射图样。
事情看来是很清楚的。但有些地方还不能令人十分满意。对奇异的微观世界谈论了那么多,自然会认为粒子的波特性总有某些不寻常的地方。
好,让我们看看效观世界自己有什么可说的。假如我们想作一个测最。至于测量仪器是什么特殊类型的,我们井不感到兴趣。但它必须能够监视着电子的行踪:它必须能测出电子在每一瞬时的速度和位置。
电子是个很小的粒子。它必须要用一个超放大率的显微

•84•
  
镜来观察。设想我们确实制成了一台具有足够的放大能力的显微镜。第一个问题是:我们如何进行这项测晕?为了看见一个物体,必须用某种方法对它进行照明。问题是:用什么方法?照明依物体的体积而定。为了获得一个清晰的象,第一个条件是,光线的波长必须小于物体的线度。一般的光学显微镜在0.4至0.8微米波长范围内工作,因而物体至少有2至3微米的线度,才能产生轮廓分朋的象。
但如果物体的线度只有半掀米,所成的象将楼糊不清。
当物体的线度与光线的波长属于同一数匾级时,光线就会发生强烈的衍射。我们获得的将不是一个清楚的物象,而是一个衍射阳样——一组复现物体轮廓的明暗交替的带。
如果物体面积再小一些,则光线就会畅通无阻地过去,好像这个物体并不存在似的。
电子不是一粒灰尘,也不是一个细菌。它的体积(往后将看到,体积这个词是很难迅用的)大约要较光的波长小十亿倍。因此,我们怎能对它进行照明呢?幸好还有T射线,它的波长也是极短的。
这样我们便用r射线对电子照明并进行观察。可是我们仍然什么也看不见。确实一无所见。原来显微镜下还有个电子,现在这个电子却不见了。甚至连衍射环也没有看到。
不论我们如何设法去制一个电子的物象,我们也永远不会成功。
关键在于电子不是一粒灰尘,而丫量子也不是一个可见光的光子)极小的一粒灰尘也有重量,而一个光子携带着一

•85•
  
些能Ji,因此也具有一些动量。
光子的动篮是从哪里获得的?我们知道光子具有粒子的特性。爱囡斯坦在他的光电效应理论中巳经证实了这一点。你不妨没想一下:真空中的光子总具有一样的速度,那就是光速,但它的波长却是不一样的。现在将光子代人德布罗意关系中:
l=f!_mv

令速度0等于光速,这样我们便获得光子的质量(这当然是光子的动质量,光子的静质量绝对等于零):
力
,n=-
入c

光子的动屉笘于其质且与速度的乘积:k
p=me=-
儿
现在只须作一点数学运算。从这个公式可以立即看出,随若光子波长的减小,它的动篮迅速增大。
当光子射到一粒灰尘上时)它将自己的一部分动量传递给后者并反跳回来,射入显微镜的光学系统,最后进入观察者的眼睛。可是这粒灰尘本身却纹丝不动。如果这粒灰尘是静止的,它仍然会是这样的;如果它是运动着的,它的运动方向几乎亳无变化。
但电子却完全不一样了。它的质量无法和一粒灰尘相比较;即使是一个速度很高的电子,它的动量也很小。现在让我


克,动拭为10-15克·厘米/秒。现在将波长等于0.5微米的光线(即可见光谱内的绿光)照射其上。这种光的光子的动量10-22克·厘米/秒,也就是说要较该尘粒的动量小几千万倍。十分明显,光子射在尘粒上是不会产生任何效应的。
再以电子为例。即使它的速度接近光速,也就是它达到了I010厘米/秒的速度,它的动量也仅约I0-11克·厘米/秒。用来照明的r光子具有极短的波长(6X10一13厘米),它的动量10一14克·厘米/秒,要比电子的大几千倍。因此当T光子射到电子上时,就像列车撞上了婴儿车一样。
这点现在应该很清楚了:获得对微观世界进行观测的仪避煦可拒性3至少可以说是非常小的。这些仪器不能以任何准确度来测量粒子的运动。
这种不精确性,或者说得更妥当些,测量的不准确性,又达到什么程度呢?这个问题要由海森堡1927年从量子力学的普遍定律推导出来的测不准关系来回答。下面就是这个测不准关系

~xX~i,x)!-·h
m

(实际上h应当由h/2元来表示,但这点无关紧要,因为二者只相差六倍)。这里~x是对粒子的位置(坐标)x的测不准撞;心X是对其在r方向上的速度11,,的测不准量;m为粒子的侦量;符号>表示两个不准横的乘积不得小于关系式右边的量。

这里怪事来了。如果我们将粒子的位置测量得绝对准

•88•
  
确,其坐标的不准鼠Ax当然必须变为零。这样根据严格的数学规律,速度的不准趾变为:
竺=正=沪=00
	Ax	0

——即无限大。这就意味着,在粒子位置正被测定的那一瞬,它的速度变成绝对地不确定的。相反地,如果在某一瞬时我们以绝对的准确度测得了粒子的速度,则我们无法判断这个粒子在此瞬时的位置。
怎么办?做点折衷一一如以某种总地说来并不太大的误差同时视鼠电子的位置和速度一一或许是可以的吧?
让我们看看测量上述尘粒和电子的误差究竟有多大。对前者来说,海森堡测不准关系的右侧的数量约为10一15。现在让两个不准批都取折衷数值:b.x=10一8厘米,心u飞=10刁厘米/秒(二者相乘得右边数匮一一10一15)。
凶x对u丈之比为10一7:10一4=10飞即于分之一。对速度的测量来说,这点汉差应该使我们满意了。绝少的速度计能有更高的精确度。
再看看尘粒位置的不准量,工。它对粒子线度之比为10一8:10一4=10飞即万分之一。这一误差相当于尘粒中一个原子的线度。
这就说明,当我们测量一粒灰尘或一些更重的物体的速度和位置时,我们甚全从未想到还存在着一个测不准关系。
可是电子的情形郘不一样了。它的休积(这里我们再次指出,以纾典为理、产的观点谈电子的体积是不很正确的,因为

•894
  
这样电子就被石成一个带电的球体了)大约为10-13厘米*,质屈to-2,克,1伏特电位差将使电子获得数擢级为107厘米/秒的中等速度。测不准关系右侧约为10。
也彴其化方法犹出Ax和心}入的数伯。设我们要求以视lj堪尘粒的准确度—--10一3、阑量电子速度,则在此情况下,凶.飞:107=10飞即电子速度的不准呈Avx=104厘米/秒,以此值代入测不准关系,得出Ax~lO一3厘米。这就是说电子位四的不准屈为其线度的几百亿倍!
没测电子速度的不准确程度为100%-—物理学家认为实际剧墙悄况就是这样——则Av_.=107厘米/秒,而心=o气匝米e这一数值仍然较电子的线度大几百万倍。
这样石来折衷的办法是行不通的。微观世界不接受这样的办法。

\section{过错在仪器还在电子}

经典物理学从未经历过这种进退维谷的处境。它认为在任何瞬时都能以绝对的准确度(至少在理论上)测出任何粒子的位置和速度。这种观念是根深蒂固的:经典物理学对质点运动的计笳总是建立在某开始时刻质点的位置和速度这样的基础之上的。
现在我们发现:甚至在理论上也根本谈不上什么测量的

*指线度。一一译者注

•90•
  
绝对精确。毛病出在哪里呢?可能在仪器吧?
当然,任何仪器也不可能以绝对的精确度来测址一个量。
我们可以说,测量技术的发展史,就是仪器精确度不断提高的历史。在许多科学技术领域内,精密仪器达到了难以想象的高水平,而且它们还在不断地改进。
这样看来,测不准关系似乎已经为仪器的精确度定了个
极限-—-一个上限。
在这个难题面前,海森堡(以及追随他的其他物理学家*)认为:困难出自仪器。观测微观世界的仪器异于用来探索宇宙的望远镜。当然,两种仪器都是需要的。我们用来观察周围世界的感觉器官具有局限性。事实上仪器的用处就在于:它把在它的范匝内的现象翻译成“人类的“感觉的语言。
望远镜观祭天体时,它一点也不影响天体的运动。可是对微观世界来说,情况则完全两样。我~]的仪器(即使是理想的超显叙镜)也会直接干扰被观察的现象,并改变它的本来过桯。不仅如此,这种于扰是如此地严重,以致我们没有办法以纯粹的形式将这一现象分析出来。这就是测不准关系的意义所在:它为观察的“纯粹“程度定了个上限。
另外一些物理学家持有相反的见解:毛病在电子本身。
他们的论点是相当有说服力的。微观世界具有它自己的规

噙即所诮哥本哈根学派,主耍成从有玻尔(长期在丹麦首都哥本哈根工作),和德国的海森凭、玻恩,英国的狄拉克等。他们强调观测者的作用,认为侬观过程是和因果性不相容的,因而属于唯心主义学派。他们之中有些已开始放弃原来的某些观点,如玻尔就是其中之一。一一译者注

•91•
  
律;一般说来,微观世界并没有要求对其存在进行测匿。当我们说电子具有波特性时,这句话又意味着什么呢?
就拿一个摆的振动频率作例子吧。如果说在某一瞬时它的频率是如何如何,则这简直是在说庆话。为了确定频率,必须对权的振动观察一段时间。同理,我们也不能说在某给定点上波长如何如何。波长的意义恰恰在于,它是一长(严格说来它是七限长)系列的波的一个特征。不管波的特性如何,波的长度不以波中某一点的位置而定。
让我们再以德布罗意关系为例,但以另一形式写出,使关系式的左侧为粒子的速度:

,,
V=--
m入

我们立即得出结论:由于波长入不依赖于波中任何一点(例如,我们相信粒子所在的那一点)的位置,因而粒子的速度也就不能依赖于拉子的位置。仪器的失误恰恰是电子的这种波特性造成的。
倒底是谁对呢?是指责仪器不能适应微观世界的这一派对呢?还是怪罪微观世界根本就不能浏量的那一派对呢?
看来似乎二者都应该受到指责,并各受五十板。间题的实质是:海森堡关系揭示了仪器和电子两方面的过失。但问题还不止于此。

•92•
  
\section{旧观念枝引进新世界中}

我们对仪器有什么爱求?首先,它应当提供我们需要的信队。当然,仪器没们任何自一七性,它只能服从于人的意志。
我们用来观测微观旧界的仪器具有质个方面,或两端:输入端和输出端。在轮入端,仪器对付那些遵从凰子规律的现象;在输出端,它提供用经典语言记录下来的信息,因为我们的感觉器官不懂得任何其他语言。
我们要求仪器告诉我们在每一瞬时电子的位置和速度。
仪器很老实地回答说它办不到。它说:在测量速度时,如果不须指明位置,它便能提供关于速度的信息;反过来,如果在该瞬时可以不去过问速度,它也可以提供有关位置的信息。
只须稍稍思考一下就会朋白:最该受到责备的还是物理学家自己3他们要求仪器提出电子的速度对位置的信息,事实上这两个擞并无任何关系。
这就是微观世界的许多奇迹中的一个奇迹,也就是粒子的波动本性的各种表现中的一种表现。一句话,物理学家几百年来运用自如的、老的经典概念和匿,在处理微观世界时不复有效。
当然,也并非就是无效。在微观世界中,这些概念仍然保存着,但在某种意义上受到约束和限制。这些概念在使用上的限制是用测不准关系来表达的。
其结果,如果电子是静止着的,它的波将无限伸延,因此

•_93•
  
在任何一个确定位置上找到它的企图都将失败。另一方面,电子运动越快,它便能越准确地被定位于其波中。但即使以

 

 

化。
当然,你也可能间,为什么物理学家不抛弃那些不能正常地应用于微观世界的旧经典概念和数量,而代之以更加符合于这个世界的不寻常特性的新概念和数量?
很难一下子就看出这个问题是多么地复杂,因为它牵涉到人类认识过程的根本性质。在本书的末尾,我们还要再次讨论这个问题。目前必须说明:在物理学或其他任何一门科学中,概念或观念的任何改变都是一个十分漫长的、复杂的、艰苦的过程。人类关于宇宙、生命和无生物的本质以及原子构造的最初的幼稚的想法,经过好几千年的时间才发生一次变革。可以容易地想象,几百年以后,我们自己的种种概念对后人来说也将显得多么幼稚。
在我们这个时代,人类的知识在以神奇的速度发展着。
尽管如此,揭示新世界、新现象的过程已经落在后面了:它而临着越来越多的困难和矛盾。爱因斯坦讲得很恰当:这是一场观念的戏剧。
当经典概念被引进微观世界中时,我们看到的正是这么一场戏。

•94•
  
\section{另一个奇迹}

儿巾翩越围墙溜进缨桃园,)园主筑起·逍更高的围墙。
小约翰怎么办?他采用了跑跳,或者取来一个梯子,或者爬上一棵树然后再翻越过去,...…当然办法是很多的。今天孩子们已经不再相信什么神话了,但如果他们真的要和效观世界打交逍的话,他们必须想象粒子能从一堵坚实的围墙钻过去!
让我们更仔细地研究一下爬墙或跳墙这件事。我们从学校里学得:身休越低,位能越小。站在地面上时,你的位能要比坐在墙上时小。而且我们知道小多少:这个数星就是我们的体壑芍在两个不同位置上身体重心高度之差的乘积;这甫心高度差粗略地等于墙的高度减去1米。
如果你从什么地方获得这个位能,你便能翻过这堵墙。
比如,你可以使用自己的肌肉,或你伙伴的肌肉,把你抛起来。但两种办法中不论用的是哪一种,所作的功都用来增加位能,以便使你越过围墙。
余下的就好办了。跳下来不须作任何努力。相反地,倒须设法缓和一下引力对降落造成的冲击己当你在墙的另外-侧着陆后,你的位能将降低到跳跃前的水平。
如果我们将跳墙前后的位能绘成一张图,我们将获得一个凸起。在物理学中,这就叫作位垒。
原子世界中也有一些这种性质的围墙。比如,一块金属包含大星的几乎是自由的电子,人们从未听说电子能凭借自己的自由意志脱离金属。关键是,它们井不是完全自由的。虽然离子对电子的束缚是弱的,但电子仍然被这样形成的离子吸引着(这点将在下章较详细地讨论)。一块金属中,所有离子对所有电子的总效应可以描绘成一个借高墙与外界隔绝的庭院,在庭院里驰骋着电子。
一块金属内的电子与洞里的球相似,这个球在讲玻尔理论时曾讨论过。在金属内部,电子无规律地运动着,但它们逃跑的可能,并不比球从洞里出来的可能大一些。由于这个原因,电子在金属内的处境被取名为位阱。
可是这些电子并非永远地被金属真地拴住了。在某种条件下,电子可以说能够越墙而逃。例如,当金属受到波长足够短的光线的照射时,这样的事情就会发生。一个高能光子能够敲击电子,使它翻越位垒,获得真正的自由。这是用传统的、经典的方法翩越位垒,与小孩翻越围墙没有两样。
但你可能已经看出,对于金属中的电子来说,位垒并不真象一堵墙,因为它有前壁而无后壁。与其说它象堵墙,不如说它象一级台阶(图5)。对于阱中的球来说,也可以沿着阱的



图7

屯子在无端地嘲井治经典物理学的种种预言;它仿佛能渗透位垒,达到伎界3似乎某种神秘的力楹打通了一个穿透位垒的隧迅,让屯子从中通过。为这个奇妙现象,物理学家取了一个名称一隧逍效应。

§拓再看看测不准原理

在我们等待屈子力学对这个新的奇迹作出解释的同时,我们的澜星仪器一直在工作着,虽然成绩令人很失望。我们要求仪器察看电子是怎样渗透位垒的,因为这种现象违反了经典力学的最根本的法则。我们希望能够证实这无非是理沦上的荒诞不经。
我们巳经提到过阱里的球的总能最等于它的动能与位能之和,而且是负的。这是因为球的位能(从阱的顶部来计'5f.'亦即从位垒的最高点来计算)是负的而且超过(在数值上)球

•98•
  
的动能3很清楚,在位垒的垒壁内,球的总能星也应该是负的,因为在渗出过程中总能量数值并不改变。但在另一方面,球的位能绝对值随着球趋近位垒的最高点而减小,达到最高点时变为专。
唯一的结论是:在位垒叩(内,球的动能成为负**的。但这是一种什么样的数昼呢?让我们将动能氐写下来:

EK
mv
=--
2
不管速度U的符号如何,它的平方总是正的,分母中的2也是正的。这就意味着粒子的质量m是负的。但不管在经典力学中也笃在最革命的量子力学中也罢,负质量总是不可思议的。你设想一列火车从莫斯科开往列宁格勒,可是该列车的车厢却与火车头背道而驰,从列宁格勒开往莫斯科!
简直是奇谈怪论!为了证实这点,我们设置了仪器,对电子进行观察。仪器巳经发现了这个电子,并已开始观察。观察的情况如下。电子接近位垒的边界。为了在电子正在穿越位垒时捕获它,甚至不需要确定其位置3只须查明这个电子的位置确实在位垒之内。
但事情不只如此。仪器必须测出电子在该瞬间的速度,以便决定它的动能是否真地变成了负的。但这里仪器又无能

*指的起位垒的垒璧内,即隧逍内。注意位垒和位阱的区别。--一译者注**请注恁图7虚线所示隧道位能与外界零位能相通情况,这样隧道内的位能便应为零,但此时球的总能阰仍为负的,因而导致球的动能成为负的。
一一-诈者注

•99•
  
为力了。只有海森堡测不准关系能挽救这个困局。
现在为了确定电子在位垒中的位置,必须用短波光子对它照明,这是因为测定电子位置的准确度不得小于位垒本身的宽度而在原子世界中这个宽度是很小的。但当光子碰撞电子时,会将一个相当可观的不准量引进其速度中,这个不准蜇将造成电子动能的不准量,后者将恰好足够使电子越过位垒的最高点。
换言之,没有任何方法能观测到正在穿过位垒内的非经典通道的电子。因为就在对它进行观测的过程中,一定的能最又巳经传递给电子,而这能量足够使后者以完全合法而经典的方式跃过位垒。这有点象是警察在帮助犯人掩盖罪证。
以上的讨论典型地代表着微观世界中发生的许多事情。
从经典物理学的观点来看,量子力学自以为是地坚持一些最最离奇的事情。使用经典仪器来证明这些主张的虚妄是根本不可能的。不要在位垒之中去寻找粒子吧,你不会找到它的。粒子在位垒内这一概念在量子力学中,正如在经典力学中一样,是荒诞的。
但粒子能透过位垒!这个神秘的线索最终还应追溯到微观世界中电子和其他粒子的波特性。

\section{再谈谈物质波}

正如我们已经看到的那样,波动特性的结果是:粒子速度不再依赖于粒子位置。微观世界里根本不存在着什么总动

•100•
  
轨道。但粒子的位置影响着它的位能,粒子的速度影响着它的动能。
因此严格地说来,在同一瞬时既准确地测得粒子的动能

又准确地测得其位能是不可能的。在任何给定的一瞬间,二者是彼此独立的。在这些可以应用经典概念的范围以内,原子世界中的不同能隘是用测不准关系来表达的。
这就意味着,位阱里的粒子具有全靠自身逸出阱外的某种几率。这也还意味着,粒子仍然留在阱内的几率也存在着。假如说在1000个电子中,有10个逸出位垒,则隧道效应的几率为1%,而无隧道效应的几率为99%。
这两种几率分别叫作位垒的可渗透性和反射能力。
可渗透性(或透明度)和反射能力是熟悉的名词。这些词描述不同物质对于光波通过时所呈现的性能。在两种不同物质的界面上,光总是能够部分地进入第二种媒质,部分地被反射。难逍位垒不就是两种媒质之间的分界吗?只是位垒不是对电磁波(包括光波),而是对德布罗意波来说的。
这是一个非常深刻的类比。隧道效应很好地符合于在两种不同物质界面上光线的反射和折射定律。
我们之所以选择一道围墙,即某种具有确切的、有限的厚度的东西,来表征我们的位垒,并不是偶然的。如果这个位垒只有前壁,象一级台阶那样,隧道效应便将全然梢失。在无限厚的即使很低的位垒中,粒子不可能打通隧道。这里经典物理学的禁令是完全钉效的。
假使在无限厚的位垒中隧逍真地打通了,仪器将能庆贺

•101•
  
某种胜利:它将有绝对的把握确定粒子确实已经进入位垒(如果真是这样的话),不管对其位置的测蜇的不准量有多么大。这就是说,测不准关系将能报告粒子的准确速度,也就是粒子的动能。这个能矗将肯定是负的。
但自然界不会自相矛盾。负动能是不可能的。因此,无限长的隧逍效应也就不能发生。
当然,上述解释也可能说服不了某些读者。我们所说的每件事难逍只是抽象的理论思维吗?这点只好由你自己去作出判断了。大噩电子会离开一块加热了的金属薄片——热能将足以使它们越过金属表面上的位垒,但对于一块冷的金属,这种现象决不会发生。
将这块金属置于强电场中,大量电子又将从金属中涌出。
这就叫冷发射。它奇迹般地证实:隧道效应并非物理学家的脖造。

\section{波函数}
方程的建立并不是为了好玩,而是为了求解。薛定谔方程也不例外。方程有时很简单,有时很复杂。而薛定谔方程则确实是很复杂的。它被称为二阶偏微分方程。解释这个方程的意义远远超出本书范围。这里只提一下:这类方程被用来描述在空间与时间中变化着的诸屈。
这类方程中的未知批可能用种种形式乔装打扮起来:可以是容器内的液体的表面形状,可以是太空中某卫星的座标,

•102•
  
可以是从发射器向接受器运动着的无线电信号的强度,也可以是机床切削速度,等等。这样的方程的解直接地给出所求
	--	-·

址对科学家感兴趣的其他诸批的依赖关系。数学家用函数一词表述所有这些关系。
薛定谔方程中的未知匿就叫作波函数。尽管从波函数中做出了数以于计的精确计鲜,它的确切意义对科学家来说仍然不消楚。我们已经说过,科学家至今还在对这个函数议论纷纷。
但在一件事上,他们似乎意见一致:波函数的平方具有几率的意义。它对座标和时间的依赖关系提供了在指定时间和指定地方发现一个粒子的几率。更确切地说,它提供了这祥的几率,即在某一瞬时粒子在指定位置根据它在该处的行为而被发现的几率。例如,这种行为可以是粒子与我们的测扯仪器之间的相互作用。这个几率就是我们在电子衍射实验中所描述的几率波。
解一般形式的薛定谔方程是个极其艰难的课题,即使我们用的是最巧妙的现代数学方法。但存在着许多场合*,它们能使这个解变得容易。这就是所谓的定态间题,在这些问题中,所求的波函数只相对于某确定的平均形式而振动,而此形式本身并不随时间变化。
可以很容易地看出,这类问题不涉及过程(当然是非周期性的)。在过程中事情是被控制着的,并在时间中变化。因此

*这里所说的“场合.,,即在原子寸趴分子、固体笘场合下,方程具有牡解。
--·译者注
•103•
  
定态间题指的是系统的结构,在该系统中过程能够发生。了解系统的结构是十分重要的,因为如果不知道过程是在什么条件下发生的,便不能对这个过程作出任何描述。
在微观世界内,构成这些条件的耍素包括原子核、原子、分子、晶体以及其他许多东西。我们知道它们都具有特别稳定的结构。定态薛定谔方程首先应用于上述诸要素中,得出的结果是极为有趣的。这些,我们将在下章中讨论。

\section{波和量子结合起来了}

昼子力学中的定态问题还有另一个显著特点。为要了解这一点,我们须回顾一下:测不准关系不仅包含粒子的位置和速度,而且也包含它的总能址和时间。
在后一种情况中,海森堡关系表明:测噩的时间越长,粒子的总能匿就测得越准确3这个关系式的形式与早先所给的那个非常相似:
AEXAt~h
(同样,式中h如写成h/2rc则更正确些。)这里AE是粒子能量E的不准噩,心是对粒子具有准确能量E这一瞬时t而言的不准量。符号>表明两个不准量的乘积不得小于普朗克常数从
所谓定态就是说粒子的能量不随时间而变化。因此在原则上我们可以无尽期地对它测量。这里量度时间的不确定不起任何作用。

当E变为负值时(这在前面已经说过,是与粒子的被束缚状态相对应的;例如,位于阱中的球和原子内的电子),方程的解发生了根本性质的变化。在这种情况下,它只对能擞E的某些特殊值来说不等于零。
E的这些数值就叫作粒子的容许能级。看一看图8的左边。除了一些容许能最状态而外,粒子存在的几率几乎处处接近于零。在容许能监状态中,粒子存在的几率显著地异于翠物理学家给这种情况取名为能级的分立性o

现在更仔细地观察一下这个图。难道它不有些象玻尔的原于秧型中的容许能级吗?当然有些象。不仅如此,二者实际上是一回扎玻尔的电子轨道正是这些能盘状态,在这些状态中电子存在的几率显著地异于零。
不同的地方是;玻尔只设想出这一轨道,但他不能证明这些轨追为什么会存在。正是量子力学枯成了这个假设的基础。
坠子力学也证实r玻尔的第二个假设:原子中电子的跳跃具有昼子性质。从薛定谔方程可以看出,原子中的电子只能在容许能嚣状态下存在。这就是说,当电子从一种状态过渡到另一种状态时,能凰不能任意改变,而必须取一些特定数值。这个数值恰好等于电子跳跃或过渡前后的两种状态的能操差c
这个能呈差就是为这一新物理学开辟道路的普朗克量子。屈子力学将两个杰出的假设一-普朗克关于能量子的假设和德布罗意关于物质波的假设-—-结合在一起,并揭示了

、~106•
  
二者之间的紧密的相互关系。
耍是没有德布罗意波*,就不会有普朗克蜇子。这样,两股小溪就汇成了一条巨大的知识河流3让我们沿籽这条逐渐扩大的河流走下丿《吧,看行什么样的新奇景物将在限前展现,

*这里诣的是物屈波本身的存在,而不是物质波理论的提出,因)J后者要较普朗克笸子假设的提出晚25年,参看\section{7。__:译者注}

